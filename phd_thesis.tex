\documentclass[phd,tocprelim]{cornell}

%Some possible packages to include
\RequirePackage{fix-cm}
\usepackage{graphicx}
\usepackage{pstricks}
\usepackage{amsmath}
\usepackage{graphics}
%\usepackage{moreverb}
%\usepackage{subfigure}
\usepackage{epsfig}
%\usepackage{hangcaption}
\usepackage{txfonts}
\usepackage{palatino}
\usepackage{booktabs}
\usepackage{cleveref}
\usepackage{subfig}
%\usepackage{afterpage}
\setcounter{tocdepth}{4}
\setcounter{secnumdepth}{4}
\newcommand{\myparagraph}[1]{\paragraph{#1}\mbox{}\\\mbox{}\\}
\Crefname{paragraph}{section}{sections}
\Crefname{paragraph}{Section}{Sections}
%\usepackage{setspace}

%if you're having problems with overfull boxes, you may need to increase
%the tolerance to 9999
\tolerance=9999

\bibliographystyle{plain}
%\bibliographystyle{IEEEbib}

%\renewcommand{\caption}[1]{\singlespacing\hangcaption{#1}\normalspacing}
\renewcommand{\topfraction}{0.85}
\renewcommand{\textfraction}{0.1}
\renewcommand{\floatpagefraction}{0.75}

\title {Analysis of Low-Frequency Climate Variability Through 
Computational Modeling and Tree-Ring Data Synthesis}
\author {Brandon Norton Benton}
\conferraldate {December}{2019}
\degreefield {Ph.D.}
\copyrightholder{Brandon Norton Benton}
\copyrightyear{2019}

\begin{document}

\maketitle
\makecopyright

\begin{abstract}
  This work looks at low-frequency variability with new tools that 
  give us unprecedented insight into decadal and centennial timescales. 
  First, thermodynamic and dynamic effects of volcanic eruptions on hurricane
  statistics are examined using two long simulations from the
  Community Earth System Model (CESM) Last Millennium Ensemble
  (LME). The first is an unforced control simulation, wherein all
  boundary conditions were held constant at their 850 CE values. The
  second is a ``fully forced'' simulation with time evolving radiative
  changes from solar, volcanic, solar, and land use changes from 850
  through present. The largest magnitude radiative forcings during
  this time period are the large tropical volcanic eruptions, which
  comprise the focus of this study. Potential and simulated hurricane
  statistics are computed from both the control and forced
  simulations. Potential Intensity is evaluated using model output at
  its native (nominally 2 degree lat/long) spatial resolution, while
  the weather research and forecasting (WRF) model is used for
  dynamically downscaling a total of 100 control years and an
  additional 100 years following the largest volcanic eruptions in the
  fully forced simulation.  Limitations of the downscaling methodology
  are examined by applying the same approach to historical ERAI
  reanalysis data and comparing the downscaled storm tracks and
  intensities to the IBTrACS database. Results suggest small effects
  are observed in averages over all last millennium eruptions which
  are non-significant in comparison to the control. However, for many
  of the major eruptions, significant reductions are seen in hurricane
  frequency, intensity, and lifetime. Strong evidence is also shown
  for correlation between eruption strength and changes in these
  diagnostics.
\par
  Second, we present preliminary efforts to synthesize raw tree-ring data
  into comprehensive paleoclimate data sets, to detrend
  this data using a suite of detrending models, and to analyze the
  resulting chronologies. The methodology developed uses four primary types 
  of detrending models to construct tree-ring chronologies using 
  data from the International Tree Ring Database (ITRDB). The detrending 
  models use varying combinations of splines, negative exponential 
  functions, tree-ring segment length constraints, and variance 
  thresholds. These combinations range from less to more aggressive (i.e. 
  filtering variance and segment length requirements)
  in constraints on tree-ring segment properties and in preserving 
  low-frequency content. Information encoded in trees reflects a 
  combination of biological effects on long timescales and climate effects 
  on shorter ones. Detrending is necessary to remove these biological 
  effects. Analysis of chronologies is made possible using a 
  combination of multiple-taper spectrum estimation 
  methods (MTM) and principal-components analysis using singular-value 
  decomposition (SVD). The MTM-SVD approach is selected in order to 
  overcome the estimation bias inherent in Fourier analysis and because 
  of the large-scale spatial structure of climatic variations. This 
  MTM-SVD analysis provides an approach for signal detection and 
  reconstruction, along with significance assessment. A robust null 
  hypothesis is used to determine significance of signals in the local 
  fractional variance spectrum, derived from the set of singular values.
  The methodology presented will be used to explore the effect of detrending 
  schemes on climatology extracted from chronologies. It will also be used in 
  future work to quantify the amplitude of low-frequency 
  hydroclimate variability in models, proxies, and observations, 
  while at the same time utilizing an ensemble of last millennium 
  numerical climate models produced by the National Center for 
  Atmospheric Research (NCAR).  
\end{abstract}

\begin{biosketch}
Brandon Norton Benton completed his B.S. in Physics at Georgia 
Southern University. He graduated magna cum laude and completed 
the University Honors Program. While an undergraduate he worked with 
Dr. Mark A. Edwards doing research on Bose-Einstein condensates. 
His honors thesis was titled "Prototyping method for Bragg-type atom 
interferometers." This was published in Physical Review A. He also 
published "Approximate mean-field equations of motion for quasi-2D 
Bose-Einstein condensate systems", in Physical Review E, and 
"Momentum-space engineering of gaseous Bose-Einstein condensates" in 
Physical Review A. During his senior year he completed an internship at 
the National Institute of Science and Technology under Charles C. Clark.
He completed his M.S. in Physics at Cornell University under Steve Marschner
in the Computer Science Department. His thesis was titled "VR Quadcopter 
Telepresence Proposal." 
\end{biosketch}

\begin{dedication}
To Mom and Dad. For your unwaivering support through eons of education.
\end{dedication}

\begin{acknowledgements}
I am grateful for the significant amount of time and energy my advisor, 
Toby R. Ault, and my committee, Chris Myers and Eberhard Bodenschatz, 
have put into my research. My undergraduate advisor, Mark A. Edwards, 
played a huge role in kindling and nurturing my passion for scientific 
research. Doug James, Jane Wang, and Steve Marschner were instrumental
in helping guide me through the graduate program. Several former and 
current graduate students have assisted me in my research including 
Carlos Carillo, Dimitris Herrera, Marc Allesi, Xiaolu Li, and Colin Evans. 
Support for my research has come from NSF, AWS, the Cornell University 
Department of Physics, and the Cornell University Department of Earth 
and Atmospheric Sciences. 
\end{acknowledgements}

\contentspage
\tablelistpage
\figurelistpage

\normalspacing \setcounter{page}{1} \pagenumbering{arabic}
\pagestyle{cornell} \addtolength{\parskip}{0.5\baselineskip}

\chapter{Introduction}

\section{Motivation}

Although the underlying relationship between hurricanes, 
radiative forcing, and climate change remains an area of considerable debate
and active inquiry \cite{ting2015,elsner2006,msadek}, a number of 
modeling studies have suggested that, in general, future storms may pose even
more severe threats to human well-being, infrastructure, and the
economy \cite{IPCC2014c}. If global greenhouse gas reduction efforts 
fail or are insufficient in the coming decades, some researchers argue that
``solar radiation management'' (SRM) interventions to the climate system
may be preferable to allowing global temperatures to increase indefinitely 
\cite{govindasamy2000,caldeira2008,kravitz2014multi,macmartin2019}.
The ultimate effect of SRM is to decrease the total amount of sunlight 
reaching the surface, which is directly analogous to the effect of 
stratospheric aerosols from volcanic eruptions. Therefore volcanic 
eruptions of the recent past and last millennium may give us a glimpse 
of the risks associated with SRM due to changes in large-scale changes in 
circulation \cite{rasch2008}. The historical period only provides a few 
clues about the relationship between volcanic forcing and hurricanes, 
and much less about the effect of volcanic eruptions on hurricanes or TCs. 
Attempts to use paleoclimate indicators of past events (i.e., paleotempestology) 
are limited by the paucity of appropriate archives and proxies 
\cite{liu20081200,mann2009atlantic,donnelly2015}. These limitations preclude the
extensive use of such data to evaluate the effects of major volcanic
eruptions on TC activity \cite{oliva2018paleotempestology, 
yan2015tropical,korty2012variations}. This suggests numerical modelling
as the most fruitful direction to examine the connection between volcanic
eruptions and hurricanes. 

\par
Empirical measurements are essential in order to reconstruct the past
climate. The discipline concerned with reconstructing the climate from 
tree-ring measurements, dendroclimatology, is indispensable in this empirical 
reconstruction effort. Tree-ring chronologies are used to estimate the climate
back in time beyond the start of recorded meteorological measurements. 
These chronologies can be analyzed to assess long-term departures from
average climate, frequency of extreme climate, changes in interannual
variability in climate, and ranges of long-term variability in climate 
\cite{sheppard2010dendroclimatology}. In order to use ring-width measurements 
for climate reconstruction the raw ring-width measurements must 
be "standardized." This is the process of fitting a curve to the raw 
ring-width data, or detrending, and normalizing the raw ring-width data 
by the fitted curve. The fitted curve is selected to reflect the natural 
growth pattern of the ring-widths. This, in turn, removes the natural 
climate variability and biological processes from the raw data. An important 
question is whether the particular type of curve selected 
for detrending impacts the information contained in the final chronologies. 

%\section{Overview of Chapters}

%\section{Implications}

\chapter{Minor Impacts of Major Volcanic Eruptions in
  Dynamically-Downscaled Last Millennium Ensemble Data}\label{chap:vol}

\section{Abstract}
  Thermodynamic and dynamic effects of volcanic eruptions on hurricane
  statistics are examined using two long simulations from the
  Community Earth System Model (CESM) Last Millennium Ensemble
  (LME). The first is an unforced control simulation, wherein all
  boundary conditions were held constant at their 850 CE values. The
  second is a ``fully forced'' simulation with time evolving radiative
  changes from solar, volcanic, solar, and land use changes from 850
  through present. The largest magnitude radiative forcings during
  this time period are the large tropical volcanic eruptions, which
  comprise the focus of this study. Potential and simulated hurricane
  statistics are computed from both the control and forced
  simulations. Potential Intensity is evaluated using model output at
  its native (nominally 2 degree lat/long) spatial resolution, while
  the weather research and forecasting (WRF) model is used for
  dynamically downscaling a total of 100 control years and an
  additional 100 years following the largest volcanic eruptions in the
  fully forced simulation.  Limitations of the downscaling methodology
  are examined by applying the same approach to historical ERAI
  reanalysis data and comparing the downscaled storm tracks and
  intensities to the IBTrACS database. Results suggest small effects
  are observed in averages over all last millennium eruptions which
  are non-significant in comparison to the control. However, for many
  of the major eruptions, significant reductions are seen in hurricane
  frequency, intensity, and lifetime. Strong evidence is also shown
  for correlation between eruption strength and changes in these
  diagnostics.

\section{Introduction}\label{intro}
\par
\textbf{Hurricanes threaten human lives and livelihoods, inflict severe 
damage to property, and incur billions of dollars in economic losses and
recovery efforts.} These events alone caused $42\%$ of the
catastrophe-insured losses in the United States in the period
$1992-2011$ \cite{hodges2017well}. For example, in 2005 Hurricane Katrina
resulted in $1833$ casualties and a financial loss of over $\$125$
billion \cite{hodges2017well}. In 2012, Hurricane Sandy killed at least
233 people and caused $\$70$ billion in damage
\cite{blake2013tropical}. Moreover, in 2017 Hurricane Harvey displaced more
than $30,000$ people and resulted in $103$ deaths
\cite{murphy2018service}. The same year Hurricane Maria resulted in nearly
$3,000$ deaths \cite{maria_rpt}. Combined, Harvey and Maria caused
more than $\$200$ billion in damage \cite{murphy2018service,maria_rpt}. 
The disruption from these extreme weather events
will likely increase with rising coastal populations and increasing
value of infrastructure in coastal areas
\cite{kerry_tc_clim}. Furthermore, anthropogenic climate change is
expected to increase average sea surface temperatures (SSTs)
\cite{ipcc_2007} and sea level. Accordingly, there is growing interest
in determining if modifications to the incoming flux of solar
radiation could potentially offset key impacts expected to occur from
rising global temperature \cite{msadek}. %%irving2019hurricane 
Whether or not
such strategies are pursued, it is critical to understand the
relationship between hurricane statistics and climate responses to
past radiative forcings to help characterize the full
range of plausible future influences on hurricane activity in the
future.
\par

\textbf{The underlying relationship between hurricanes, radiative
  forcing, and climate change remains an area of considerable debate
  and active inquiry \cite{ting2015,elsner2006,msadek}}. Yet, a number of 
  modeling studies have suggested that, in general, future storms may pose even
more severe threats to human well-being, infrastructure, and the
economy \cite{IPCC2014c}. For example, \cite{villarini2013} and 
\cite{emanuel12219} used data from the Climate Model Intercomparison 
5 (CMIP5) ensemble to evaluate storm intensity during climate change and 
if their tracks were similar to those that unfolded over the 21st Century. 
Such studies suggest that, in general, the number and intensity of the largest
storms (e.g., category 4 and 5 hurricanes) will increase in a warmer
climate due, primarily, to increases in sea surface temperature
(SSTs). 

\textbf{If global greenhouse gas reduction efforts fail or are
  insufficient in the coming decades, some researchers argue that
  ``solar radiation management'' (SRM) interventions to the climate system
  may be preferable to allowing global temperatures to increase
  indefinitely \cite{govindasamy2000,caldeira2008,kravitz2014multi,
  macmartin2019}.} 
  Regardless of the
details of how SRM interventions are modeled, their ultimate effect is
to decrease the total amount of sunlight reaching the surface, which
is directly analogous to the effect of stratospheric aerosols from
volcanic eruptions. Therefore volcanic eruptions of the recent past
and last millennium may give us a glimpse of the risks associated with
SRM due to changes in large-scale changes in circulation \cite{rasch2008}.

\textbf{Volcanic eruptions reduce incoming solar surface radiation,
  which can strongly impact global temperatures, circulation patterns,
  and water cycles} For example, asymmetric
volcanic forcings (e.g., from volcanic aerosols being concentrated in
the stratosphere of one hemisphere) would alter the position of the
Inter Tropical Convergence Zone for at least one year following the
eruption. These effects could potentially be even longer lasting if
coupled interactions between the ocean and atmosphere are engaged 
\cite{colose2016hemispherically,raible2016tambora,stevenson2016nino, 
schurer2014small,schurer2013separating}. 
Given that hurricanes are sensitive to the
regions where moisture convergence occurs, it follows that such
radiative effects on the global circulation would influence hurricane
statistics.


\textbf{Large eruptions were suggested to be the dominant forcing for
  the pre-industrial period by comparisons between all-forcing and
  single-forcing last millennium model simulation and multiproxy
  constructions, though greenhouse gases also had detectable
  contributions \cite{schurer2013separating, schurer2014small}}. 
  Studies have also shown that large tropical volcanic eruptions 
  may have long-lasting
influences on the Atlantic multi-decadal oscillation and lead to El
Ni\~no–-like warming in the cold season after the eruption \cite{gcm_lme,
stevenson2016nino} such impacts could, in turn, affect hurricane 
statistics because these large-scale modes help
govern, in part, the frequency and intensity of storms occurring in any
given year.


\textbf{The historical period only provides a few clues about the
  relationship between volcanic forcing and hurricanes, and much less
  about the effect of volcanic eruptions on hurricanes or TCs.}
Nevertheless, modeling studies suggest that a reduction in TC
accumulated energy, TC duration, and lifetime maximum intensity occurs
following a volcanic eruption due to a decrease in SST and increase in
upper tropospheric/lower stratospheric temperature \cite{volc_hurrs2},
all of which decreases TC efficiency \cite{trop_cool}.  There is
evidence that after some eruptions, an asymmetric increase in
stratospheric aerosols occurs in the hemisphere in which the eruption
took place, modifying the sea surface temperature gradient
\cite{asym_forcing}.  This gradient shifts the location of the
Inter-tropical Convergence Zone (ITCZ) to the opposite hemisphere of
the eruption \cite{asym_forcing}, which hinders hurricane development
in the volcano’s own hemisphere due to a decrease in convection and
increase in wind shear.  In fact, after the northern hemisphere
eruptions of Mount Pinatubo (1991) and El Chichon (1982), North
Atlantic TC activity decreased, while TC activity increased following
the southern hemisphere eruption of Agung (1964)
\cite{volc_hurrs3,volc_hurrs2}.

\textbf{Attempts to use paleoclimate indicators of past events (i.e.,
  paleotempestology) are limited by the paucity of appropriate
  archives and proxies \cite{liu20081200,mann2009atlantic,donnelly2015}}. 
  Importantly, most paleorecords are commonly
constrained to certain geographic areas and currently provide limited
information from other regions also commonly affected by TCs, such as
the Caribbean \cite{oliva2018paleotempestology}. The use of paleotempestological
records is further limited because it is not possible to fully
reconstruct the tracks and lifetime of past TCs, most of which occur
over the ocean \cite{emanuel2005increasing}. These limitations preclude the
extensive use of such data to evaluate the effects of major volcanic
eruptions on TC activity \cite{oliva2018paleotempestology, 
yan2015tropical,korty2012variations}.

\textbf{Hurricanes are mesoscale features of the tropical circulation,
  and as such they depend critically on quantities that are typically
  unresolvable in the coarse resolution grid of the CMIP5 generation
  of models, which typically have nominal horizontal resolutions on
  the order of 50-200km.} Overcoming this limitation requires one of
three approaches. The first approach is relatively simple, and entails
calculating thermodynamic metrics like \textit{potential intensity}
(PI) at the native (coarse) resolution of GCM output
\cite{wang,ke_nolan,tang,bister2002}. Such indices can then be used to 
infer what would have happened if hurricanes had been resolved in a given
simulation. A recent study \cite{yan2018divergent} 
did just that using the
last millennium ensemble (LME) to determine the theoretical effects of
volcanic eruptions during the last 1000 years on hurricane/TC
potential intensity. The authors found a significant relationship
lasting up to 3 years post-eruption, but also ``divergent'' responses
at the mid and high latitudes to the volcanic forcing. While this
approach is computationally efficient, it doesn't explicitly attempt
to \textit{simulate} Hurricanes/TCs.

A second approach has been to use a statistical method to downscale
model output \cite{down_method_ke,cam_down_ke}. Although this approach
is computationally lightweight, allowing it to be used to investigate
long term variability in a fully coupled GCM simulation of the last
millennium \cite{lme_down_ke}, it does not directly simulate the
sub-grid scale features of Hurricanes and TCs.

The third approach employs a regional model to \textit{dynamically}
downscale GCM output \cite{down_21st_gv}. Dynamical
downscaling typically requires high performance computing
infrastructure as well as boundary conditions from the ``parent'' GCM
at six hourly temporal resolution. It is therefore much more
computationally expensive than the other two methods, but it provides
greater insight into the storms that would have occurred in a given
GCM framework if it were run with sufficiently high spatial
resolution. Dynamical downscaling has been widely used to evaluate
hurricane statistics during the 20th and 21st century \cite{emanuel12219}, yet
it has not been widely adapted to the last millennium paleoclimate
modeling context.

\section{Data \& Methods}\label{methods}

\textbf{We dynamically downscale two members of the ``Last Millennium
  Ensemble'' \cite{gcm_lme}.} The LME consists of over two dozen
fully forced, and single forcing, experiments from the period spanning
850 CE to 2005. While monthly data was archived for most of the
members of the LME, two simulations were run with sufficiently high
temporal output to allow for high resolution dynamical downscaling
using a regional model. One of these runs was a fully forced last
millennium simulation and the other was a long control simulation with
time invariant boundary conditions. We further evaluate the strengths
and limitations of our methodology by comparing downscaled reanalysis
data to an historical database of hurricane tracks and
intensities. The details of this approach are described below.

\subsection{Data}
\subsubsection{Last Millennium Ensemble}
Output from only two members of the LME were archived at sufficiently
high temporal resolution to be used as boundary conditions for WRF: a
fully-forced simulation with time varying boundary conditions
($LME_{forced}$) and a pre-industrial control simulation with time
invariant boundary conditions ($LME_{control}$). Both of these
simulations were run from 850 to 2005 CE using the Community Earth
System Model (CESM) version 1.1, with the Community Atmosphere Model
(CAM) version 5. The resolution of the atmosphere and land grids are
nominally ${\sim}2^\circ$, and ${\sim}1^\circ$ for ocean and sea ice
grids. While both runs were spun up for 200 years under control
conditions prior to 850 CE, $LME_{forced}$ was forced with the
transient evolution of solar intensity, volcanic emissions, greenhouse
gases, aerosols, land-use conditions, as well as insolation changes
from planetary orbit and tilt. In the $LME_{control}$ the boundary
conditions were simply held fixed at their pre-industrial levels, thus
providing an unforced baseline for evaluating changes
in hurricane statistics following large volcanic eruptions.

\subsubsection{ERAI \& IBTRACS}
The native $2^o$ resolution of CAM5 in the LME simulations would make
it impossible to resolve hurricanes, hence we cannot evaluate the
reliability of our downscaling methodology (see \Cref{WRF}) with LME
data alone. We therefore also downscaled the ERA-Interim (ERA-I)
\cite{erai_reanal} reanalysis data to retrospectively predict historical
hurricanes, then compared those predictions against the IBTrACS
\cite{ibtracs} database.

ERA-I comprises a reanalysis dataset starting in 1979 and available
until August 2019. It uses four-dimensional variational data
assimilation (4DVAR), yielding a significant advantage over reanalysis
products using 3DVAR. This improves asynoptic data handling and allows
for the influence of an observation to be more strongly controlled by
model dynamics \cite{tc_reanal:2}. This data assimilation method is
coupled with the ECMWF Integrated Forecast Model (IFS) to extrapolate
fields between observations. A detailed description of IBTrACS is
provided in \cite{ibtracs} and ERAI is comprehensively discussed in
\cite{erai_reanal}. In our work, $6$-hourly ERAI data was downscaled
in WRF and compared to IBTrACS for the period $1995-2005$. The
comparison was made using the suite of diagnostics described in
\Cref{diags}.

\subsection{Methods}
\subsubsection{Dynamical Downscaling}\label{WRF}

We used WRF version 3.9 (WRFV3.9) \cite{wrf_tech} to dynamically
downscale archived data from LME simulations with the physics schemes
shown in \Cref{wrf:specs}: (1) WRF single-moment 6-class for
micro-physics \cite{mp_phys}, (2) Yonsei University for PBL
\cite{pbl_phys}, (3) Kain-Fritsch for convection \cite{cu_phys}, (4,5)
rapid radiative transfer model with greenhouse gases for long-wave and
short-wave radiation \cite{rad_phys}, (6) Noah for land surface
\cite{sfc_phys}, (7) fifth generation mesoscale model for surface
layer \cite{sfclay_phys:1,sfclay_phys:2,sfclay_phys:3}, and
(8) simple mixed-layer for ocean \cite{ocn_phys}. We also turned on
heat and moisture surfaces fluxes (isfflx=1) and modification of
exchange coefficients $C_d$ and $C_k$ according to surface winds
(isftcflx=1).

WRF was run for a total of 250 simulation years over a domain spanning
the North American sector from $130W$ to $15E$, allowing us to
identify and track storms from both the Atlantic and Eastern Pacific,
even after making landfall in North America. We elected to use a
horizontal grid spacing ($\Delta X$) of $30km$. The $30km$ spacing
represents a compromise between our competing requirements for high
resolution output and a large sample size; each downscaled year used
approximately 2,000 core hours on the Cheyenne supercomputer totaling
about one million core hours for all years. While the $30km$
resolution is somewhat coarse for resolving certain features of
hurricanes, our results did not seem particularly sensitive to this
choice in comparison to runs with $\Delta X$ set to only $10km$. However, the
three fold increase in spatial resolution would have translated into
more than a 10 fold increase in core hours, or a 10 fold reduction in
the number of years simulated. All data from the LME were prepared for WRF 
using procedure and code described in \cite{tech_notes}.

\subsubsection{Tracking Tropical Cyclones with TSTORMS}\label{tstorms}

We applied the TSTORMS \cite{tc_algo} tracking software, developed and
supported by GFDL, to analyze the results of downscaling. This routine
uses minimum pressure and maximum vorticity criteria to identify
cyclones. Events are stored as ``storms'' if they satisfy the
following conditions for a preset number of days ($n_{days}$): (1)
That the maximum vorticity location is within a threshold radius
($r_{crit}$) of the minimum pressure location, (2) that the core
temperature of the cyclone is higher than outside of the core by a
threshold difference ($twc_{crit}$) and (3) the difference in vertical
distance between pressure levels at $200hPa$ and $1000hPa$ outside and
inside the core exceeds a threshold value ($thick_{crit}$).

As described in \cite{kerry_clivar} and \cite{tc_algo}, tracking
results are sensitive to the details of the tracking scheme that is
employed, and especially the threshold values selected for identifying
storms \cite{tc_track}. To identify sensitivity to threshold values,
we conducted a limited parameter sweep to determine optimal threshold
values. We calculated the difference between ERAI downscaled output
and IBTrACS data, for each set of parameters, using the diagnostics
described in \Cref{diags}. We used the set of parameters that
achieved the minimum difference of ${\sim}13.5\%$. This parameter set
was $r_{crit} = 1.5^{\circ}$, $twc_{crit} = 1.0^{\circ}C$,
$thick_{crit} = 50m$, and $n_{days} = 2$.

\subsubsection{Diagnostics}\label{diags}

Once hurricanes were identified in our downscaled LME data using
TSTORMS, we calculated 15 diagnostic metrics to evaluate differences
in the statistics of storms in $LME_{control}$ and those occurring
after large eruptions in $LME_{forced}$.  These diagnostics consist of
storm number vs (1) month, (2) year, (3) latitude, (4) longitude, (5)
maximum wind speed, (6) minimum pressure, (7) decay time from maximum
wind speed, and (8) decay time from minimum pressure. Additionally, we
calculated percentage of storms within (9) May to November, (10)
$0-25N$ latitude, (11) $100W-50W$ longitude, (12) $1020hPa-980hPa$
pressure, (13) $0m/s-40m/s$ maximum wind speed, (14) $0-100hrs$ decay
time from maximum wind speed, and (15) $0-100hrs$ decay time from
minimum pressure. Mean values and quantile values (expressed as
percentages) were used to calculate fractional differences between
$LME_{control}$ and $LME_{forced}$, and these differences were
averaged over all diagnostics for a composite percentage
difference. We refer to the mean difference of diagnostics 1-8 as the
total ``average difference'' and the mean difference of diagnostics
9-15 as the total ``percentage difference.'' We also computed these
metrics from our downscaled ERA-I data to compare them to the IBTrACS
database as a test of our methodology, as described in sections
\Cref{WRF} and \Cref{tstorms}.

The diagnostics described above were used as test statistics to
evaluate whether volcanic eruptions have a measurable effect on
hurricane behavior. These diagnostics were selected in order to assess
hurricane behavior across a broad range of characteristics. The
diagnostics not only quantify hurricane behavior across the temporal
and spatial domain, but also assess more fundamental physical
characteristics. In addition, the diagnostics can be used with limited
data consisting only of time, location, wind speed, and surface
pressure. This presents a versatile and efficient approach to capture
both mean climatology and more fine structured hurricane behavior.

To determine whether volcanic eruptions effect hurricane statistics,
we performed two-sample KS-tests for distributions of each of the
diagnostics. The two samples tested for each diagnostic came from
downscaled $LME_{control}$ and $LME_{forced}$ data. Since
$LME_{control}$ does not include volcanic eruptions, agreement with
$LME_{control}$ is confirmation of the null hypothesis.

\subsubsection{Potential Intensity}
In addition to downscaling, we use the original CESM data to compute
potential intensity (PI) fields. This gives us insight into the
theoretical effect of eruptions on hurricanes, without the
computational overhead of downscaling. Following the thermodynamic
analysis in \cite{pi_ke}, we use Equation (\Cref{PI:eqn}) to calculate
PI:

\begin{equation}
{V_m} \propto \sqrt{\frac{T_s-T_{o}}{T_{o}}(k^{*}-k)},
\label{PI:eqn}
\end{equation}

where \Cref{PI:eqn} $V_m$ is the maximum tangential wind speed, $T_s$
is the temperature at the ocean surface, $T_o$ is the outflow
temperature at the top of the troposphere, and $k^{*}-k$ is the
enthalpy flux (or latent heat flux) at the sea-air interface. The
enthalpy flux is given by $c_p(T_{SST}-T_{air})+L(q^{*}-q)$, where
$c_p$ is the specific heat capacity at constant pressure, $T_{SST}$ is
the sea surface temperature, $T_{air}$ is the temperature of air at
the surface, $L$ is the latent heat of vaporization, $q^{*}$ is the
saturated specific humidity at the surface, and $q$ is the specific
humidity of air at the surface. We are only interested in the relative
difference between $LME_{control}$ and $LME_{forced}$ given by
Equation (\Cref{dpi}), so we are unconcerned with additional scaling
factors.

\begin{equation}
\delta PI = \frac{{V_{m}}^{LME_{forced}}}{{V_{m}}^{LME_{control}}}-1
\label{dpi}
\end{equation}

\subsubsection{Case Studies}\label{cases}

In this section we present some limited case studies of two well 
known hurricanes: (1) Mitch (1998) and (2) Katrina (2005). As previously 
mentioned, we did not expect the downscaled ERAI to perfectly match 
IBTrACS. This is due to the inherently chaotic nature of hurricanes, 
the native resolution of ERAI, and systematic under-estimation of 
intensities by WRF. However, by increasing the resolution of the WRF 
simulations we can roughly match some individual hurricanes. Instead 
of the $30 km$ resolution used for the rest of the downscaled 
simulations, here we use $10 km$. This significantly increases the 
computational overhead but allows us to better identify TCs. We handled 
this constraint by reducing the domain to $100 W$-$20 W$ and $5 N$-$45 N$. 
We also only simulated the month containing the storm of interest, rather 
than the entire year. All physics schemes were kept the same as previously 
described. 

\myparagraph{Hurricane Mitch}
Hurricane Mitch is the second deadliest Atlantic hurricane in 
recorded history, responsible for over 11,000 deaths in Central 
America alone due to rain-induced flooding \cite{hellin}.  After 
forming in the southwestern Caribbean Sea on 22 October 1998, 
Mitch strengthened into a Category 5 hurricane, attaining a 
minimum pressure of 905 mb, which is the eighth lowest pressure 
ever recorded in the Atlantic Hurricane Basin \cite{pasch}.  Mitch 
turned southward and slowly weakened before making landfall as a 
minimal hurricane in Honduras.  The nearly stationary movement of 
the hurricane (4 kt for a week) pulled moisture from the Pacific 
Ocean and Caribbean into the mountains of Honduras and Nicaragua, 
resulting in daily rainfalls of over one foot due to orographic 
lift.  A peak storm total of 75 in. was estimated from satellite-derived 
methods \cite{hellin}.  After the center of the remnant low emerged 
over the Gulf of Mexico, it regained tropical storm strength and 
made landfall in Florida on November 5 \cite{pasch}.  Mitch accelerated 
northeast, underwent extra-tropical transition, and dissipated.
\par
The results from downscaling ERAI at 10 km are shown in figure 
\Cref{mitch:tracks} (A) and IBTrACS is shown in \Cref{mitch:tracks} 
(B). TSTORMS detects Mitch in the 10 km downscaled ERAI output on 
October 21, 1998. However, TSTORMS only resolves Mitch until October 
25, 1998. We see from the figures that the location of genesis is in 
good agreement with IBTrACS but the intensity is underestimated. This 
is likely why the track is observed to end on October 24. To contrast 
we show the closest candidate for Mitch from downscaling with 30 km 
resolution in \Cref{mitch:tracks} (C). TSTORMS resolved this 
storm from October 23 to October 25. We see a difference in location 
as well as intensity variation over the track.  

\myparagraph{Hurricane Katrina}
Hurricane Katrina is one of the deadliest Atlantic hurricanes and 
the costliest hurricane in United States history \cite{beven}.  
Katrina developed over the central Bahamas on 24 August 2005 and 
strengthened into a Category 1 hurricane before making landfall 
in southern Florida.  After briefly weakening to a tropical storm, 
Katrina strengthened once over the Gulf of Mexico and underwent 
an eye replacement cycle that doubled the size of the tropical 
wind field. A newly defined eyewall developed and the storm 
underwent rapid intensification, with peak winds reaching 150 
kt \cite{beven}.  Katrina made landfall in southeastern Louisiana 
on 29 August.  Its large wind field drove a storm surge into 
Mississippi, Alabama, and Lake Pontchartrain, with the latter 
infamously breaking the levees in New Orleans, leading to widespread 
destruction and casualties.  Katrina dissipated and was absorbed 
by a frontal zone as it moved from the Southeast to the Northeast US.
\par
The results from downscaling ERAI are shown in figure 
\Cref{katrina:tracks} (A) and IBTrACS is shown in figure 
\Cref{katrina:tracks} (B). TSTORMS detected Katrina on August 
21, 2005 and the track ended August 29, 2005. We see from the 
figures that again the location is in reasonably good agreement 
with IBTrACS and the intensity is also in close agreement. To 
contrast we show the closest candidate for Katrina using $30km$ 
resolution in \Cref{katrina:tracks} (C). TSTORMS tracked 
this storm from September 1, 2005 to September 3, 2005. Not only 
is the date significantly divergent but the location is as well. 
Finally, we see huge degradation in the intensity profile as 
compared to that from $10km$ resolution.    

\section{Results}\label{results}

He we show comparisons between downscaled output from ERAI and IBTrACS 
as well as between $LME_{forced}$ and $LME_{control}$. The ERAI vs. 
IBTrACS comparison provides a baseline of absolute accuracy. The 
comparison between $LME_{forced}$ and $LME_{control}$ is focused 
specifically on the effect of aerosol forcing from volcanic eruptions. 
We downscaled 150 consecutive years of $LME_{control}$ and 100 years 
of $LME_{forced}$ combined from 2 year runs after 50 separate volcanic 
eruptions. The reconstruction of these eruptions is described in detail 
in \cite{erups_recon}. 
\par
In addition to the downscaling results, we look at the much less 
computationally intensive PI analysis. It is possible to compute the 
PI field for the entirety of the LME simulations, but it is not feasible 
to do with dynamical downscaling. We present the average potential 
intensity anomaly for all eruptions and for the strongest eruptions. We 
also show the PI anomaly for the 10 strongest eruptions and 10 weakest 
eruptions. PI shows what hurricane behavior should be expected if all 
hurricanes achieved maximum possible intensity based on thermodynamic 
conditions of the environment. However, this theoretical intensity is 
rarely achieved by hurricanes in practice.

\subsection{ERAI vs IBTrACS}
As shown in \Cref{evi:table}, using our suite of diagnostics, we 
found an overall agreement between ERAI and IBTrACS of ${\sim}86.5\%$, or 
a composite difference of ${\sim}13.5\%$. $6$-hourly ERAI data downscaled 
in WRF was compared to IBTrACS for the same time period ($1995-2005$). 
Diagnostics distributions for both ERAI and IBTrACS are shown in 
\Cref{evi:diags}, and tracks for both cases are shown in 
\Cref{erai:ibtracs:tracks}. It is worth noting that truncation of 
the domain in our ERAI simulations contributes to the differences in 
latitude and longitude peaks seen in \Cref{evi:diags}.  
\par
We also implemented a rudimentary version of our own track matching 
algorithm and we saw similar agreement to that in \cite{hodges2017well}. 
We also saw close agreement comparing results produced by other 
diagnostics, similar to those described in \Cref{diags}. The physics 
schemes in \Cref{WRF} and threshold values in \Cref{tstorms} 
were used in response to a self-selected $15\%$ difference threshold 
imposed between ERAI and IBTrACS, as quantified by our diagnostics suite. 

\subsection{Effect of Eruptions on Hurricane Statistics}
\subsubsection{Average Effect of Eruptions}
In comparing $LME_{control}$ and $LME_{forced}$ we focused on the 
effect of aerosol forcing from volcanic eruptions. These signals are 
shown in \Cref{erups:plot}. We selected 50 eruptions 
from $LME_{forced}$ and ran WRF for two years after each of the 
eruptions. $LME_{control}$ was run using WRF for 150 years to give a 
sufficient sample of natural variability. The $LME_{control}$ run was 
ensured to have sufficient length by looking at the SST signal in 
frequency space, as shown in \Cref{spectrum}. This figure shows 
no significant low-frequency variability is missing from the control 
sample. The hurricane tracks over 100 years of both $LME_{control}$ 
and $LME_{forced}$ are shown in \Cref{forced:ctrl:tracks}.
\par
Distributions of the diagnostics for $LME_{control}$ and $LME_{forced}$, 
with all 50 eruptions included, are shown in \Cref{50:erups}. 
Performing two sample ks-tests on the distributions, along with 
significance tests on the difference of mean values, shows that the 
overall effect of all 50 eruptions is consistent with the null hypothesis. 
That is, the overall effect of all 50 eruptions is consistent with 
the natural climate variability seen in $LME_{control}$. 
Figures \Cref{ks:all} and \Cref{sig:all} show the results of these tests. 
The ks-tests show a maximum difference between the two samples (D-value), 
and a probability that the two samples are drawn from the same 
distribution (P-value). The significance tests show the fraction 
of the $LME_{control}$ sample which is greater than and less than 
the mean value of the corresponding $LME_{forced}$ diagnostic. 
\par
Although the aggregate effect of eruptions on hurricanes seems 
non-significant, we also calculated pearson correlation coefficients 
on eruption strength and diagnostic changes. The significance of the 
calculated coefficients can be evaluated by determining the confidence 
interval for zero correlation. The 90\% confidence interval for zero 
correlation with all eruptions is $[-0.235,0.235]$. This is the 
interval of a discrete normal distribution with $N=50$ samples 
(eruptions) which includes correlation values with probabilities 
greater than 0.05. Due to the symmetry of the normal distribution 
the left and right tails outside this interval total a probability 
of 0.1. Thus, we can say with at least 90\% confidence that yearly 
number, intensity, and lifetime, correspond with eruption strength. 
The correlation coefficients are listed in \Cref{corr:all}.   

\subsubsection{Effect of Strongest Eruptions}
Distributions of diagnostics with only the 10 strongest eruptions
included are shown in \Cref{10:erups}. Tables showing the
results of ks-tests and significance tests on the strongest eruptions
are in figures \Cref{ks:10} and \Cref{sig:10}. We see in the
significance table that the $LME_{forced}$ mean values for yearly
number, intensity, and lifetime suggest we should reject the null
hypothesis at only the $70\%-80\%$ confidence limit. These signals
show that for the 10 largest eruptions the yearly number, intensity,
and lifetimes are reduced. Interestingly, the eruptions with the
largest net effects are 1213 (8th strongest) and 1815 (3rd
strongest). This clearly demonstrates that other factors are at work
besides amount of aerosol forcing. Both the 1213 and 1815 eruptions
have ${\sim}13\%$ total average difference from
$LME_{control}$. Tables shows the respective significance tests are
shown in figures \Cref{sig:1213} and \Cref{sig:1815}.

\subsection{Potential Intensity}
The PI anomalies ($\delta PI$) for the strongest eruptions are shown 
in figures \Cref{pi:10:avg} and \Cref{pi:10:str}. $\delta PI$ for the 
weakest eruptions is shown in \Cref{pi:10:wk}. Strength of eruptions 
were determined by the peak aerosol mass, as shown in 
\Cref{erups:plot}. The average fractional PI anomaly for all 
eruptions is shown in \Cref{pi:all:avg}. The hatching in 
\Cref{pi:all:avg} is based on a p-value threshold of 0.01 for 
a two-sided t-test at each grid point. This analysis was done for 
\Cref{pi:10:avg} as well. The disparity between the two figures 
supports the notion that an effect on hurricanes is observed only 
for the largest eruptions. The average decrease in PI for the 
strongest eruptions is ${\sim}2.2\%$. The average fractional PI 
anomaly for all eruptions is a decrease of ${\sim}1.0\%$.   

\section{Discussion \& Conclusion}\label{discuss}

In this work we have explored the effect of volcanic eruptions in 
the past millennium on hurricane climatology. To do this we first 
validated our approach of downscaling CESM data with WRF by comparing 
results of ERAI downscaling with IBTrACS data. We also performed a 
parameter search for our cyclone tracking algorithm in order to achieve 
high accuracy and to understand sensitivity to selected parameters. 
We then compared the results of downscaling our control data from CESM 
with forced data from CESM, where we focused on the years in the forced 
data which bounded the volcanic eruptions. 
\par
Here, our results also show volcanic eruptions suppress hurricane intensity by
influencing both the surface temperature and the vertical temperature
profile, although only large eruptions show strong influences on
hurricane intensity. We found that the aggregate effect of eruptions is 
consistent with the null hypothesis: the control case. However, we see evidence 
that sufficiently strong eruptions do result in lower annual hurricane count, 
reduced intensity, and shorter lifetimes. This evidence is in the form of 
KS and significance tests on diagnostic distributions, as well as 
correlations between strength and changes in the mean values of 
these diagnostics. PI analysis also supports these conclusions. 
\par
By dynamically downscaling the ERAI dataset we were able to generate 
hurricane statistics based on observational climate data. 
IBTrACS provided observational data for hurricanes for roughly the 
same time period as ERAI. Thus, comparing
the downscaled ERAI results to hurricane tracks and intensities from
IBTrACS allowed us to evaluate the accuracy of our approach. 
In ref. \cite{hodges2017well}, the authors assess how well TCs are
represented in reanalysis products. This work used two TC-track
matching approaches, referred to as (1) ``direct matching" and (2)
``objective matching". The authors further used several diagnostics in
order to compare reanalysis TC tracks to those found in IBTrACS. The
objective matching approach, which employs a tracking algorithm
similar to TSTORMS, found an agreement of ${\sim}60\%$ with ERAI in
the Northern Hemisphere. A simple ``direct matching" implementation of
our own achieved similar agreement. Due to the inherently chaotic nature 
of hurricane genesis exact agreement between ERAI and IBTrACS was not expected. 
Assessing our approach was the primary objective in comparing ERAI with 
IBTrACS. ERAI data has resolution on the order of one degree which 
limits the ability to match individual hurricanes through downscaling. 
We expected ERAI to capture the observational record for mean climate 
and to provide good agreement between downscaled results and overall hurricane
statistics seen in IBTrACS.
\par
The main focus of our work was the comparison between downscaled results of the
$LME_{control}$ and $LME_{forced}$ datasets. There has been extensive use of 
PI analysis to assess hurricane behavior in various environments 
\cite{yan2018divergent,ting2015,Kossin2009,vecchi2007effect}, and PI is 
sometimes preferred for its theoretical and computational simplicity. Our 
PI results give us a single snapshot of the thermodynamic environment for 
a given time period. This is a useful low cost supplement to higher 
resolution analysis but doesn't tell us anything about the yearly number, 
lifetime, or intensity distributions. It also doesn't tell us anything 
about individual hurricane tracks and only provides a rough expected 
spatial distribution. 
\par
Our PI results are in good agreement with those shown 
But at the moment,in \cite{yan2018divergent,vecchi2007effect}. Volcanic aerosols are 
seen to reduce SSTs and this is reflected
in reduced PI. However, in ref \cite{wehner2015}, caution is urged in using PI 
to draw strong conclusions about tropical cyclone projections as it fails to 
capture features seen in high-resolution climate models. Dynamical 
downscaling provides far greater detail in both the spatial and temporal domain. 
This is evident when looking at the cyclone track plots and extensive suite of 
diagnostics used to analyze downscaled results. At the cost of increased 
computation time we used dynamical downscaling to overcome some of these 
resolution deficiencies. 
\par
As the anthropogenic climate crisis worsens, global mitigation efforts 
without climate engineering may be insufficient to avoid a 2 degree C 
warming scenario above pre-industrial levels by 2100 
\cite{intergovernmental2018global}. A research agenda was recently 
published to pursue methods of carbon dioxide removal as part of a climate 
geoengineering mitigation initiative \cite{national2018negative}.  However, 
these technologies would need to remove more than 10 GtCO2 per year by 2050 
in addition to a major phase out of fossil fuel energy source 
\cite{united2017emissions}, and would be costly. As a cheaper technology, 
stratospheric aerosol geoengineering (SAG) may need to be implemented 
temporarily in addition to these methods to avoid, for example, an increase 
in the intensity of hurricanes under a warmer climate. 
\par
The results in this study have significant implications for hurricane 
development in a potential future climate under a SAG regime.  Although we 
analyzed the effects of an increase in stratospheric aerosols from volcanic 
eruptions, the results are analogous as to what could occur under a SAG regime. 
The initial PI analysis demonstrated a slight decrease in the maximum potential 
strength of hurricanes across the Atlantic Ocean during the strongest eruptions, 
presumably from a decrease in the efficiency of a hurricane due to an increase 
in upper tropospheric temperature and a decrease in lower tropospheric 
temperature. Although analyses of impacts were once limited by historical 
observation and courser resolution, we were able to evaluate the direct 
influence of many volcanic eruptions on individual hurricanes.  For example, 
the strongest eruptions in the downscaled $LME_{forced}$ experiment produced 
a slight reduction in hurricane frequency, intensity, and lifetime. These 
impacts would be similarly felt if SAG was implemented, removing some 
uncertainty associated with regional changes in tropical cyclone development 
for the Northern Atlantic Ocean. Under a relatively strong SAG regime and 
according to our results, hurricanes would either remain the same or slightly 
decrease in frequency and intensity.
\par
Although our results show moderate correlation between eruption strength and 
certain diagnostic measures, it is not necessarily true that stronger 
eruptions have a larger effect on hurricane statistics. Additionally, research 
has shown large uncertainties in volcanic reconstructions and 
seasonality of volcanic eruptions 
\cite{schmidt2012climate,schmidt2011climate,stevenson2017role,raible2016tambora}.
This presents a direction for further investigation. 
In this vein, one could look at an ensemble of higher resolution GCM simulations 
on one or two of the strongest volcanic eruptions. This eruption profile 
will be simulated both in the climate conditions during the historical eruption 
as well as under future climate change conditions. An ensemble average or 
simulations with perturbed initial conditions will allow us to home in on the 
sole effect of aerosol forcing. This will also allow us to explore the 
question of whether downscaling introduced any unknown biases. An ensemble 
under future climate change conditions will allow us to explore the interplay 
of large aerosol forcing and strong anthropogenic forcing.    
\par

\clearpage
\newpage

%chapter 1 tables


\begin{table}[!tbp]
\centering
\begin{tabular}{lrrr}
\toprule
             Physics Schemes &  Name & Parameter & Value \\ 
\midrule
            (1) Micro-physics &     WSM6 &  mp\_physics & 6 \\  
            (2) PBL &    YSU &  bl\_pbl\_physics &  1 \\    
            (3) Convection &   Kain-Fritsch &  cu\_physics & 1 \\    
            (4) Long-wave radiation &    RRTMG &   ra\_lw\_physics & 4 \\    
            (5) Short-wave radiation &    RRTMG &   ra\_sw\_physics & 4 \\    
            (6) Land surface &   Noah &   sf\_surface\_physics & 2 \\    
            (7) Surface layer &    MM5 &  sf\_sfclay\_physics &  1 \\    
            (8) Ocean &    Mixed-layer &  sf\_ocean\_physics &  1 \\    
\bottomrule
\end{tabular}
\caption{WRF Physics Schemes}
\label{wrf:specs}
\end{table}

\begin{table}[!tbp]
\centering
\begin{minipage}[b]{0.45\textwidth}
\begin{tabular}{lrrr}
\toprule
             Averages &         ERAI &      IBTrACS \\ 
\midrule
            month &     7.65 &     8.24 \\  
       yearly num &    31.27 &    34.82 \\   
              lat &    18.23 &    21.39 \\    
              lon &   -80.72 &   -88.38 \\    
     max wind m/s &    30.26 &    34.87 \\    
    min press hPa &   988.56 &   979.65 \\    
           w-life &    45.37 &    44.69 \\    
           p-life &    44.02 &    52.53 \\    
\bottomrule
\end{tabular}
\end{minipage}
\hfill
\begin{minipage}[b]{0.45\textwidth}
\begin{tabular}{lrrr}
\toprule
             Percents &         ERAI &      IBTrACS \\ 
\midrule


             May-Nov &     0.88 &     0.99 \\    
               0-25N &     0.78 &     0.73 \\   
             100-50W &     0.65 &     0.44 \\   
             0-40m/s &     0.97 &     0.70 \\   
         1020-980hPa &     0.81 &     0.62 \\   
        (w) 0-100hrs &     0.94 &     0.91 \\   
        (p) 0-100hrs &     0.94 &     0.87 \\
 \\

\bottomrule
\end{tabular}
\end{minipage}

\noindent\fbox{\parbox{\textwidth}{%
\centering
total average difference: 0.096\\
total percent difference: 0.18\\
composite difference: 0.135}}
\caption{ERAI vs IBTrACS stats}
\label{evi:table}
\end{table}

\begin{table}[!tbp]
\centering
\begin{minipage}[b]{0.45\textwidth}
\begin{tabular}{lrrr}
\toprule
             KS-Tests & D-value & P-value \\
\midrule
month & 0.004 & 1.0 \\
yearly num & 0.018 & 1.0 \\
lats & 0.036 & 0.9 \\
lons & 0.038 & 0.86 \\
max wind & 0.024 & 1.0 \\
min press & 0.048 & 0.6 \\
w-life & 0.014 & 1.0 \\
p-life & 0.012 & 1.0 \\

\bottomrule
\end{tabular}
\caption{KS-tests with strong eruptions}
\label{ks:10}
\end{minipage}
\hfill
\begin{minipage}[b]{0.45\textwidth}
\begin{tabular}{lrrr}
\toprule
             Sig-Tests & \% greater & \% less \\
\midrule

month & 0.461 & 0.513 \\
yearly num & 0.584 & 0.351 \\
lats & 0.487 & 0.513 \\
lons & 0.318 & 0.682 \\
max wind & 0.773 & 0.227 \\
min press & 0.286 & 0.714 \\
w-life & 0.675 & 0.325 \\
p-life & 0.708 & 0.292 \\

\bottomrule
\end{tabular}
\caption{Significance tests with strong eruptions}
\label{sig:10}
\end{minipage}
\end{table}


\begin{table}[!tbp]
\centering
\begin{minipage}[b]{0.45\textwidth}
\begin{tabular}{lrrr}
\toprule
             Sig-Tests & \% greater &  \% less \\

\midrule

month & 0.63 & 0.357 \\
yearly num & 0.812 & 0.169 \\
lats & 0.747 & 0.253 \\
lons & 0.708 & 0.292 \\
max wind & 1.0 & 0.0 \\
min press & 0.0 & 1.0 \\
w-life & 0.896 & 0.104 \\
p-life & 0.981 & 0.019 \\

\bottomrule
\end{tabular}
\caption{Significance tests for 1213 eruption}
\label{sig:1213}
\end{minipage}
\hfill
\begin{minipage}[b]{0.45\textwidth}
\begin{tabular}{lrrr}
\toprule
             Sig-Tests & \% greater &  \% less \\
\midrule

month & 0.513 & 0.481 \\
yearly num & 0.883 & 0.084 \\
lats & 0.325 & 0.675 \\
lons & 0.195 & 0.805 \\
max wind & 0.831 & 0.169 \\
min press & 0.058 & 0.942 \\
w-life & 0.896 & 0.104 \\
p-life & 0.942 & 0.058 \\

\bottomrule
\end{tabular}
\caption{Significance tests for 1815 eruption}
\label{sig:1815}
\end{minipage}
\end{table}

\begin{table}[!tbp]
\centering
\begin{minipage}[b]{0.45\textwidth}
\begin{tabular}{lrrr}
\toprule
             KS-Tests &     D-value &      P-value\\
\midrule

month & 0.0 & 1.0 \\
yearly num & 0.006 & 1.0 \\
lats & 0.004 & 1.0 \\
lons & 0.0 & 1.0 \\
max wind & 0.006 & 1.0 \\
min press & 0.006 & 1.0 \\
w-life & 0.002 & 1.0 \\
p-life & 0.0 & 1.0 \\

\bottomrule
\end{tabular}
\caption{KS-tests with all eruptions}
\label{ks:all}
\end{minipage}
\hfill
\begin{minipage}[b]{0.45\textwidth}
\begin{tabular}{lrrr}
\toprule
             Sig-Tests & \% greater &  \% less \\
\midrule

month & 0.513 & 0.474 \\
yearly num & 0.435 & 0.565 \\
lats & 0.494 & 0.506 \\
lons & 0.455 & 0.545 \\
max wind & 0.519 & 0.474 \\
min press & 0.513 & 0.487 \\
w-life & 0.565 & 0.435 \\
p-life & 0.506 & 0.494 \\

\bottomrule
\end{tabular}
\caption{Significance tests with all eruptions}
\label{sig:all}
\end{minipage}
\end{table}

\begin{table}[!tbp]
\centering
\begin{tabular}{lrrr}
\toprule
             Correlation-Tests &     Correlations \\
\midrule

month & -0.1095 \\
yearly num & -0.2305 \\
lats & 0.0201 \\
lons & 0.2199 \\
max wind & -0.3185 \\
min press & 0.2913 \\
w-life & -0.0901 \\
p-life & -0.2753 \\

\bottomrule
\end{tabular}
\caption{correlations with all eruptions}
\label{corr:all}
\end{table}

\clearpage
\newpage

%chapter 1 figures

\begin{figure}[!tbp]
\centering
%\begin{minipage}[b]{0.45\textwidth}
\includegraphics[width=\textwidth,height=0.95\textheight,keepaspectratio]{./figures/Hurricane_Mitch_tracks.eps}
\caption{Hurricane Mitch: (A) ERAI 10 km, (B) IBTRACS, (C) ERAI 30 km}
\label{mitch:tracks}
\end{figure}

\begin{figure}[!tbp]
\centering
%\begin{minipage}[b]{0.45\textwidth}
\includegraphics[width=\textwidth,height=0.95\textheight,keepaspectratio]{./figures/Hurricane_Katrina_tracks.eps}
\caption{Hurricane Katrina: (A) ERAI 10 km, (B) IBTRACS, (C) ERAI 30 km}
\label{katrina:tracks}
\end{figure}

\begin{figure}[!tbp]
\centering
\includegraphics[width=\textwidth,height=0.95\textheight,keepaspectratio]{./figures/ERAI_vs_IBTRACS_tracks.eps}
\caption{ERAI (A) vs IBTRACS (B) 1995-2005. Here we see good agreement in the location of TC tracks. We note that our WRF domain truncates the ERAI tracks. We also see some underestimation of intensities in downscaled results. Resolvable intensity depends strongly on WRF resolution.}
\label{erai:ibtracs:tracks}
\end{figure}

\begin{figure}[!tbp]
\centering
\includegraphics[width=\textwidth,height=0.95\textheight,keepaspectratio]{./figures/Forced_vs_Control_tracks.eps}
\caption{$LME_{forced}$ all eruptions (A) vs $LME_{control}$ 1000-1100 (B). We see close agreement between forced and control when comparing all simulation years.}
\label{forced:ctrl:tracks}
\end{figure}

\begin{figure}[!tbp]
\centering
\includegraphics[width=\textwidth,height=0.95\textheight,keepaspectratio]{./figures/erai_ibtracs_diags.eps}
\caption{ERAI vs IBTrACS 1995-2005. Here we see good agreement in lifetime and frequency metrics. Location metrics differ mainly due to WRF domain truncation of ERAI tracks. We also see some slight intensity underestimation in ERAI due to WRF resolution.}
\label{evi:diags}
\end{figure}

\begin{figure}[!tbp]
\centering
\includegraphics[width=\textwidth,height=0.95\textheight,keepaspectratio]{./figures/50_erups_dists.eps}
\caption{$LME_{control}$ vs $LME_{forced}$ with all eruptions. Here we see qualitatively similar profiles for each metric. Notable is the frequency reduction in the forced distributions. }
\label{50:erups}
\end{figure}

\begin{figure}[!tbp]
\centering
\includegraphics[width=\textwidth,height=0.95\textheight,keepaspectratio]{./figures/10_erups_dists.eps}
\caption{$LME_{control}$ vs $LME_{forced}$ with strongest eruptions. Here we again see qualitatively similar profiles for each metric. The frequency reduction here is more pronounced than for the comparison with all eruptions. }
\label{10:erups}
\end{figure}

\begin{figure}[!tbp]
\centering
\includegraphics[width=\textwidth,height=0.95\textheight,keepaspectratio]{./figures/PI_diff_50_avg.eps}
\caption{Average PI anomaly for all eruptions. Hatching is based on a p-value threshold of 0.01 for a two-sided t-test at each grid point. We see that any anomalies in the main development region are non-significant and even observe some warming in the North Atlantic.}
\label{pi:all:avg}
\end{figure}

\begin{figure}[!tbp]
\centering
\includegraphics[width=\textwidth,height=0.95\textheight,keepaspectratio]{./figures/PI_diff_10_avg.eps}
\caption{Average PI anomaly for strongest eruptions. All points are below a p-value threshold of 0.01 for a two-sided t-test. Here we see cooling across the Atlantic basin although the main development region sees some warming.}
\label{pi:10:avg}
\end{figure}

\begin{figure}[!tbp]
\centering
\includegraphics[width=\textwidth,height=0.95\textheight,keepaspectratio]{./figures/PI_diff_50.eps}
\caption{PI anomaly for weakest eruptions with plots ordered by ascending eruption strength. Weakest eruptions are those with lowest peak aerosol mass. Here we do not observe any anomalies consistent across all eruptions.}
\label{pi:10:wk}
\end{figure}

\begin{figure}[!tbp]
\centering
\includegraphics[width=\textwidth,height=0.95\textheight,keepaspectratio]{./figures/PI_diff_10.eps}
\caption{PI anomaly for strongest eruptions with plots ordered by ascending eruption strength. Strongest eruptions are those with highest peak aerosol mass. Here we see cooling within the Atlantic basin for all eruptions.}
\label{pi:10:str}
\end{figure}

\begin{figure}[!tbp]
\centering
\begin{minipage}[b]{0.45\textwidth}
\includegraphics[width=\textwidth,height=0.95\textheight,keepaspectratio]{./figures/eruptions_plot.eps}
\caption{Aerosol mass signals for volcanic eruptions 500-2000 C.E. The peak signals shown here are used to determine eruption strength. }
\label{erups:plot}
\end{minipage}
\hfill
\begin{minipage}[b]{0.45\textwidth}
\includegraphics[width=\textwidth,height=0.95\textheight,keepaspectratio]{./figures/power_spectrum.eps}
\caption{$LME_{control}$ SST Power Spectrum. This plot shows that using 100 years of control data is sufficient and in doing so we are not missing any low-frequency content.}
\label{spectrum}
\end{minipage}
\end{figure}

\clearpage
\newpage

\chapter{Tree-Ring Data Synthesis and Analysis of Low Frequency 
Climate Variability in Tree-Ring Chronologies} %COMMENT: OK?

\section{Abstract}
  This work focuses on the development and testing of a novel methodology 
  to synthesize raw tree-ring data into comprehensive paleoclimate 
  data sets, detrend this data using a suite of detrending models, 
  and analyze the resulting chronologies. The methodology developed uses four types 
  of detrending models to construct tree-ring chronologies using 
  data from the International Tree Ring Database (ITRDB). The detrending 
  models use varying combinations of splines, negative exponential 
  functions, tree-ring segment length constraints, and variance 
  thresholds. These combinations range from less to more aggressive 
  in constraints on tree-ring segment properties and in preserving 
  low-frequency content. Analysis of chronologies is made possible using a 
  combination of multiple-taper spectrum estimation 
  methods (MTM) and principal-components analysis using singular-value 
  decomposition (SVD). The MTM-SVD approach is selected in order to 
  overcome the estimation bias inherent in Fourier analysis and because 
  of the large-scale spatial structure of climatic variations. This 
  MTM-SVD analysis provides an approach for signal detection and 
  reconstruction, along with significance assessment. A robust null 
  hypothesis is used to determine significance of signals in the local 
  fractional variance spectrum, derived from the set of singular values. 
  The methodology presented will be used to explore the effect of 
  detrending schemes, or standardization, on climatology extracted 
  from chronologies. It will also be used in future work to quantify 
  the amplitude of low-frequency hydroclimate variability in models, 
  proxies, and observations, while at the same time utilizing an ensemble 
  of last millennium numerical climate models produced by the National 
  Center for Atmospheric Research (NCAR).  

\section{Introduction}\label{intro:2}
\par
\textbf{Megadroughts, prolonged periods of aridity unlike anything
seen during the 20th century, have the potential to
pose strong threats to modern civilization.} Megadroughts
have been linked to the demise of several pre-industrial
civilizations, including the Maya, Pueblo, and Khmer. 
Furthermore, the likelihood of megadroughts occurring 
this century may be increasing due to climate change. Even 
though the threats posed by megadroughts could be quite severe 
we still currently lack a complete understanding of their underlying 
physical origin and statistics. This lack of understanding stems, 
in large part, from the insufficient extent of the instrumental
record. The decadal and centennial (dec-cen) timescales associated 
with megadroughts poses difficulties for empirical investigation.  
The comparatively short observational record is at best only 
able to shine a dim light onto the true nature of dec-cen 
variations. This prevents us from using the observational record
to determine if dec-cen variations make weak, moderate, or 
strong contributions to the total variance of local or 
regional climate. However, changes in climate seen in this century 
will be governed by both rising temperatures and natural 
climate variations. This, of course, includes dec-cen variability. 
\textbf{Understanding climate variability at 
dec-cen timescales is essential to anticipate the risk of 
megadroughts.} This will be critical for managing water, among 
countless other natural resources, in the face of climate change. 
If dec-cen variations are weak, then climate change will likely 
unfold as consequence of slow-varying changes in radiative forcing 
from burning fossil fuels. But if dec-cen variations are strong, 
then the combined effects of anthropogenic warming and natural 
variability on long timescales need to be considered for any 
adaptation and planning strategies to succeed.
\par
What exactly controls the behavior of the climate system at the 
dec-cen timescales is still a matter of active debate. 
\textbf{There are two main paradigms used to explain unforced dec-cen 
variability:} (1) The Bjerknes hypothesis and (2) the Hasselmann 
hypothesis. There is a third paradigm which states that variations 
on long timescales are forced, but here the focus is on unforced 
variability. The Bjerknes hypothesis proposes that variations 
occurring at long timescales influence those occurring at shorter 
ones. More specifically, the Bjerknes hypothesis suggests that the oceans 
are the principal source of low-frequency variability in the climate 
system and this behavior can affect shorter time scales by, for 
example, modulating interannual variability 
\cite{gershunov1998interdecadal,power1999decadal,oglesby2012role}. 
The Bjerknes approach has been bolstered by numerical experiments which
show that low-frequency modes which modulate heat flow in oceans can 
generate coherent dec-cen patterns
\cite{latif2006review,meehl2011model,delworth2012multicentennial}.
\par
The Hasselmann hypothesis proposes that dec-cen variations are the 
product of faster-varying forcings. More specifically, higher-frequency 
forcings \cite{hasselmann1976stochastic,newman2003enso} which affect 
the radiative budget \cite{schurer2013separating,ault2013continuum} 
can generate lower-frequency residuals which produce dec-cen variations.
These high-frequency forcings can include processes with large-scale 
spatial structure, such as El Nino-Southern Oscillation (ENSO) and 
its teleconnections 
\cite{vimont2005contribution,ault2009intensified,wittenberg2009historical,ault2013continuum}.  
The Hasselmann approach has been demonstrated to be a viable explanation 
for decadal-to-multidecadal variability in the Pacific sector using 
observations \cite{newman2003enso,shakun2009tropical} and coral proxies 
\cite{ault2009intensified,ault2013continuum}. The Hasselmann paradigm can 
be formulated as a red-noise process, or more formally as an Ornstein-Uhlenbeck 
process. This formulation has been shown to be consistent with megadroughts 
\cite{ault2018mega}.
\par
It is, of course, quite possible that both paradigms are needed 
to fully explain the continuum of climate variability. 
Decadal variations in sea surface temperature (SST) in the 
Pacific Ocean may be best described by the Hasselmann hypothesis 
\cite{newman2003enso}, while the North Atlantic climate could be 
better described by the Bjerknes hypothesis \cite{delworth2000observed}. 
Regions influenced by both basins would therefore potentially 
integrate the effects of both types of low-frequency behavior.
Additionally, both the Bjerknes and Hasselmann paradigms allow 
for interactions between external variations in radiative 
forcings and dynamic unforced processes on dec-cen timescales. 
A number of numerical modelling experiments have suggested a 
dynamical role for solar cycles in decadal climate variability 
\cite{emile2007nino,meehl2009amplifying,anchukaitis2012tree}. 
Similarly, radiative balance is affected by volcanic eruptions
which results in cooling \cite{robock2005cooling,mann2007climate} 
that can last for several years if the aerosols remain 
in the stratosphere \cite{solomon2011decadal,vernier2011major}.
\par
\textbf{The debate over which paradigm is best is important 
because it effectively determines which approach 
is most suitable to forecast the risk of future decadal or 
multidecadal droughts.} The Bjerknes hypothesis implies that
non-linearity in the climate system makes models the best source 
of information about the future climate. Alternatively, the 
Hasselmann hypothesis suggests that long-term climate risks 
can only be predicted using statistical methods, since the 
longer timescales are controlled by unpredictable high-frequency 
components of the climate system. Distinguishing between 
these two paradigms is therefore especially relevant as 
national and international efforts attempt to generate decadal 
predictions to anticipate and adapt to future climate change.
\par
Of course, the effort to generate predictions about the future 
climate relies heavily on empirical measurements. In order to 
understand how effects manifest in future climate changes it is 
essential to understand the past climate. These empirical 
measurements take varied forms, including ice cores, lake sediment 
analysis, tree-rings, and others. Tree-ring width measurements 
have been the dominant source of information about dec-cen 
behavior due to their high resolution and dating accuracy, in 
addition to their wide distribution, high replication and 
well-understood and validated relationship with local climate. 
The growth limiting relationship between low precipitation and 
high temperatures with tree-rings can be used to reconstruct a 
variety of climatic variables. Tree-ring chronologies are used to 
estimate the climate back in time beyond the start of recorded 
meteorological measurements. Tree-ring chronologies can be analyzed to 
assess long-term departures from average climate, frequency of 
extreme climate, changes in interannual variability in climate, 
and ranges of long-term variability in climate 
\cite{sheppard2010dendroclimatology}. 
\par
\textbf{Two principal approaches have been used to extract 
information about dec-cen behavior from tree-ring records:} 
(1) Band-pass filtering to focus on particular timescales and 
(2) using tree-ring records from specific locations where 
climate modes of interest are expected to manifest. For the 
first method, band-pass filtering is performed after reconstructing 
an annually-resolved target variable to estimate low-frequency 
variability in the past climate \cite{cook2004climate,meko2007medieval}. 
Paleoclimate reconstructions based on tree-rings are often combined 
and then filtered to estimate the time-evolution of dec-cen 
variability across larger regions \cite{d2005variability}. The 
second method relies on the idea that local climate, which directly 
affects tree-ring measurements, is influenced by climate modes that 
exhibit strong dec-cen fluctuations. As a result, sets of tree-ring 
records from one or more “centers of action” within a given mode’s 
spatial pattern can be used to estimate the behavior of that mode. 
Drought \cite{steinman2013isotopic}, wildfire \cite{hessl2004drought}, 
glacial mass balance \cite{larocque2005little} are some of the 
phenomena which have been analyzed using tree-ring reconstructions of 
major dec-cen climate modes. \cite{d2005temperature,gedalof2001interdecadal}. 
\par
\textbf{Despite the widespread use of tree-rings as proxies 
for dec-cen climate behavior, important questions remain about 
their ability describe this low-frequency climate variability.} 
(1) In order to use ring-width measurements for climate 
reconstruction the raw ring-width measurements must be 
"standardized." It is not clear how dec-cen signals in tree-ring 
networks are influenced by this standardization process. 
(2) Paleoclimate data products are not able to address the 
possibility that the structure and amplitude of dec-cen variability 
exhibit variations across space and through time. (3) There is 
disagreement among various estimates of leading decadal climate 
modes derived from tree-rings. (4) Tree-ring growth patterns may not be 
as coherent on long timescales as they are on shorter ones. 
In this preliminary work we focus primarily on (1) and (2).  
\par
Tree-ring "standardization" is the process of fitting a curve to raw 
ring-width data, or detrending, and normalizing the raw ring-width 
data by the fitted curve. The fitted curve is selected to reflect the 
natural growth pattern of the ring-widths. This, in turn, removes 
the natural climate variability and biological processes from the 
raw data. Tree-ring data from networks measurement sites have been 
used in numerous studies by climate scientists, paleoclimatologists, 
and statisticians \cite{tingley2013recent}. However, these studies  
are not designed for studies of low-frequency behavior. These studies 
include relatively small subsets of the total data available 
(hundreds of records, rather than thousands) \cite{briffa2002tree}, 
or use tree-ring data preprocessed using undocumented 
standardization methods \cite{wahl2012comparative}. Because of 
these shortcomings, \textbf{existing community data resources do not 
provide an adequate basis to test the ability of tree-ring data to 
track dec-cen climate variability across large regions.} In addition, 
combining tree-ring chronologies over continental spatial scales 
often shows strong dec-cen variability \cite{cook2004climate} but 
does not allow study of spatial patterns at regional scales.
The North American Drought Atlas \cite{cook2004climate}, which is 
one of the most important products derived from tree-ring data, cannot 
address this issue because it combines measurements from locations 
often several hundred kilometers apart. Thus, by construction, some 
of the spatial autocorrelation structure is filtered out of the Atlas 
along with embedded dec-cen variability.

%\par
%Although tree-rings have been used to reconstruct the Pacific Decadal 
%Oscillation index prior to the 20th century, the set of reconstructions 
%show limited agreement at both high and low frequencies during the 
%18th and 19th centuries \cite{mantua2002pacific}. Similarly, proxy 
%reconstructions of Atlantic multidecadal variability, including 
%those derived from tree-rings, match instrumental records of SST 
%anomalies during the modern period, but do not show coherent 
%multidecadal-scale variability during the pre-instrumental period 
%\cite{kilbourne2013paleoclimate}. These discrepancies may indicate 
%that teleconnection patterns associated with these modes may change 
%over time \cite{holmes2009climate}, which would invalidate the use 
%of a limited number of tree-ring records to make inferences about 
%the behavior of basin-wide dec-cen behavior.
%\par
%Fourth, tree-ring growth patterns may not be as coherent on long 
%timescales as they are on shorter ones. Ring-width records from 
%northwestern New Mexico, which possess some of the strongest climatic 
%signals of any tree-ring data in the hemisphere \cite{george2014imprint}
%show excellent between-site agreement at high frequencies but 
%progressively weaker agreement at decadal and centennial timescales. 
%Because these records are produced from a substantial number 
%of old trees, the discordant behavior at lower frequencies cannot 
%be due to the “fading record” problem \cite{swetnam2010comparing} or the 
%“segment-length curse” \cite{cook1995segment}. New analysis of 
%drought-sensitive tree-ring records used to produce paleohydrological 
%estimates for the Colorado River basin \cite{meko2007medieval} likewise 
%appear to show strong agreement at interannual frequencies, but weaker 
%agreement at longer timescales. Taken together, these 
%preliminary analysis may indicate that moisture-sensitive trees 
%are less sensitive to variations at longer timescales (responding 
%instead to other non-climate factors). On the other hand, these 
%findings may imply that dec-cen variability in the climate system 
%may be less spatially coherent than high-frequency components. 
%At the moment, we do not know whether this low-frequency incoherence 
%is common across the hemispheric network of tree-ring records or if 
%it could be an artifact of age-size standardization procedures.

\par

The following work is a preliminary presentation of methodology 
to be used in overcoming two of the shortcomings of previous 
tree-ring studies. This methodology will be used in exploring 
the effect of different detrending schemes, or standardization, 
on climatology extracted from tree-ring chronologies. Our 
long-term focus is to study dec-cen variability in these 
chronologies. However, in this preliminary presentation, we will 
quantify the amplitude and spatial characteristics of climate 
variability in tree-ring data across mainly the interannual timescale. 
This allows us to showcase the methodology with focus on the ENSO band, 
which is generally better understood than dec-cen signals. We will also use 
our methodology to analyze a synthetic dataset, with a priori known 
signals. First, we perform a novel data synthesis of available tree-ring 
data from the ITRDB. We layout a set of detrending schemes with a range 
of characteristics, most importantly the quality of preserving 
low-frequency content in raw tree-ring data. We also describe the 
MTM-SVD method to be used in analysis of the resulting tree-ring 
chronologies, constructed following the detrending stage. This method 
is selected in order to overcome the estimation bias inherent in Fourier 
analysis and because of the large-scale spatial structure of climatic 
variations. This MTM-SVD analysis allows for signal detection and 
reconstruction, along with significance assessment. A robust null 
hypothesis is used to determine the significance of signals in the 
local fractional variance spectrum, derived from the set of singular 
values.
\par

\section{Data \& Methods}
Tree-ring chronologies are constructed by combining measurements from  
multiple tree-ring segments. In order to build chronologies 
from ring-width measurements these measurements must 
be "standardized." In this step, a curve is fitted to the 
raw ring-width data, or detrended, and the raw ring-width 
data is normalized by the fitted curve. The fitted curve is 
assumed to reflect the natural growth pattern of the ring-widths. 
This, in turn, removes the natural climate variability and 
biological processes from the raw data. Any anomalous growth in 
tree-rings is relative only to a tree's average tree-ring growth at a 
particular age and stage of life. The detrending step allows us to 
extract and compare anomalous growth from separate trees. 
The data resulting from detrending are called ring-width indices. 
In this work we used four different model types for the detrending step. 
Both synthetic data and data from the International Tree-Ring 
Data Bank (ITRDB) was then used in analysis of chronologies. 
Here the primary objective was to validate our methodology. 
This was also done with considerations of future use in 
determining how the detrending approach might effect the 
conclusions drawn from the final chronologies and in 
quantifying the amplitude and spatial structure of low-frequency 
hydroclimate variability in models, proxies, and observations.
\par
In order to analyze the chronologies constructed using our suite of 
detrending models, described in \Cref{detrending}, we rely on 
a combination of multi-taper spectrum estimation methods (MTM) and 
principal component analysis using singular value decomposition (SVD). 
We follow the description of MTM-SVD seen in \cite{mann1999oscillatory}. 
This methodology is used for signal detection and signal reconstruction. 
The MTM-SVD method makes use of two fundamental concepts, (1) spectrum 
estimation and (2) principal component analysis. First, the spectrum 
is estimated for a dataset by making use of linearly independent 
eigentapers. These eigentapers provide $K$ spectral estimates at each 
frequency $f$ in a spatial dimension $M$. Secondly, principal component 
analysis on the multitapered transformed data is used to find the optimal 
linear combinations of eigentapers that maximize the variance associated with 
a particular frequency component. 

\subsection{Data}\label{data:2}

\subsubsection{Synthetic Data}\label{synth:data}

To verify the accuracy of our methodology (see \Cref{mtmsvd}), we use 
the synthetic 'test data' described in \cite{mann1999oscillatory}.
This dataset is constructed so as to contain two irregular oscillatory 
signals and a trend, each widely correlated over the synthetic domain, 
linearly added to noise which exhibits near-neighbor spatial correlation 
and an underlying red noise temporal autocorrelation structure. The 
noise is generated with a first-order autoregressive red noise process. 
The temporal characteristics of the (1) trend, (2) interdecadal oscillation, 
and (3) interannual oscillation are half-cosine, amplitude modulation, 
and both amplitude and phase modulation, respectively. The spatial 
characteristics of these synthetic signals are (1) variable amplitude 
and sign, (2) variable amplitude and phase, and (3) uniform amplitude 
with a variable phase. This dataset has a time-step $\Delta t=1$ month, 
$N=1200$ total time-steps, and $M=25$ different sites. All the 
synthetic signals are summarized in \Cref{synth:info}.

\subsubsection{Last Millennium Ensemble}\label{lme:soi}

To further validate our methodology we use soil moisture data from the 
fully forced run of the Last Millennium Ensemble. This simulation is 
described in detail in \Cref{chap:vol}. In \Cref{chap:vol} we used the 
CAM5 data from this simulation. Here we use the data from the Community 
Land Model 4 (CLM4). This data uses a uniform global grid with lat/lon 
dimensions of 96 x 144. It also uses 15 soil layers. The time period used 
is from 1850 to 2005, sampled monthly. In our analysis we summed the data 
from all soil layers and confined our spatial domain to the western 
United States, with a lat/lon domain of $[30.0,55.0]/[-130.0,-100.0]$.

\subsubsection{ITRDB}\label{itrdb_data}

%data collection
%tree-ring width measurements - Trees will record climate signals
%when climate limits their growth (Fritts 1976)

Trees will record climate signals when climate limits their 
growth \cite{fritts1976}. This is why building tree-ring datasets 
is so essential for studying the past climate. In order to build 
a tree-ring database many steps must be taken. First, tree-ring 
measurements must be taken in the field. This entails physically 
going to a forest measurement site and extracting cores from live 
trees and cross-sections from dead trees. These dead trees are often 
referred to as remnant trees, which can also be cored. These cores are 
typically on the order of one centimeter in diameter with a typical 
length on the order of ten centimeters \cite{vroblesky2008user}. 
Although, this can vary depending on the size of the tree from which 
the core is extracted. The tools used to extract cores are called 
increment borers. Extreme care must be taken to keep borers clean and 
sharp and to extract cores slowly and carefully. This is to ensure high 
quality cores. A particular site for core extraction is selected in 
order to maximize a particular signal. For example, a steep, rocky, 
south facing slope may be selected to find trees under maximum water 
stress, such that growth rates become a strong proxy for precipitation 
\cite{grissino1997,cook2013methods}. 

%ring-width measurements by researchers
%COFECHA, cross-dating, master chronology
\par
After cores are taken in the field they are stored and transported 
back to the lab. Frequently cores are stored in paper straws. Paper 
straws are rigid enough to prevent the cores from warping during air 
drying yet porous enough to allow the cores to dry without developing 
too much mold \cite{phipps1985collecting}. Once the cores are back in 
the lab the painstaking process of recording the ring-widths must be 
completed. This involves recording the width of each tree-ring for 
each core, on paper or through electronic data entry. Before this the 
cores must be prepared. Cores are mounted and finely sanded to achieve 
a flat surface which enables better ring resolution 
\cite{phipps1985collecting}.
After recording the tree-ring widths cores must be crossdated. This 
is the process by which absolute dates are assigned to each tree ring. 
This is done by matching a pattern of tree-rings in a given core to a 
"master chronology" with dates of tree-rings already known 
\cite{phipps1985collecting}. The computer program "COFECHA" is a 
frequently used tool for assessing the quality of crossdating and 
measurement accuracy of tree-ring series. COFECHA provides guidance 
in the process of crossdating by computing the correlation between 
core segments and a master chronology \cite{grissino2000}.

%uses in the literature
%Usually used in single site temp or precip reconstructions. Also
%used to construct fields or streamflow/snowpack - 
%meko 2007, pederson, anchakaitis
\par
Tree-ring series have been used in numerous single-site and multi-site 
temperature and precipitation reconstructions 
\cite{salzer2005,shao2005,graumlich1987,hughes1984}. Networks of sites have 
also seen use in reconstructing other climatological fields 
\cite{seftigen2015,shi2012}. Further uses are seen in streamflow and 
snowpack reconstructions. In \cite{meko2007medieval} tree-ring records 
from remnants were used to extend the record of reconstructed annual 
flows of the Colorado River into the Medieval Climate Anomaly. 
In \cite{maxwell2011multispecies} tree-rings from multiple tree 
species were used to reconstruct streamflow in the Potomac River 
from 950-2001 CE. In \cite{pederson2011unusual} a network of tree-ring 
sites was used to develop multi-century to millennial length snowpack 
reconstructions for the headwaters of the Columbia, Missouri, and 
Colorado Rivers. 

%archival and formatting
%tuscon decadal format
\par
The most commonly used method for archiving and formatting raw tree-ring 
data is called the "Tucson decadal file format." This in an ASCII 
file format designed in the 1970s and in many ways is a digital 
representation of the punch cards widely used for transferring 
software and data at that time. Tucson RWL files are used to store 
raw tree-ring values. The format for a .rwl file uses a core identification 
number in columns 1-6, the decade corresponding to the 10 following tree 
rings in columns 9-12, ring-width values in columns 13-73 using 6 columns 
per measurement, and optional side identification in columns 74-78 
\cite{brewer2014data,grissino1997,brewer2011}. A single file 
following the Tucson RWL format includes all the measurements for a 
given site. This can sometimes be fifty or more core identification 
numbers and data series. Following detrending and standardization a 
file with the same properties (extension .rwi) is used to store the 
ring-width indices. Tucson CRN files are used to store chronology data. 
The format for a .crn file use a series identification number in columns 
1-6, the year of the first chronology value in columns 7-10, and ten data 
value blocks consisting of four integer characters to give the data 
value, followed by a space, then two integer characters specifying the 
number of ring-width index values averaged to give the chronology 
value \cite{brewer2011}. The Tucson file format is that used by the 
International Tree-Ring Data Bank (ITRDB). The ITRDB is the world's primary 
repository of dendrochronology data and as such is widely regarded 
as the standard for storing tree-ring width data. 

%problems associated with the previous sections
%chons rely on best judgement of authors, which is okay for specific 
%sites but bad for large scale sythesis. itrdb doesnt have a strict
%screening process for uploaded datasets. Sometimes detrending makes 
%records explode. Segment curse issues. 
\par
There are a few problems associated with the aforementioned uses, 
archival, and formatting of tree-ring data. Although the Tucson file 
format is widely used it is sometimes completely ignored, and ITRDB 
does not have a strict screening process to handle formatting errors. 
In order to use the entire ITRDB database it is necessary to handle 
these formatting issues ourselves. ITRDB offers pre-constructed 
chronologies for each measurement site but the detrending procedure 
used in constructing these chronologies is determined by the judgement 
of the investigators who uploaded the data to ITRDB. This leaves 
open the possibility that all chronologies use different detrending methods. 
To examine the effect of detrending methods on frequency content in 
chronologies detrending must also be performed ourselves. However, 
in performing this detrending one has to take care to account for errors 
in curve fitting like divergences. Furthermore, as described by 
\cite{cook1995segment}, the minimum resolvable frequency from a 
chronology is determined by the length of the segments used to build 
the chronology. This requires restrictions on segment lengths based 
on the frequency band under investigation. 

\par
The data used for our detrending and chronology construction is 
from the ITRDB. Since this data mostly follows the Tucson format it 
provides raw tree-ring width time series along with information on 
the different sites where the tree-ring cores were initially collected. 
Matlab code was written to download this data from different servers 
and to combine data from $M$ separate sites into a single database. 
This represents a novel data synthesis effort. The ITRDB contains 
over 4000 data sets from 66 countries on 6 continents. Over 800 
investigators have contributed to ITRDB and new data additions average 
roughly 200 collections per year. The ITRDB is described in extensive 
detail in \cite{zhao2019itrdb}. As with the LME data we confined our 
spatial domain to the western United States, with a lat/lon domain 
of $[30.0,55.0]/[-130.0,-100.0]$.


\subsection{Methods}\label{methods:2}

\subsubsection{Detrending}\label{detrending}
Detrending can be further clarified with a simple example, also used in 
\cite{cook1995segment}. Suppose we take cores from $S$ trees, in an ideal 
forest, and take raw ring-width measurements from each of these cores. 
This provides a set of $S$ ring-width series ${R'}_n^{(s)}$ each of which 
are some number $N^s$ years in length. Following natural growth patterns,
a given ring-width series ${R'}_n^{(s)}$ will usually exhibit a decreasing 
trend with age which approaches a finite limit $c^s$ for mature trees. 
Various smooth curves can be fitted to the empirical data ${R'}_n^{(s)}$. 
One model example is the following negative exponential function.

\begin{equation}
{R}_{n}^{(s)} = a^s e^{-b^s n \Delta t}+c^s
\label{negexp}
\end{equation}

The usual practice is to use remove the natural growth pattern from
the raw ring-width data using a model like equation \Cref{negexp} 
\cite{cook1995segment}. The fitted curve can be thought of as a 
series of expected values of ring-with growth that would naturally occur 
in the absence of external forcings like climate variability. Taking 
the ratio of the raw ring-widths ${{R'}_n^{(s)}}$ to the expected 
ring-widths ${{R}_{n}^{(s)}}$ yields a set of dimensionless tree-ring 
"indices". The indices, $I_n^{(s)} = {R'}_n^{(s)} / {R}_n^{(s)}$, 
will have a defined mean of 1.0, since $E[{R'}_n^{(s)}]={R}_{n}^{(s)}$, 
where $E[]$ is the expected value operator. After detrending
all $S$ segments, the set of $S$ indices $I_n^{(s)}$ are pieced together 
by matching end patterns to construct a full chronology $I_n$. 
The process of detrending and transforming the tree-ring variables 
to dimensionless indices is know as "standardization" because it 
tends to equalize the growth variations of trees over time regardless 
of age or size \cite{cook1995segment}.

\par
Matlab code was written to implement the process described in 
\Cref{detrending}. This was parallelized using the Matlab parallel 
toolbox to provide significant speed increase when running on 
multiple processors. The four different $R_n^{(s)}$ model types we 
used for detrending are as follows: (1) "Traditional": Detrending 
performed with a spline function using all segments and $50\%$ of 
variance removed at $2/3$ segment length. (2) "Aggressive": 
Detrending with spline using all segments and $95\%$ of variance 
removed at time-scales greater than 30 years. (3) "Low-frequency 
preserving with all segments": Detrending performed with a negative 
exponential function using all segments, with reversion to spline 
upon failure with exponential. (4.1-4.3) "Long segments": Using segments 
at least 250 years long with correlation between detrending function 
and chronology at least 0.9. (4.1) Using spline with $95\%$ variance 
removed at segment length. (4.2) Using spline with $95\%$ variance 
removed at 250 years. (4.3) Negative exponential used with reversion 
to spline upon failure, with $95\%$ variance removed at segment length. 
These detrending approaches are summarized in \Cref{dt:schemes}. 
Each of these schemes are applied to $M$ measurement sites across 
the time periods summarized in \Cref{dt:cases}.

  
\subsubsection{MTM-SVD}\label{mtmsvd}
In order to analyze the chronologies constructed using our suite of 
detrending models, described in \Cref{detrending}, we rely on 
a combination of multi-taper spectrum estimation methods (MTM) and 
principal component analysis using singular value decomposition (SVD). 
We follow the description of MTM-SVD seen in \cite{mann1999oscillatory}. 
This methodology is used for signal detection and signal reconstruction. 
Matlab code was written to implement this signal detection and 
reconstruction as well. The work in \cite{correa2012multitaper} was 
especially helpful in refining our code for signal reconstruction. 
Again, this code was parallelized to provide significant increases 
in speed when running on multiple processors. 

\myparagraph{Signal Detection}\label{sig:det}

The MTM-SVD method makes use of two fundamental concepts, (1) spectrum 
estimation and (2) principal component analysis. First, the spectrum 
is estimated for a dataset by making use of linearly independent 
eigentapers. These eigentapers provide $K$ spectral estimates at each 
frequency $f$ in a spatial dimension $M$. In all of our analysis we use $K=3$, 
as is standard for most climate applications, because provides an appropriate 
compromise between spectral resolution and accuracy \cite{mann1999oscillatory}.
The spatial dimension $M$ here represents the measurement site location 
for each chronology. Secondly, principal component analysis on the 
multitapered transformed data is used to find the optimal linear 
combinations of eigentapers that maximize the variance associated with 
a particular frequency component. To perform the MTM-SVD analysis we 
first standardize the chronologies to be analyzed for each of the 
$M$ site locations. We remove the mean over the $N$ time steps to 
yield an "anomaly" chronology $\{{x'}_{n}\}^{(m)} = I_n^{(m)}-\bar{I}^{(m)}$ 
and normalize the resulting chronology by its standard deviation 
$\sigma^{(m)}$. We normalize the chronologies again here, even after 
they have been standardized for detrending, to ensure hey have equal 
weight over the time period we conduct the MTM-SVD analysis. To 
represent the data in the frequency domain, we calculate the multitapered 
Fourier transforms for each normalized chronology 
$\{x_{n}\}^{(m)}={x'}^{(m)}/{\sigma^{(m)}}$

\begin{equation}
    {Y_{k}}^{(m)}(f) = \sum_{n=1}^{N} w_{n}^{(k)} x_{n}^{(m)} e^{i2\pi fn \Delta t}
\end{equation}

for a given choice of $K$. We use the $k$-th order discrete prolate 
slepian sequences, $w_{n}^{(k)}$, as our eigentapers. 

A singular value decomposition is performed on the $M x K$ matrix $A(f)$ 
to isolate spatially-coherent signals at frequency $f$.

\begin{equation}
    A(f) = 
       \begin{bmatrix} 
       Y_{1}^{1} & Y_{1}^{2} & \dots \\
          \vdots & \ddots & \\
       Y_{M}^{1} &        & Y_{M}^{K} 
       \end{bmatrix}
\label{Amat}       
\end{equation}

\begin{equation}
    \mathbf{A(f)}=\sum_{k=1}^{K} \lambda_{k}(f) \mathbf{u_{k}}(f)^{\dagger}\mathbf{v_{k}}
\label{svd:eqn}    
\end{equation}

In \Cref{svd:eqn} $\lambda_{k}$ are the complex singular values, 
$\mathbf{u_{k}(f)}$ are orthonormal $M$ dimensional vectors, representing complex spatial 
empirical orthogonal functions ("spatial EOFs"), and $\mathbf{v_{k}}$ are 
orthonormal $K$ dimensional vectors, termed "spectral EOFs" 
\cite{mann1999oscillatory,mann1994global}. The "normalized principle 
eigenvalue" $\lambda_{1}^{2}(f)/{\sum_{k=1}^{K} \lambda_{k}^{2}(f)}$ 
provides a signal detection parameter. We refer to this parameter, 
as a function of frequency, as the "local fractional variance 
spectrum" (LFV). Here "local" refers to locality in frequency space, 
not in physical space. The LFV spectrum varies between 
$1/K$ and unity in magnitude. 
\par
Determining the significance of signals in the LFV requires an accurate 
estimate of chance occurrence. This, in turn, requires a robust null hypothesis. 
To develop this null hypothesis we appeal to the Hasselmann paradigm, that dec-cen 
variability results from high-frequency stochastic processes. 
The hypothesis used here is that the observed behavior arises from statistical 
fluctuations of a spatiotemporal noise process with a spatial correlation 
structure estimated from the dataset itself. Such a noise process is 
implemented by performing a singular value decomposition on randomly sampled 
subsets of $A(f)$ for all frequencies. Subsequently, an LFV can be calculated 
from this process. In calculating the likelihood of a particular LFV value 
appearing in this noise derived spectrum the significance levels can be 
accurately estimated. 

\myparagraph{Signal Reconstruction}\label{sig:con}

The spatial pattern of a signal associated with a peak in the LFV 
spectrum at a frequency $f$ is described by the $M$ dimensional complex 
vector $\mathbf{u_1}$. The $m$-th component of $\mathbf{u_1}$, $u_1^{(m)}$, 
indicates the relative amplitude and phase of the signal at the $m$-th 
measurement site. The full reconstructed spatiotemporal signal is given by,

\begin{equation}
\tilde{x_{n}}^{(m)}={\rm I\!R}\{\sigma^{(m)}u_{1}^{(m)}\bar{A_1}(n \Delta t)e^{i2\pi f n \Delta t}\}
\end{equation}

The temporal envelope $\bar{A_1}(n \Delta t)$ is given by,

\begin{equation}
\label{mann:env}
\bar{A_1}(n \Delta t) = \sum_{k=1}^{K} \xi_{k}^{-1} \lambda_{k}(f)(v_{1}^{(k)})^{*}w_{n}^{(k)},
\end{equation}

where $v_{1}^{(k)}$ is the $k$-th component of the vector 
$\mathbf{v_{1}}(f)$. The $\xi_{k}$ are the bandwidth retention 
factors of the slepian tapers ($\{w_{n}^{k}\}_{n=1}^{N}$). 

Alternatively, we can construct the envelope $\bar{A_1}$ following 
the approach in \cite{park1992envelope}.

\begin{equation}
\label{park:env}
\bar{A_1}(n \Delta t) = \sum_{k=1}^{K} \lambda_k^{-1}(f) Y_k(f) w_n^{(k)}
\end{equation}

It is important to note that the methods for envelope reconstruction 
described in \Cref{mann:env} and \Cref{park:env} rely on a minimum 
norm constraint. However, this, in turn, minimizes the size of the 
envelope and tends to favor an asymptotic approach to zero at the 
boundaries. Since minimization of the norm favors $\bar{A_1}\rightarrow 0$ 
at the boundaries this constraint might not be appropriate for secular 
trends. It is possible to instead impose minimum conditions on the first 
derivative, second derivative, or a linear combination of these constraints 
\cite{mann1999oscillatory}. Minimizing the fist derivative instead favors 
zero slope at the envelope boundaries. This does not discriminate against 
a near zero frequency trend but may not be appropriate for signals which 
vary rapidly at the envelope boundaries. Minimizing the second derivative 
favors neither zero slope or zero mean at the envelope boundaries. We 
implemented options to minimize both the first and second derivatives in 
our code in order to accommodate signals with varying characteristics. 

The spatial pattern can be represented
by the complex field,

\begin{equation}
\hat{x}^{(m)}=\sigma^{(m)}u_{1}^{(m)}A_{rms}(f)
\label{eofs}
\end{equation}

The temporal signal can be derived from the spatiotemporal signal 
by summing over all grid points,

\begin{equation}
x_{n} = \sum_{m=1}^{M} \tilde{x_{n}}^{(m)}
\end{equation}

The pattern described by \Cref{eofs} is accurate for signals 
associated with narrow frequency bands. However, in order to 
describe the pattern associated with a broad frequency range it may 
be necessary to use an evolutive procedure. This calculates \Cref{eofs} 
for overlapping windows of the total time period and averages over 
the windows to get a final spatial pattern. We implemented an 
evolutive calculation for \Cref{eofs} and LFV in our code. This 
evolutive process is also further described in \cite{mann1999oscillatory}.

\section{Results}\label{results:2}

\subsection{Synthetic data analysis}\label{anal:synth}
We first present results from analysis of the synthetic dataset 
described in \Cref{synth:data}. All the synthetic signals are also 
described in \Cref{synth:info}. The analysis here highlights the steps 
of the MTM-SVD methodology described in \Cref{mtmsvd}. We reproduce the
local fractional variance (LFV) spectrum for the synthetic data 
\cite{mann1999oscillatory}, shown in \Cref{mann:conf}, using
the approach described in \Cref{sig:det}. \Cref{mann:conf} also 
includes the null hypothesis spectrum, used to determine the significance 
of signals seen in the LFV spectrum. \Cref{mann:conf:evol} shows the 
evolutive calculation of the LFV spectrum for the synthetic data. 
Using the approach described in \Cref{sig:con} we reconstruct the 
temporal and spatial signals associated with the secular trend, 
interdecadal oscillation, and 
interannual oscillation in the synthetic dataset. 
These constructions are shown in 
\Cref{sec:ts,sec:phase,interdec:ts,interdec:phase,interann:ts,interann:phase}. 
We see in these figures that we are able to reproduce the 
phase plots seen in \cite{mann1999oscillatory} for the secular 
trend and the interdecadal oscillation. The time series plots for the 
secular and interdecadal signals also show close agreement with the 
results in \cite{mann1999oscillatory}. In the interannual phase plot 
we see the expected poleward migration but not as significantly as in 
\cite{mann1999oscillatory}. The interannual time series also reflects 
the same qualitative behavior but is in not as close agreement as the 
other signals. In \cite{mann1999oscillatory} the time series for the 
interannual range was calculated using multivariate projection filtering, 
which was not implemented here. The interannual phase plot was computed 
using the evolutive process we referenced in \Cref{sig:con}. However, the 
reason for the remaining differences remains unclear. It is possible 
this is due to the non-uniqueness of the solutions to the signal 
reconstruction approach. The number of windows used in the evolutive 
procedure, for reconstruction the interannual spatial pattern in 
\cite{mann1999oscillatory}, is also unknown and could be a factor in 
the discrepancy. Further, the method for combining signals of 
different frequencies is not explained in detail in 
\cite{mann1999oscillatory}. 

\subsection{LME Analysis}\label{anal:lme}
As with the synthetic data we performed signal detection and reconstruction 
on the LME data described in \Cref{lme:soi}. The LFV spectrum for this 
data is shown in \Cref{lme:lfv}. We see only one peak in the interannual 
range which exceeds 90\% significance. The temporal and phase reconstructions 
for this signal are shown \Cref{lme:ts,lme:phase}. The phase reconstruction 
displays significant coherence, as expected from a signal in the ENSO band. 

\subsection{ITRDB Analysis}\label{anal:itrdb}


\subsubsection{Signal Detection}
We apply all the detrending models, described in \Cref{dt:schemes}, 
to all the cases described in \Cref{dt:cases}. The sites used in 
detrending are required to have data for at least $90\%$ of the specified
time period. This results in a different number of sites being available
for the different time periods listed in \Cref{dt:cases}. 
The LFV spectra for each detrending model and case combination are 
shown in comparison and side-by-side figures. We also include evolutive 
LFV spectra plots for each case and detrending model. 
Detrending scheme 1 spectra are shown in 
\Cref{compsig1,conf1:1500,conf1:1700,evol1:1500,evol1:1700}. 
Detrending scheme 2 spectra are shown in 
\Cref{compsig2,conf2:1500,conf2:1700,evol2:1500,evol2:1700}. 
Detrending scheme 3 spectra are shown in 
\Cref{compsig3,conf3:1500,conf3:1700,evol3:1500,evol3:1700}. 
Detrending scheme 4.1 spectra are shown in 
\Cref{compsig4.1,conf4.1:1500,conf4.1:1700,evol4.1:1500,evol4.1:1700}. 
Detrending scheme 4.2 spectra are shown in 
\Cref{compsig4.2,conf4.2:1500,conf4.2:1700,evol4.2:1500,evol4.2:1700}. 
Detrending scheme 4.3 spectra are shown in 
\Cref{compsig4.3,conf4.3:1500,conf4.3:1700,evol4.3:1500,evol4.3:1700}. 
LFV spectra for all detrending schemes, for 1500-2010, are shown 
in \Cref{lfv:all:1500}. The total average LFV difference between time 
periods is seen not to significantly exceed $15\%$ for any of the 
detrending models. For each case and detrending model combination we see 
significant secular, interdecadal, and interannual signals. These 
frequencies associated with detected signals exceeding 99\% significance 
are summarized in \Cref{fs99}. The frequencies with signals in between 
95\% and 99\% significance are summarized in \Cref{fs95}. 

\subsubsection{Signal Reconstruction}
For each detrending model we reconstruct the time series and spatial 
patterns associated with the signals exceeding 99\% significance. 
First we focus on the signals in the interannual range, or the ENSO band. 
Our spatial focus is also narrowed the Western United States, where we 
expect ENSO signals to manifest. Again, the reason for this narrowed 
focus is to highlight the methodology. The larger time period 1500-2010 
provides increased frequency resolution, and signals in the interannual 
range appear stronger for this case. Thus, we also focus our reconstruction 
efforts on this time period alone. The average of the 1500-2010 interannual 
signals for all detrending schemes are shown in 
\Cref{ts1avg,map1avg,ts2avg,map2avg,ts3avg,map3avg,ts4.1avg,map4.1avg,ts4.2avg,map4.2avg,ts4.3avg,map4.3avg}. 
We also show the 1500-2010 temporal and spatial reconstructions, in the 
interannual range, for the individual signals going into the averages. 
Those for detrending scheme 1 are shown in 
\Cref{ts1p1,map1p1,ts1p2,map1p2,ts1p3,map1p3,ts1p4,map1p4}. 
Detrending scheme 2 reconstructions are shown in 
\Cref{ts2p1,map2p1,ts2p2,map2p2,ts2p3,map2p3,ts2p4,map2p4}.
Detrending scheme 3 reconstructions are shown in 
\Cref{ts3p1,map3p1,ts3p2,map3p2,ts3p3,map3p3}.
Detrending scheme 4.1 reconstructions are shown in 
\Cref{ts4.1p1,map4.1p1,ts4.1p2,map4.1p2,ts4.1p3,map4.1p3}.
Scheme 4.2 reconstructions are shown in 
\Cref{ts4.2p1,map4.2p1,ts4.2p2,map4.2p2,ts4.2p3,map4.2p3,ts4.2p4,map4.2p4}.
Scheme 4.3 reconstructions are shown in 
\Cref{ts4.3p1,map4.3p1,ts4.3p2,map4.3p2,ts4.3p3,map4.3p3,ts4.3p4,map4.3p4,ts4.3p5,map4.3p5,ts4.3p6,map4.3p6}.
\par
Across all of the spatial reconstructions we see coherence expected in a 
strong ENSO signal. In the temporal reconstructions we see significant 
amplitude modulation and in the average temporal signals we also see the 
expected frequency modulation. This aligns with what we see in the temporal 
reconstruction for the synthetic interannual signal, described in 
\Cref{anal:synth}. We do not see significant differences between 
reconstructions using different detrending models in the interannual 
frequency range. In \Cref{nino1,nino2,nino3,nino4.1,nino4.2,nino4.3} we 
compare the average interannual time signals to NINO3.4 data. We normalize 
both the reconstructed signals and NINO3.4 by their respective maximum 
values. NINO3.4 data is sampled monthly so we annually average this as well. 
We see generally good agreement between the reconstructed signals 
and NINO3.4. The discrepancy between some peak positions is easily 
attributed to the fact that NINO3.4 is based on SST data and our time 
series reconstructions are based on tree-ring widths. There is likely a 
phase shift between the signals appearing in SSTs and tree-rings. 
\par
We also perform reconstructions on the dec-cen signals above 99\% 
significance. The first detrending scheme which features a lower 
frequency signal above 99\% significance is scheme 2. This interdecadal 
mode is shown in \Cref{ts2pl1,map2pl1}. 
Detrending scheme 4.1 features 
two interdecadal modes above 99\% significance. These signals are shown 
in \Cref{ts4.1pl1,map4.1pl1,ts4.1pl2,map4.1pl2}. 
Detrending scheme 4.2 features a secular mode and an interdecadal mode, 
shown in 
\Cref{ts4.2pl1,map4.2pl1,ts4.2pl2,map4.2pl2}. 
Detrending scheme 4.3 also includes a secular mode and an interdecadal 
mode, shown in 
\Cref{ts4.3pl1,map4.3pl1,ts4.3pl2,map4.3pl2}. 
We can infer from these reconstructions that the detrending scheme 
affects the low-frequency climate variability seen in reconstructions. 
This was our main motivation for using a suite of detrending schemes 
to construct chronologies. 

\par
%what will this data enable?
%new tests of reconstruction sensitivity to detrending
%new code base for developing "screened" versions of the ITRDB
%for other applications. 
%re-do colorado streamflow records
%where will it be?
%include link to repo
%ecrl website
%RDMS Mann Library, doi
%NOAA, NAU
Beyond our results here this data will enable countless 
other reconstruction efforts. Previous tree-ring based 
reconstructions can be re-done using different detrending 
schemes to examine reconstruction sensitivity to detrending. 
The code base developed here will allow "screened" versions 
of the ITRDB to be generated to explore the effect of 
various restrictions on the data used for reconstructions. 
For instance, the reconstruction of the Colorado River 
streamflow record could be re-done and studied under a variety 
of input data conditions. Screened versions of the ITRDB 
could also be used for applications beyond reconstructions. 
Our code base will be hosted on bitbucket at 
https://bitbucket.org/ecrl/itrdb\_synth. The data products 
generated here will be hosted on the Cornell University 
Emergent Climate Risk Lab server at ecrl.eas.cornell.edu. Through 
the Research Data Management Group at Cornell University a DOI 
will be minted to enable hosting of this data by the Mann 
Library Digital Archives. Additionally, the data will be 
hosted by the National Oceanic and Atmospheric Administration 
as well as Northern Arizona University. 


\section{Discussion \& Conclusion}
In this work we synthesized a database of raw tree-ring 
data samples from the ITRDB. We described a complete 
methodology for detrending this raw ring-width data, 
constructing chronologies, and analyzing the resulting 
chronologies. We described the advantages of our analysis 
methodology and using MTM-SVD to perform signal detection 
and signal reconstruction. All these steps, synthesizing 
the database, detrending, constructing chronologies, 
signal detection, and signal reconstruction, were implemented 
in Matlab code and parallelized for increased speed when 
running on multiple processors. We applied MTM-SVD analysis 
to a synthetic dataset with a priori known signals and 
reproduced results to demonstrate the efficacy of this analysis. 
Our implementation of the MTM-SVD analysis is shown to work as 
expected, reproducing known results, and public release will 
make it easier for other data scientists to deepen our understanding 
of low-frequency variability. Furthermore, we applied our entire 
methodology to ITRDB to construct chronologies and to analyze 
these chronologies across both interannual and dec-cen timescales. 
We see expected coherence in the spatial patterns for the interannual 
timescale but no significant differences between different detrending 
models. We do, however, see that different detrending schemes 
may produce different low-frequency signals. That dec-cen variability 
is sensitive to detrending is an important result.
\par
Our work here has sought to address two main objectives: 
(1) To provide comprehensive paleoclimate data sets which  
can serve as model test data sets analogous to instrumental 
observations; (2) to enable synthesis of paleoclimate data 
and climate model output to understand the response of the 
longer-term and higher magnitude variability of the climate 
system. Our first objective can be accomplished by developing 
and publicly releasing a new homogenized database of tree-ring 
width and density records optimized to recover dec-cen 
variability and designed to be compared against long climate 
model simulations. Our second objective will be achieved by 
comparing dec-cen patterns and amplitudes from climate 
simulations against those derived from the observed and 
simulated tree growth during the last millennium.
\par
In future work we will use the methodology described above to
quantify the amplitude of low-frequency hydroclimate 
variability in models, proxies, and observations. In doing this
we will take advantage of an ensemble of last millennium 
numerical climate models produced by the National Center 
for Atmospheric Research (NCAR). As described in \Cref{chap:vol}, 
we have already applied dynamical downscaling to some of the model 
output produced by NCAR. We will expand the set of models used in 
this dynamical downscaling and also apply statistical downscaling 
techniques to this ensemble. This will include one ultra-high 
(temporal) resolution control simulation. The results of this 
downscaling application with be used to drive forward models of 
tree-ring growth. We have already taken significant steps 
to reprocess all tree-ring width and density measurements from 
the Northern Hemisphere archived by the ITRDB to optimize the 
amplitude and spatial structure of dec-cen variability. We will 
finish this effort to produce an optimized tree-ring dataset. 
Both of these products--the high-resolution tree-ring simulations 
from climate models and the new dec-cen optimized tree-ring 
network--will be evaluated using the MTM-SVD methodology.
\par
%Many recent “pseudo-proxy” studies have sought improve our 
%understanding of climate, including dec-cen variability. 
%These studies attempt to understand the limitations, assumptions, 
%and strengths of various reconstruction methods using forward models 
%of paleoclimate archives \cite{smerdon2010pseudoproxy}. Other 
%efforts have used forward models to study climate variability 
%that would be encoded into proxies. These efforts have explored the 
%effect of driving these forward models with climate model output 
%\cite{thompson2011signatures} and noise \cite{steig2015influence}. 
%These studies offer some insights into how non-climatic processes 
%might effect the effort to understand dec-cen variability. 
%(1) It is clear that low-frequency variability in proxies is not 
%necessarily evidence of low-frequency variability in climate. If 
%a natural climate system, which includes a storage component, 
%records a low-frequency climate signal the corresponding proxy 
%record can show low-frequency behavior resulting from from 
%non-climatic sources \cite{steig2015influence}. The idea of proxies 
%modulating high-frequency climate variability is an example of the 
%Hasselmann hypothesis. (2) Driving forward models with climate model 
%output provides a powerful framework in which dec-cen variability 
%can be studied. This is because the “true” signals of the underlying 
%climate field are known a priori. 
%\par
Future work will use tree-ring data and multi-century climate model 
simulations to understand climate variability at timescales longer 
than the instrumental record. We have made progress on this front 
using tree-ring data, and have begun to estimate the magnitude of 
past spatiotemporal climate variations on dec-cen timescale. 
Furthermore, we will (1) determine the role of external versus 
internal forcings in driving dec-cen variability in tree-ring growth 
and climate model driven simulated growth; and (2) evaluate the 
accuracy of models in simulating the amplitude and spatial patterns 
of dec-cen variability. These activities will, in turn, improve 
our understanding of the full magnitude dec-cen variability, 
its spatial footprint, and hence its implications for future 
megadrought risk estimates.


%\bibliographystyle{spphys}
\bibliography{phd_thesis} 

\clearpage
\newpage

%chapter 2 tables


\begin{table}[!tbp]
\centering
\resizebox{\textwidth}{!}{\begin{tabular}{lrrrr}
\toprule
    Signal & Spatial Char & Temporal Char & $T$ (years) & $f$ (cyc/yr) \\ 
\midrule
    Secular & Variable amp/sign & half-cosine trend & 200 & 0.005 \\ 
    Interdecadal & Variable amp/phase & amp mod & 15 & 0.065 \\ 
    Interannual & Uniform amp/var phase & amp/freq mod & 3-5 & 0.33-0.2 \\ 
    Red Noise & near-neighbor spat. correlation & AR(1) red noise &  &  \\ 
\bottomrule
\end{tabular}}
\caption{Synthetic dataset signals}
\label{synth:info}
\end{table}


\begin{table}[!tbp]
\centering
\begin{tabular}{lrrr}
\toprule
             Scheme Number &     Description \\
\midrule

1 & Spline, all segments, $50\%$ of $\sigma^2$ removed at $2/3$ length \\
2 & Spline, all segments, $95\%$ of $\sigma^2$ removed above 30 years \\
3 & Exponential, all segments, revert to spline on failure \\
4.1 & Spline, segments $\geq$ 250 years, 0.9 correlation \\
& required between chronology and detrending model, \\
& $95\%$ of $\sigma^2$ removed at full length  \\
4.2 & Spline, segments $\geq$ 250 years, 0.9 correlation \\
& required between chronology and detrending model, \\
& $95\%$ of $\sigma^2$ removed at 250 years  \\
4.3 & Exponential, segments $\geq$ 250 years, 0.9 correlation \\
& required between chronology and detrending model, \\
& revert to spline on failure \\

\bottomrule
\end{tabular}
\caption{Detrending Schemes}
\label{dt:schemes}
\end{table}

\begin{table}[!tbp]
\centering
\begin{tabular}{lrrr}
\toprule
             Case Number & Time Period & Number of Sites \\
\midrule

1 & 1500-2010 & 467 \\
2 & 1700-1850 & 1220 \\

\bottomrule
\end{tabular}
\caption{Detrending Case Summary}
\label{dt:cases}
\end{table}

\begin{table}[!tbp]
\centering
\resizebox{\textwidth}{!}{\begin{tabular}{lrrr}
\toprule
             Detrending Scheme & Frequencies (cyc/yr) \\
\midrule

1 & 0.1301, 0.1536, 0.1893, 0.1913 \\
2 & 0.0113, 0.1301, 0.1536, 0.1893, 0.1913 \\
3 & 0.1296, 0.1536, 0.1893 \\
4.1 & 0.0435, 0.0744, 0.1296, 0.1893, 0.1908, 0.2177 \\
4.2 & 0.0020, 0.0318, 0.1306, 0.2177, 0.3165, 0.3180 \\
4.3 & 0.002, 0.0435, 0.1296, 0.1893, 0.1913, 0.2177, 0.3772, 0.3801 \\

\bottomrule
\end{tabular}}
\caption{Frequencies with significance above 99\%}
\label{fs99}
\end{table}

\begin{table}[!tbp]
\centering
\resizebox{\textwidth}{!}{\begin{tabular}{lrrr}
\toprule
             Detrending Scheme & Frequencies (cyc/yr) \\
\midrule

1  & 0.0382, 0.0396, 0.0431, 0.0744, 0.1654, 0.1703, 0.1732, 0.1776, \\
& 0.1864, 0.2177, 0.2402, 0.2446, 0.2671, 0.3102, 0.3165, 0.3185, 0.4012 \\
2 & 0.0205, 0.0235, 0.0411, 0.0435, 0.0744, 0.1732, 0.1776, 0.1864, \\
& 0.2177, 0.2446, 0.2671, 0.3801 \\
3 & 0.0015, 0.0744, 0.1913, 0.2177 \\
4.1 & 0.0719, 0.1243, 0.1336, 0.1531, 0.1556, 0.1732, 0.1776, 0.2202, \\
& 0.2676, 0.317, 0.3185, 0.3777, 0.3796, 0.4706 \\
4.2 & 0.0235, 0.0431, 0.1629, 0.1893, 0.1913, 0.1937, 0.2676, 0.3395, \\
& 0.3777, 0.3801, 0.3992, 0.4604, 0.4902 \\
4.3 & 0.0235, 0.0744, 0.1243, 0.1336,0 .1531, 0.1683, 0.1732, 0.1776, \\
& 0.2202, 0.2676, 0.3170, 0.3185, 0.4653 \\

\bottomrule
\end{tabular}}
\caption{Frequencies with significance between 95\% and 99\%}
\label{fs95}
\end{table}


%% chapter 2 figures

\begin{figure}[!tbp]
\centering
\includegraphics[width=\textwidth,height=0.95\textheight,keepaspectratio]{./figures/mdat_conf.eps}
\caption{Local fractional variance for synthetic data. Red noise used for significance comparison also plotted. We see secular, interdecadal, and interannual signals}
\label{mann:conf}
\end{figure}

\begin{figure}[!tbp]
\centering
\includegraphics[width=\textwidth,height=0.95\textheight,keepaspectratio]{./figures/mann_evol_lfv.eps}
\caption{Evolutive local fractional variance for synthetic data. The window size used is 40 years.}
\label{mann:conf:evol}
\end{figure}

\begin{figure}[!tbp]
\centering
\begin{minipage}[b]{0.45\textwidth}
\includegraphics[width=\textwidth,height=0.95\textheight,keepaspectratio]{./figures/sec_ts.eps}
\caption{Temporal reconstruction for synthetic secular mode. Frequency is 0.005 cyc/yr. Period is 200 years.}
\label{sec:ts}
\end{minipage}
\hfill
\begin{minipage}[b]{0.45\textwidth}
\includegraphics[width=\textwidth,height=0.95\textheight,keepaspectratio]{./figures/sec_phase.eps}
\caption{Spatial signal reconstruction for synthetic secular mode. Frequency is 0.005 cyc/yr. Period is 200 years.}
\label{sec:phase}
\end{minipage}
\end{figure}

\begin{figure}[!tbp]
\centering
\begin{minipage}[b]{0.45\textwidth}
\includegraphics[width=\textwidth]{./figures/interdec_ts.eps}
\caption{Temporal reconstruction for synthetic interdecadal mode. Frequency is 0.0065 cyc/yr. Period is 15 years.}
\label{interdec:ts}
\end{minipage}
\hfill
\begin{minipage}[b]{0.45\textwidth}
\includegraphics[width=\textwidth]{./figures/interdec_phase.eps}
\caption{Spatial signal reconstruction for synthetic interdecadal mode. Frequency is 0.0065 cyc/yr. Period is 15 years.}
\label{interdec:phase}
\end{minipage}
\end{figure}

\begin{figure}[!tbp]
\centering
\begin{minipage}[b]{0.45\textwidth}
\includegraphics[width=\textwidth]{./figures/interann_ts.eps}
\caption{Temporal signal reconstruction for synthetic interannual mode. Frequency is 0.33-0.2 cyc/yr. Period is 3-5 years.}
\label{interann:ts}
\end{minipage}
\hfill
\begin{minipage}[b]{0.45\textwidth}
\includegraphics[width=\textwidth]{./figures/interann_phase.eps}
\caption{Spatial signal reconstruction for synthetic interannual mode. Frequency is 0.33-0.2 cyc/yr. Period is 3-5 years.}
\label{interann:phase}
\end{minipage}
\end{figure}

\begin{figure}[!tbp]
\centering
\includegraphics[width=\textwidth,height=0.95\textheight,keepaspectratio]{./figures/LME_conf.eps}
\caption{LFV spectrum for LME soil moisture data. We see an interannual signal which exceeds 90\% significance.}
\label{lme:lfv}
\end{figure}

\begin{figure}[!tbp]
\centering
\begin{minipage}[b]{0.45\textwidth}
\includegraphics[width=\textwidth]{./figures/LME_ts_23878.eps}
\caption{Temporal signal reconstruction for LME interannual mode. Frequency is $f=0.239$ cyc/yr.}
\label{lme:ts}
\end{minipage}
\hfill
\begin{minipage}[b]{0.45\textwidth}
\includegraphics[width=\textwidth]{./figures/LME_phase_23878.eps}
\caption{Spatial signal reconstruction for LME interannual mode. Frequency is $f=0.239$ cyc/yr.}
\label{lme:phase}
\end{minipage}
\end{figure}


\begin{figure}[!tbp]
\centering
\includegraphics[width=\textwidth,height=0.95\textheight,keepaspectratio]{./figures/itrdb_Lyear_spline50pct67wl_lateHolocene_seglen0_1500-2010vs1700-1850_compsig.eps}

\noindent\fbox{\parbox{0.45\textwidth}{%
\centering
total average difference: 0.125}}

\caption{LFV comparison for detrending scheme 1. Spline fit with all segments, 50\% of variance removed at $2/3$ segment length.}

\label{compsig1}
\end{figure}



\begin{figure}[!tbp]
\centering
\begin{minipage}[b]{0.45\textwidth}
\includegraphics[width=\textwidth]{./figures/itrdb_Lyear_spline50pct67wl_lateHolocene_seglen0_1500-2010_conf.eps}

\caption{LFV plot for detrending scheme 1 (1500-2010). We see signals in the interannual range which exceed 99\% significance. We also see signals in the interdecadal range which exceed 95\% significance.}
\label{conf1:1500}

\end{minipage}
\hfill
\begin{minipage}[b]{0.45\textwidth}
\includegraphics[width=\textwidth]{./figures/itrdb_Lyear_spline50pct67wl_lateHolocene_seglen0_1700-1850_conf.eps}
\caption{LFV plot for detrending scheme 1 (1700-1850). We signals in the quasidecadal and quasibiennial ranges which exceed 99\% significance. We also see signals in the interannual range which exceed 95\% significance. }
\label{conf1:1700}
\end{minipage}
\end{figure}

\begin{figure}[!tbp]
\centering
\begin{minipage}[b]{0.45\textwidth}
\includegraphics[width=\textwidth]{./figures/itrdb_Lyear_spline50pct67wl_lateHolocene_seglen0_1500-2010_evol_lfv.eps}

\caption{Evolutive LFV plot for detrending scheme 1 (1500-2010).}
\label{evol1:1500}

\end{minipage}
\hfill
\begin{minipage}[b]{0.45\textwidth}
\includegraphics[width=\textwidth]{./figures/itrdb_Lyear_spline50pct67wl_lateHolocene_seglen0_1700-1850_evol_lfv.eps}
\caption{Evolutive LFV plot for detrending scheme 1 (1700-1850).}
\label{evol1:1700}
\end{minipage}
\end{figure}

\begin{figure}[!tbp]
\centering
\includegraphics[width=\textwidth,height=0.95\textheight,keepaspectratio]{./figures/itrdb_Lyear_spline95pct30yrs_lateHolocene_seglen0_1500-2010vs1700-1850_compsig.eps}

\noindent\fbox{\parbox{\textwidth}{%
\centering
total average difference: 0.151}}

\caption{LFV comparison for detrending scheme 2. Spline fit with all segments, 95\% of variance removed at timescales equal to 30 years and longer.}

\label{compsig2}
\end{figure}

\begin{figure}[!tbp]
\centering
\begin{minipage}[b]{0.45\textwidth}
\includegraphics[width=\textwidth]{./figures/itrdb_Lyear_spline95pct30yrs_lateHolocene_seglen0_1500-2010_conf.eps}
\caption{LFV plot for detrendig scheme 2 (1500-2010). We see signals in the secular and interannual and ranges which exceed 99\% significance. We also see signals in the interdecadal range which exceed 95\% significance.}
\label{conf2:1500}
\end{minipage}
\hfill
\begin{minipage}[b]{0.45\textwidth}
\includegraphics[width=\textwidth]{./figures/itrdb_Lyear_spline95pct30yrs_lateHolocene_seglen0_1700-1850_conf.eps}
\caption{LFV plot for detrending scheme 2 (1700-1850). We see signals in the secular and interdecadal ranges which exceed 99\% significance. Some interannual signals approach 95\% significance.}
\label{conf2:1700}
\end{minipage}
\end{figure}

\begin{figure}[!tbp]
\centering
\begin{minipage}[b]{0.45\textwidth}
\includegraphics[width=\textwidth]{./figures/itrdb_Lyear_spline95pct30yrs_lateHolocene_seglen0_1500-2010_evol_lfv.eps}

\caption{Evolutive LFV plot for detrending scheme 2 (1500-2010).}
\label{evol2:1500}

\end{minipage}
\hfill
\begin{minipage}[b]{0.45\textwidth}
\includegraphics[width=\textwidth]{./figures/itrdb_Lyear_spline95pct30yrs_lateHolocene_seglen0_1700-1850_evol_lfv.eps}
\caption{Evolutive LFV plot for detrending scheme 2 (1700-1850).}
\label{evol2:1700}
\end{minipage}
\end{figure}


\begin{figure}[!tbp]
\centering
\includegraphics[width=\textwidth,height=0.95\textheight,keepaspectratio]{./figures/itrdb_Lyear_expspline50pct67wl_lateHolocene_seglen0_1500-2010vs1700-1850_compsig.eps}

\noindent\fbox{\parbox{\textwidth}{%
\centering
total average difference: 0.118}}

\caption{LFV comparison for detrending scheme 3. Exponential fit with all segments, reverting to spline fit upon exponential fit failure.}

\label{compsig3}
\end{figure}

\begin{figure}[!tbp]
\centering
\begin{minipage}[b]{0.45\textwidth}
\includegraphics[width=\textwidth]{./figures/itrdb_Lyear_expspline50pct67wl_lateHolocene_seglen0_1500-2010_conf.eps}
\caption{LFV plot for detrending scheme 3 (1500-2010). We see interannual signals which exceed 99\% significance. We also see signals in the interdecadal range which exceed or approach 95\% significance.}
\label{conf3:1500}
\end{minipage}
\hfill
\begin{minipage}[b]{0.45\textwidth}
\includegraphics[width=\textwidth]{./figures/itrdb_Lyear_expspline50pct67wl_lateHolocene_seglen0_1700-1850_conf.eps}
\caption{LFV plot for detrending scheme 3 (1700-1850). We see signals in both the interdecadal and quasibiennial ranges which exceed 99\% significance. We also see signals in the interannual range which exceed 95\% significance.}
\label{conf3:1700}
\end{minipage}
\end{figure}

\begin{figure}[!tbp]
\centering
\begin{minipage}[b]{0.45\textwidth}
\includegraphics[width=\textwidth]{./figures/itrdb_Lyear_expspline50pct67wl_lateHolocene_seglen0_1500-2010_evol_lfv.eps}

\caption{Evolutive LFV plot for detrending scheme 3 (1500-2010).}
\label{evol3:1500}

\end{minipage}
\hfill
\begin{minipage}[b]{0.45\textwidth}
\includegraphics[width=\textwidth]{./figures/itrdb_Lyear_expspline50pct67wl_lateHolocene_seglen0_1700-1850_evol_lfv.eps}
\caption{Evolutive LFV plot for detrending scheme 3 (1700-1850).}
\label{evol3:1700}
\end{minipage}
\end{figure}


\begin{figure}[!tbp]
\centering
\includegraphics[width=\textwidth,height=0.95\textheight,keepaspectratio]{./figures/itrdb_Lyear_spline95pct100wl_lateHolocene_seglen251_1500-2010vs1700-1850_compsig.eps}

\noindent\fbox{\parbox{\textwidth}{%
\centering
total average difference: 0.148}}

\caption{LFV comparison for detrending scheme 4.1. Spline fit with only segments at least 250 years long, 95\% of variance removed over timescales equal to the full segment length.}

\label{compsig4.1}
\end{figure}

\begin{figure}[!tbp]
\centering
\begin{minipage}[b]{0.45\textwidth}
\includegraphics[width=\textwidth]{./figures/itrdb_Lyear_spline95pct100wl_lateHolocene_seglen251_1500-2010_conf.eps}
\caption{LFV plot for detrending scheme 4.1 (1500-2010). Here we see signals in both the interdecadal and interannual ranges which exceed 99\% significance. We also see signals in the quasibiennial range which exceed 95\% significance.}
\end{minipage}
\hfill
\label{conf4.1:1500}
\begin{minipage}[b]{0.45\textwidth}
\includegraphics[width=\textwidth]{./figures/itrdb_Lyear_spline95pct100wl_lateHolocene_seglen251_1700-1850_conf.eps}
\caption{LFV plot for detrending scheme 4.1 (1700-1850). Here we see signals in both the interdecadal and quasibiennial ranges which exceed 99\% significance. We also see signals in the interannual range which exceed 95\% significance.}
\label{conf4.1:1700}
\end{minipage}
\end{figure}

\begin{figure}[!tbp]
\centering
\begin{minipage}[b]{0.45\textwidth}
\includegraphics[width=\textwidth]{./figures/itrdb_Lyear_spline95pct100wl_lateHolocene_seglen251_1500-2010_evol_lfv.eps}

\caption{Evolutive LFV plot for detrending scheme 4.1 (1500-2010).}
\label{evol4.1:1500}

\end{minipage}
\hfill
\begin{minipage}[b]{0.45\textwidth}
\includegraphics[width=\textwidth]{./figures/itrdb_Lyear_spline95pct100wl_lateHolocene_seglen251_1700-1850_evol_lfv.eps}
\caption{Evolutive LFV plot for detrending scheme 4.1 (1700-1850).}
\label{evol4.1:1700}
\end{minipage}
\end{figure}


\begin{figure}[!tbp]
\centering
\includegraphics[width=\textwidth,height=0.95\textheight,keepaspectratio]{./figures/itrdb_Lyear_spline95pct250yrs_lateHolocene_seglen251_1500-2010vs1700-1850_compsig.eps}

\noindent\fbox{\parbox{\textwidth}{%
\centering
total average difference: 0.132}}

\caption{LFV comparison for detrending scheme 4.2. Spline fit with only segments at least 250 years long, 95\% of variance removed timescales equal to 250 years and longer.}

\label{compsig4.2}
\end{figure}

\begin{figure}[!tbp]
\centering
\begin{minipage}[b]{0.45\textwidth}
\includegraphics[width=\textwidth]{./figures/itrdb_Lyear_spline95pct250yrs_lateHolocene_seglen251_1500-2010_conf.eps}
\caption{LFV plot for detrending scheme 4.2 (1500-2010). Here we see signals in the interdecadal and interannual ranges which exceed 99\% significance. We also see signals in the quasibiennial range which exceed 95\% significance.}
\label{conf4.2:1500}
\end{minipage}
\hfill
\begin{minipage}[b]{0.45\textwidth}
\includegraphics[width=\textwidth]{./figures/itrdb_Lyear_spline95pct250yrs_lateHolocene_seglen251_1700-1850_conf.eps}
\caption{LFV plot for detrending scheme 4.2 (1700-1850). We see signals equal to or exceeding 99\% significance in both the interdecadal and quasibiennial ranges. We also see interannual signals which approach, equal, or exceed 95\% significance.}
\label{conf4.2:1700}
\end{minipage}
\end{figure}

\begin{figure}[!tbp]
\centering
\begin{minipage}[b]{0.45\textwidth}
\includegraphics[width=\textwidth]{./figures/itrdb_Lyear_spline95pct250yrs_lateHolocene_seglen251_1500-2010_evol_lfv.eps}

\caption{Evolutive LFV plot for detrending scheme 4.2 (1500-2010).}
\label{evol4.2:1500}

\end{minipage}
\hfill
\begin{minipage}[b]{0.45\textwidth}
\includegraphics[width=\textwidth]{./figures/itrdb_Lyear_spline95pct250yrs_lateHolocene_seglen251_1700-1850_evol_lfv.eps}
\caption{Evolutive LFV plot for detrending scheme 4.2 (1700-1850).}
\label{evol4.2:1700}
\end{minipage}
\end{figure}


\begin{figure}[!tbp]
\centering
\includegraphics[width=\textwidth,height=0.95\textheight,keepaspectratio]{./figures/itrdb_Lyear_expspline95pct100wl_lateHolocene_seglen251_1500-2010vs1700-1850_compsig.eps}

\noindent\fbox{\parbox{\textwidth}{%
\centering
total average difference: 0.12}}

\caption{LFV comparison for detrending scheme 4.3. Exponential fit with only segments at least 250 years long, reverting to spline fit upon exponential fit failure.}

\label{compsig4.3}
\end{figure}

\begin{figure}[!tbp]
\centering
\begin{minipage}[b]{0.45\textwidth}
\includegraphics[width=\textwidth]{./figures/itrdb_Lyear_expspline95pct100wl_lateHolocene_seglen251_1500-2010_conf.eps}
\caption{LFV plot for detrending scheme 4.3 (1500-2010). Here we see signals in the secular, interdecadal, and interannual ranges which exceed 99\% significance. We also see signals in the quasibiennial range which approach or exceed 95\% significance.}
\label{conf4.3:1500}
\end{minipage}
\hfill
\begin{minipage}[b]{0.45\textwidth}
\includegraphics[width=\textwidth]{./figures/itrdb_Lyear_expspline95pct100wl_lateHolocene_seglen251_1700-1850_conf.eps}
\caption{LFV plot for detrending scheme 4.3 (1700-1850). Here we have signals in both the interdecadal and quasibiennial ranges which exceed 99\% significance. We also see signals in the interannual range which approach or exceed 95\% significance.}
\label{conf4.3:1700}
\end{minipage}
\end{figure}

\begin{figure}[!tbp]
\centering
\begin{minipage}[b]{0.45\textwidth}
\includegraphics[width=\textwidth]{./figures/itrdb_Lyear_expspline95pct100wl_lateHolocene_seglen251_1500-2010_evol_lfv.eps}

\caption{Evolutive LFV plot for detrending scheme 4.3 (1500-2010).}
\label{evol4.3:1500}

\end{minipage}
\hfill
\begin{minipage}[b]{0.45\textwidth}
\includegraphics[width=\textwidth]{./figures/itrdb_Lyear_expspline95pct100wl_lateHolocene_seglen251_1700-1850_evol_lfv.eps}
\caption{Evolutive LFV plot for detrending scheme 4.3 (1700-1850).}
\label{evol4.3:1700}
\end{minipage}
\end{figure}

\begin{figure}[!tbp]
\centering
\includegraphics[width=\textwidth]{./figures/all_lfv.eps}
\caption{LFV plot for all detrending schemes (1500-2010).}
\label{lfv:all:1500}
\end{figure}

\clearpage
\newpage


\begin{figure}[!tbp]
\centering
\begin{minipage}[b]{0.45\textwidth}
\includegraphics[width=\textwidth]{./figures/itrdb_Lyear_spline50pct67wl_lateHolocene_seglen0_1500-2010_ts_avg.eps}
\caption{Time series plot for interannual signal average, detrending scheme 1 (1500-2010).}
\label{ts1avg}
\end{minipage}
\hfill
\begin{minipage}[b]{0.45\textwidth}
\includegraphics[width=\textwidth]{./figures/itrdb_Lyear_spline50pct67wl_lateHolocene_seglen0_1500-2010_phase_avg.eps}
\caption{Phase plot for interannual signal average, detrending scheme 1 (1500-2010).}
\label{map1avg}
\end{minipage}
\end{figure}

\begin{figure}[!tbp]
\centering
\begin{minipage}[b]{0.45\textwidth}
\includegraphics[width=\textwidth]{./figures/itrdb_Lyear_spline95pct30yrs_lateHolocene_seglen0_1500-2010_ts_avg.eps}
\caption{Time series plot for interannual signal average, detrending scheme 2 (1500-2010).}
\label{ts2avg}
\end{minipage}
\hfill
\begin{minipage}[b]{0.45\textwidth}
\includegraphics[width=\textwidth]{./figures/itrdb_Lyear_spline95pct30yrs_lateHolocene_seglen0_1500-2010_phase_avg.eps}
\caption{Phase plot for interannual signal average, detrending scheme 2 (1500-2010).}
\label{map2avg}
\end{minipage}
\end{figure}

\begin{figure}[!tbp]
\centering
\begin{minipage}[b]{0.45\textwidth}
\includegraphics[width=\textwidth]{./figures/itrdb_Lyear_expspline50pct67wl_lateHolocene_seglen0_1500-2010_ts_avg.eps}
\caption{Time series plot for interannual signal average, detrending scheme 3 (1500-2010).}
\label{ts3avg}
\end{minipage}
\hfill
\begin{minipage}[b]{0.45\textwidth}
\includegraphics[width=\textwidth]{./figures/itrdb_Lyear_expspline50pct67wl_lateHolocene_seglen0_1500-2010_phase_avg.eps}
\caption{Phase plot for interannual signal average, detrending scheme 3 (1500-2010).}
\label{map3avg}
\end{minipage}
\end{figure}

\begin{figure}[!tbp]
\centering
\begin{minipage}[b]{0.45\textwidth}
\includegraphics[width=\textwidth]{./figures/itrdb_Lyear_spline95pct100wl_lateHolocene_seglen251_1500-2010_ts_avg.eps}
\caption{Time series plot for interannual signal average, detrending scheme 4.1 (1500-2010).}
\label{ts4.1avg}
\end{minipage}
\hfill
\begin{minipage}[b]{0.45\textwidth}
\includegraphics[width=\textwidth]{./figures/itrdb_Lyear_spline95pct100wl_lateHolocene_seglen251_1500-2010_phase_avg.eps}
\caption{Phase plot for interannual signal average, detrending scheme 4.1 (1500-2010).}
\label{map4.1avg}
\end{minipage}
\end{figure}

\begin{figure}[!tbp]
\centering
\begin{minipage}[b]{0.45\textwidth}
\includegraphics[width=\textwidth]{./figures/itrdb_Lyear_spline95pct250yrs_lateHolocene_seglen251_1500-2010_ts_avg.eps}
\caption{Time series plot for interannual signal average, detrending scheme 4.2 (1500-2010).}
\label{ts4.2avg}
\end{minipage}
\hfill
\begin{minipage}[b]{0.45\textwidth}
\includegraphics[width=\textwidth]{./figures/itrdb_Lyear_spline95pct250yrs_lateHolocene_seglen251_1500-2010_phase_avg.eps}
\caption{Phase plot for interannual signal average, detrending scheme 4.2 (1500-2010).}
\label{map4.2avg}
\end{minipage}
\end{figure}

\begin{figure}[!tbp]
\centering
\begin{minipage}[b]{0.45\textwidth}
\includegraphics[width=\textwidth]{./figures/itrdb_Lyear_expspline95pct100wl_lateHolocene_seglen251_1500-2010_ts_avg.eps}
\caption{Time series plot for interannual signal average, detrending scheme 4.3 (1500-2010).}
\label{ts4.3avg}
\end{minipage}
\hfill
\begin{minipage}[b]{0.45\textwidth}
\includegraphics[width=\textwidth]{./figures/itrdb_Lyear_expspline95pct100wl_lateHolocene_seglen251_1500-2010_phase_avg.eps}
\caption{Phase plot for interannual signal average, detrending scheme 4.3 (1500-2010).}
\label{map4.3avg}
\end{minipage}
\end{figure}

\clearpage
\newpage

\begin{figure}[!tbp]
\centering
\begin{minipage}[b]{0.45\textwidth}
\includegraphics[width=\textwidth]{./figures/itrdb_Lyear_spline50pct67wl_lateHolocene_seglen0_1500-2010_ts_13014.eps}
\caption{Time series plot for interannual signal at $f=0.13$ cyc/yr, detrending scheme 1 (1500-2010).}
\label{ts1p1}
\end{minipage}
\hfill
\begin{minipage}[b]{0.45\textwidth}
\includegraphics[width=\textwidth]{./figures/itrdb_Lyear_spline50pct67wl_lateHolocene_seglen0_1500-2010_phase_13014.eps}
\caption{Phase plot for interannual signal at $f=0.13$ cyc/yr, detrending scheme 1 (1500-2010).}
\label{map1p1}
\end{minipage}
\end{figure}


\begin{figure}[!tbp]
\centering
\begin{minipage}[b]{0.45\textwidth}
\includegraphics[width=\textwidth]{./figures/itrdb_Lyear_spline50pct67wl_lateHolocene_seglen0_1500-2010_ts_15362.eps}
\caption{Time series plot for interannual signal at $f=0.153$ cyc/yr, detrending scheme 1 (1500-2010).}
\label{ts1p2}
\end{minipage}
\hfill
\begin{minipage}[b]{0.45\textwidth}
\includegraphics[width=\textwidth]{./figures/itrdb_Lyear_spline50pct67wl_lateHolocene_seglen0_1500-2010_phase_15362.eps}
\caption{Phase plot for interannual signal at $f=0.153$ cyc/yr, detrending scheme 1 (1500-2010).}
\label{map1p2}
\end{minipage}
\end{figure}

\begin{figure}[!tbp]
\centering
\begin{minipage}[b]{0.45\textwidth}
\includegraphics[width=\textwidth]{./figures/itrdb_Lyear_spline50pct67wl_lateHolocene_seglen0_1500-2010_ts_18933.eps}
\caption{Time series plot for interannual signal at $f=0.189$ cyc/yr, detrending scheme 1 (1500-2010).}
\label{ts1p3}
\end{minipage}
\hfill
\begin{minipage}[b]{0.45\textwidth}
\includegraphics[width=\textwidth]{./figures/itrdb_Lyear_spline50pct67wl_lateHolocene_seglen0_1500-2010_phase_18933.eps}
\caption{Phase plot for interannual signal at $f=0.189$ cyc/yr, detrending scheme 1 (1500-2010).}
\label{map1p3}
\end{minipage}
\end{figure}

\begin{figure}[!tbp]
\centering
\begin{minipage}[b]{0.45\textwidth}
\includegraphics[width=\textwidth]{./figures/itrdb_Lyear_spline50pct67wl_lateHolocene_seglen0_1500-2010_ts_19129.eps}
\caption{Time series plot for interannual signal at $f=0.191$ cyc/yr, detrending scheme 1 (1500-2010).}
\label{ts1p4}
\end{minipage}
\hfill
\begin{minipage}[b]{0.45\textwidth}
\includegraphics[width=\textwidth]{./figures/itrdb_Lyear_spline50pct67wl_lateHolocene_seglen0_1500-2010_phase_19129.eps}
\caption{Phase plot for interannual signal at $f=0.191$ cyc/yr, detrending scheme 1 (1500-2010).}
\label{map1p4}
\end{minipage}
\end{figure}


\clearpage
\newpage

\begin{figure}[!tbp]
\centering
\begin{minipage}[b]{0.45\textwidth}
\includegraphics[width=\textwidth]{./figures/itrdb_Lyear_spline95pct30yrs_lateHolocene_seglen0_1500-2010_ts_13014.eps}
\caption{Time series plot for interannual signal at $f=0.13$ cyc/yr, detrending scheme 2 (1500-2010).}
\label{ts2p1}
\end{minipage}
\hfill
\begin{minipage}[b]{0.45\textwidth}
\includegraphics[width=\textwidth]{./figures/itrdb_Lyear_spline95pct30yrs_lateHolocene_seglen0_1500-2010_phase_13014.eps}
\caption{Phase plot for interannual signal at $f=0.13$ cyc/yr, detrending scheme 2 (1500-2010).}
\label{map2p1}
\end{minipage}
\end{figure}

\begin{figure}[!tbp]
\centering
\begin{minipage}[b]{0.45\textwidth}
\includegraphics[width=\textwidth]{./figures/itrdb_Lyear_spline95pct30yrs_lateHolocene_seglen0_1500-2010_ts_15362.eps}
\caption{Time series plot for interannual signal at $f=0.153$ cyc/yr, detrending scheme 2 (1500-2010).}
\label{ts2p2}
\end{minipage}
\hfill
\begin{minipage}[b]{0.45\textwidth}
\includegraphics[width=\textwidth]{./figures/itrdb_Lyear_spline95pct30yrs_lateHolocene_seglen0_1500-2010_phase_15362.eps}
\caption{Phase plot for interannual signal at $f=0.153$ cyc/yr, detrending scheme 2 (1500-2010).}
\label{map2p2}
\end{minipage}
\end{figure}

\begin{figure}[!tbp]
\centering
\begin{minipage}[b]{0.45\textwidth}
\includegraphics[width=\textwidth]{./figures/itrdb_Lyear_spline95pct30yrs_lateHolocene_seglen0_1500-2010_ts_18933.eps}
\caption{Time series plot for interannual signal at $f=0.189$ cyc/yr, detrending scheme 2 (1500-2010).}
\label{ts2p3}
\end{minipage}
\hfill
\begin{minipage}[b]{0.45\textwidth}
\includegraphics[width=\textwidth]{./figures/itrdb_Lyear_spline95pct30yrs_lateHolocene_seglen0_1500-2010_phase_18933.eps}
\caption{Phase plot for interannual signal at $f=0.189$ cyc/yr, detrending scheme 2 (1500-2010).}
\label{map2p3}
\end{minipage}
\end{figure}

\begin{figure}[!tbp]
\centering
\begin{minipage}[b]{0.45\textwidth}
\includegraphics[width=\textwidth]{./figures/itrdb_Lyear_spline95pct30yrs_lateHolocene_seglen0_1500-2010_ts_19129.eps}
\caption{Time series plot for interannual signal at $f=0.191$ cyc/yr, detrending scheme 2 (1500-2010).}
\label{ts2p4}
\end{minipage}
\hfill
\begin{minipage}[b]{0.45\textwidth}
\includegraphics[width=\textwidth]{./figures/itrdb_Lyear_spline95pct30yrs_lateHolocene_seglen0_1500-2010_phase_19129.eps}
\caption{Phase plot for interannual signal at $f=0.191$ cyc/yr, detrending scheme 2 (1500-2010).}
\label{map2p4}
\end{minipage}
\end{figure}

\clearpage
\newpage

\begin{figure}[!tbp]
\centering
\begin{minipage}[b]{0.45\textwidth}
\includegraphics[width=\textwidth]{./figures/itrdb_Lyear_expspline50pct67wl_lateHolocene_seglen0_1500-2010_ts_12965.eps}
\caption{Time series plot for interannual signal at $f=0.13$ cyc/yr, detrending scheme 3 (1500-2010).}
\label{ts3p1}
\end{minipage}
\hfill
\begin{minipage}[b]{0.45\textwidth}
\includegraphics[width=\textwidth]{./figures/itrdb_Lyear_expspline50pct67wl_lateHolocene_seglen0_1500-2010_phase_12965.eps}
\caption{Phase plot for interannual signal at $f=0.13$ cyc/yr, detrending scheme 3 (1500-2010).}
\label{map3p1}
\end{minipage}
\end{figure}

\begin{figure}[!tbp]
\centering
\begin{minipage}[b]{0.45\textwidth}
\includegraphics[width=\textwidth]{./figures/itrdb_Lyear_expspline50pct67wl_lateHolocene_seglen0_1500-2010_ts_15362.eps}
\caption{Time series plot for interannual signal at $f=0.154$ cyc/yr, detrending scheme 3 (1500-2010).}
\label{ts3p2}
\end{minipage}
\hfill
\begin{minipage}[b]{0.45\textwidth}
\includegraphics[width=\textwidth]{./figures/itrdb_Lyear_expspline50pct67wl_lateHolocene_seglen0_1500-2010_phase_15362.eps}
\caption{Phase plot for interannual signal at $f=0.154$ cyc/yr, detrending scheme 3 (1500-2010).}
\label{map3p2}
\end{minipage}
\end{figure}

\begin{figure}[!tbp]
\centering
\begin{minipage}[b]{0.45\textwidth}
\includegraphics[width=\textwidth]{./figures/itrdb_Lyear_expspline50pct67wl_lateHolocene_seglen0_1500-2010_ts_18933.eps}
\caption{Time series plot for interannual signal at $f=0.189$ cyc/yr, detrending scheme 3 (1500-2010).}
\label{ts3p3}
\end{minipage}
\hfill
\begin{minipage}[b]{0.45\textwidth}
\includegraphics[width=\textwidth]{./figures/itrdb_Lyear_expspline50pct67wl_lateHolocene_seglen0_1500-2010_phase_18933.eps}
\caption{Phase plot for interannual signal at $f=0.189$ cyc/yr, detrending scheme 3 (1500-2010).}
\label{map3p3}
\end{minipage}
\end{figure}

\clearpage
\newpage

\begin{figure}[!tbp]
\centering
\begin{minipage}[b]{0.45\textwidth}
\includegraphics[width=\textwidth]{./figures/itrdb_Lyear_spline95pct100wl_lateHolocene_seglen251_1500-2010_ts_12965.eps}
\caption{Time series plot for interannual signal at $f=0.13$ cyc/yr, detrending scheme 4.1 (1500-2010).}
\label{ts4.1p1}
\end{minipage}
\hfill
\begin{minipage}[b]{0.45\textwidth}
\includegraphics[width=\textwidth]{./figures/itrdb_Lyear_spline95pct100wl_lateHolocene_seglen251_1500-2010_phase_12965.eps}
\caption{Phase plot for interannual signal at $f=0.13$ cyc/yr, detrending scheme 4.1 (1500-2010).}
\label{map4.1p1}
\end{minipage}
\end{figure}

\begin{figure}[!tbp]
\centering
\begin{minipage}[b]{0.45\textwidth}
\includegraphics[width=\textwidth]{./figures/itrdb_Lyear_spline95pct100wl_lateHolocene_seglen251_1500-2010_ts_18933.eps}
\caption{Time series plot for interannual signal at $f=0.189$ cyc/yr, detrending scheme 4.1 (1500-2010).}
\label{ts4.1p2}
\end{minipage}
\hfill
\begin{minipage}[b]{0.45\textwidth}
\includegraphics[width=\textwidth]{./figures/itrdb_Lyear_spline95pct100wl_lateHolocene_seglen251_1500-2010_phase_18933.eps}
\caption{Phase plot for interannual signal at $f=0.189$ cyc/yr, detrending scheme 4.1 (1500-2010).}
\label{map4.1p2}
\end{minipage}
\end{figure}

\begin{figure}[!tbp]
\centering
\begin{minipage}[b]{0.45\textwidth}
\includegraphics[width=\textwidth]{./figures/itrdb_Lyear_spline95pct100wl_lateHolocene_seglen251_1500-2010_ts_21771.eps}
\caption{Time series plot for interannual signal at $f=0.218$ cyc/yr, detrending scheme 4.1 (1500-2010).}
\label{ts4.1p3}
\end{minipage}
\hfill
\begin{minipage}[b]{0.45\textwidth}
\includegraphics[width=\textwidth]{./figures/itrdb_Lyear_spline95pct100wl_lateHolocene_seglen251_1500-2010_phase_21771.eps}
\caption{Phase plot for interannual signal at $f=0.218$ cyc/yr, detrending scheme 4.1 (1500-2010).}
\label{map4.1p3}
\end{minipage}
\end{figure}


\clearpage
\newpage

\begin{figure}[!tbp]
\centering
\begin{minipage}[b]{0.45\textwidth}
\includegraphics[width=\textwidth]{./figures/itrdb_Lyear_spline95pct250yrs_lateHolocene_seglen251_1500-2010_ts_13063.eps}
\caption{Time series plot for interannual signal at $f=0.13$ cyc/yr, detrending scheme 4.2 (1500-2010).}
\label{ts4.2p1}
\end{minipage}
\hfill
\begin{minipage}[b]{0.45\textwidth}
\includegraphics[width=\textwidth]{./figures/itrdb_Lyear_spline95pct250yrs_lateHolocene_seglen251_1500-2010_phase_13063.eps}
\caption{Phase plot for interannual signal at $f=0.13$ cyc/yr, detrending scheme 4.2 (1500-2010).}
\label{map4.2p1}
\end{minipage}
\end{figure}

\begin{figure}[!tbp]
\centering
\begin{minipage}[b]{0.45\textwidth}
\includegraphics[width=\textwidth]{./figures/itrdb_Lyear_spline95pct250yrs_lateHolocene_seglen251_1500-2010_ts_21771.eps}
\caption{Time series plot for interannual signal at $f=0.218$ cyc/yr, detrending scheme 4.2 (1500-2010).}
\label{ts4.2p2}
\end{minipage}
\hfill
\begin{minipage}[b]{0.45\textwidth}
\includegraphics[width=\textwidth]{./figures/itrdb_Lyear_spline95pct250yrs_lateHolocene_seglen251_1500-2010_phase_21771.eps}
\caption{Phase plot for interannual signal at $f=0.218$ cyc/yr, detrending scheme 4.2 (1500-2010).}
\label{map4.2p2}
\end{minipage}
\end{figure}

\begin{figure}[!tbp]
\centering
\begin{minipage}[b]{0.45\textwidth}
\includegraphics[width=\textwidth]{./figures/itrdb_Lyear_spline95pct250yrs_lateHolocene_seglen251_1500-2010_ts_31654.eps}
\caption{Time series plot for interannual signal at $f=0.317$ cyc/yr, detrending scheme 4.2 (1500-2010).}
\label{ts4.2p3}
\end{minipage}
\hfill
\begin{minipage}[b]{0.45\textwidth}
\includegraphics[width=\textwidth]{./figures/itrdb_Lyear_spline95pct250yrs_lateHolocene_seglen251_1500-2010_phase_31654.eps}
\caption{Phase plot for interannual signal at $f=0.317$ cyc/yr, detrending scheme 4.2 (1500-2010).}
\label{map4.2p3}
\end{minipage}
\end{figure}

\begin{figure}[!tbp]
\centering
\begin{minipage}[b]{0.45\textwidth}
\includegraphics[width=\textwidth]{./figures/itrdb_Lyear_spline95pct250yrs_lateHolocene_seglen251_1500-2010_ts_318.eps}
\caption{Time series plot for interannual signal at $f=0.318$ cyc/yr, detrending scheme 4.2 (1500-2010).}
\label{ts4.2p4}
\end{minipage}
\hfill
\begin{minipage}[b]{0.45\textwidth}
\includegraphics[width=\textwidth]{./figures/itrdb_Lyear_spline95pct250yrs_lateHolocene_seglen251_1500-2010_phase_318.eps}
\caption{Phase plot for interannual signal at $f=0.318$ cyc/yr, detrending scheme 4.2 (1500-2010).}
\label{map4.2p4}
\end{minipage}
\end{figure}


\clearpage
\newpage

\begin{figure}[!tbp]
\centering
\begin{minipage}[b]{0.45\textwidth}
\includegraphics[width=\textwidth]{./figures/itrdb_Lyear_expspline95pct100wl_lateHolocene_seglen251_1500-2010_ts_12965.eps}
\caption{Time series plot for interannual signal at $f=0.13$ cyc/yr, detrending scheme 4.3 (1500-2010).}
\label{ts4.3p1}
\end{minipage}
\hfill
\begin{minipage}[b]{0.45\textwidth}
\includegraphics[width=\textwidth]{./figures/itrdb_Lyear_expspline95pct100wl_lateHolocene_seglen251_1500-2010_phase_12965.eps}
\caption{Phase plot for interannual signal at $f=0.13$ cyc/yr, detrending scheme 4.3 (1500-2010).}
\label{map4.3p1}
\end{minipage}
\end{figure}

\begin{figure}[!tbp]
\centering
\begin{minipage}[b]{0.45\textwidth}
\includegraphics[width=\textwidth]{./figures/itrdb_Lyear_expspline95pct100wl_lateHolocene_seglen251_1500-2010_ts_18933.eps}
\caption{Time series plot for interannual signal at $f=0.189$ cyc/yr, detrending scheme 4.3 (1500-2010).}
\label{ts4.3p2}
\end{minipage}
\hfill
\begin{minipage}[b]{0.45\textwidth}
\includegraphics[width=\textwidth]{./figures/itrdb_Lyear_expspline95pct100wl_lateHolocene_seglen251_1500-2010_phase_18933.eps}
\caption{Phase plot for interannual signal at $f=0.189$ cyc/yr, detrending scheme 4.3 (1500-2010).}
\label{map4.3p2}
\end{minipage}
\end{figure}

\begin{figure}[!tbp]
\centering
\begin{minipage}[b]{0.45\textwidth}
\includegraphics[width=\textwidth]{./figures/itrdb_Lyear_expspline95pct100wl_lateHolocene_seglen251_1500-2010_ts_19129.eps}
\caption{Time series plot for interannual signal at $f=0.191$ cyc/yr, detrending scheme 4.3 (1500-2010).}
\label{ts4.3p3}
\end{minipage}
\hfill
\begin{minipage}[b]{0.45\textwidth}
\includegraphics[width=\textwidth]{./figures/itrdb_Lyear_expspline95pct100wl_lateHolocene_seglen251_1500-2010_phase_19129.eps}
\caption{Phase plot for interannual signal at $f=0.191$ cyc/yr, detrending scheme 4.3 (1500-2010).}
\label{map4.3p3}
\end{minipage}
\end{figure}

\begin{figure}[!tbp]
\centering
\begin{minipage}[b]{0.45\textwidth}
\includegraphics[width=\textwidth]{./figures/itrdb_Lyear_expspline95pct100wl_lateHolocene_seglen251_1500-2010_ts_21771.eps}
\caption{Time series plot for interannual signal at $f=0.218$ cyc/yr, detrending scheme 4.3 (1500-2010).}
\label{ts4.3p4}
\end{minipage}
\hfill
\begin{minipage}[b]{0.45\textwidth}
\includegraphics[width=\textwidth]{./figures/itrdb_Lyear_expspline95pct100wl_lateHolocene_seglen251_1500-2010_phase_21771.eps}
\caption{Phase plot for interannual signal at $f=0.218$ cyc/yr, detrending scheme 4.3 (1500-2010).}
\label{map4.3p4}
\end{minipage}
\end{figure}

\begin{figure}[!tbp]
\centering
\begin{minipage}[b]{0.45\textwidth}
\includegraphics[width=\textwidth]{./figures/itrdb_Lyear_expspline95pct100wl_lateHolocene_seglen251_1500-2010_ts_3772.eps}
\caption{Time series plot for interannual signal at $f=0.377$ cyc/yr, detrending scheme 4.3 (1500-2010).}
\label{ts4.3p5}
\end{minipage}
\hfill
\begin{minipage}[b]{0.45\textwidth}
\includegraphics[width=\textwidth]{./figures/itrdb_Lyear_expspline95pct100wl_lateHolocene_seglen251_1500-2010_phase_3772.eps}
\caption{Phase plot for interannual signal at $f=0.377$ cyc/yr, detrending scheme 4.3 (1500-2010).}
\label{map4.3p5}
\end{minipage}
\end{figure}

\begin{figure}[!tbp]
\centering
\begin{minipage}[b]{0.45\textwidth}
\includegraphics[width=\textwidth]{./figures/itrdb_Lyear_expspline95pct100wl_lateHolocene_seglen251_1500-2010_ts_38014.eps}
\caption{Time series plot for interannual signal at $f=0.38$ cyc/yr, detrending scheme 4.3 (1500-2010).}
\label{ts4.3p6}
\end{minipage}
\hfill
\begin{minipage}[b]{0.45\textwidth}
\includegraphics[width=\textwidth]{./figures/itrdb_Lyear_expspline95pct100wl_lateHolocene_seglen251_1500-2010_phase_38014.eps}
\caption{Phase plot for interannual signal at $f=0.38$ cyc/yr, detrending scheme 4.3 (1500-2010).}
\label{map4.3p6}
\end{minipage}
\end{figure}

\clearpage
\newpage

\begin{figure}[!tbp]
\centering
\begin{minipage}[b]{0.45\textwidth}
\includegraphics[width=\textwidth]{./figures/itrdb_Lyear_spline95pct30yrs_lateHolocene_seglen0_1500-2010_ts_011252.eps}
\caption{Time series plot for interdecadal signal at $f=0.0113$ cyc/yr, detrending scheme 2 (1500-2010).}
\label{ts2pl1}
\end{minipage}
\hfill
\begin{minipage}[b]{0.45\textwidth}
\includegraphics[width=\textwidth]{./figures/itrdb_Lyear_spline95pct30yrs_lateHolocene_seglen0_1500-2010_phase_011252.eps}
\caption{Phase plot for interdecadal signal at $f=0.0113$ cyc/yr, detrending scheme 2 (1500-2010).}
\label{map2pl1}
\end{minipage}
\end{figure}

\begin{figure}[!tbp]
\centering
\begin{minipage}[b]{0.45\textwidth}
\includegraphics[width=\textwidth]{./figures/itrdb_Lyear_spline95pct100wl_lateHolocene_seglen251_1500-2010_ts_043542.eps}
\caption{Time series plot for interdecadal signal at $f=0.0435$ cyc/yr, detrending scheme 4.1 (1500-2010).}
\label{ts4.1pl1}
\end{minipage}
\hfill
\begin{minipage}[b]{0.45\textwidth}
\includegraphics[width=\textwidth]{./figures/itrdb_Lyear_spline95pct100wl_lateHolocene_seglen251_1500-2010_phase_043542.eps}
\caption{Phase plot for interdecadal signal at $f=0.0435$ cyc/yr, detrending scheme 4.1 (1500-2010).}
\label{map4.1pl1}
\end{minipage}
\end{figure}

\begin{figure}[!tbp]
\centering
\begin{minipage}[b]{0.45\textwidth}
\includegraphics[width=\textwidth]{./figures/itrdb_Lyear_spline95pct100wl_lateHolocene_seglen251_1500-2010_ts_074364.eps}
\caption{Time series plot for interdecadal signal at $f=0.0744$ cyc/yr, detrending scheme 4.1 (1500-2010).}
\label{ts4.1pl2}
\end{minipage}
\hfill
\begin{minipage}[b]{0.45\textwidth}
\includegraphics[width=\textwidth]{./figures/itrdb_Lyear_spline95pct100wl_lateHolocene_seglen251_1500-2010_phase_074364.eps}
\caption{Phase plot for interdecadal signal at $f=0.0744$ cyc/yr, detrending scheme 4.1 (1500-2010).}
\label{map4.1pl2}
\end{minipage}
\end{figure}

\begin{figure}[!tbp]
\centering
\begin{minipage}[b]{0.45\textwidth}
\includegraphics[width=\textwidth]{./figures/itrdb_Lyear_spline95pct250yrs_lateHolocene_seglen251_1500-2010_ts_0019569.eps}
\caption{Time series plot for secular signal at $f=0.002$ cyc/yr, detrending scheme 4.2 (1500-2010).}
\label{ts4.2pl1}
\end{minipage}
\hfill
\begin{minipage}[b]{0.45\textwidth}
\includegraphics[width=\textwidth]{./figures/itrdb_Lyear_spline95pct250yrs_lateHolocene_seglen251_1500-2010_phase_0019569.eps}
\caption{Phase plot for secular signal at $f=0.002$ cyc/yr, detrending scheme 4.2 (1500-2010).}
\label{map4.2pl1}
\end{minipage}
\end{figure}

\begin{figure}[!tbp]
\centering
\begin{minipage}[b]{0.45\textwidth}
\includegraphics[width=\textwidth]{./figures/itrdb_Lyear_spline95pct250yrs_lateHolocene_seglen251_1500-2010_ts_0318.eps}
\caption{Time series plot for interdecadal signal at $f=0.0318$ cyc/yr, detrending scheme 4.2 (1500-2010).}
\label{ts4.2pl2}
\end{minipage}
\hfill
\begin{minipage}[b]{0.45\textwidth}
\includegraphics[width=\textwidth]{./figures/itrdb_Lyear_spline95pct250yrs_lateHolocene_seglen251_1500-2010_phase_0318.eps}
\caption{Phase plot for interdecadal signal at $f=0.0318$ cyc/yr, detrending scheme 4.2 (1500-2010).}
\label{map4.2pl2}
\end{minipage}
\end{figure}

\begin{figure}[!tbp]
\centering
\begin{minipage}[b]{0.45\textwidth}
\includegraphics[width=\textwidth]{./figures/itrdb_Lyear_expspline95pct100wl_lateHolocene_seglen251_1500-2010_ts_0019569.eps}
\caption{Time series plot for secular signal at $f=0.002$ cyc/yr, detrending scheme 4.3 (1500-2010).}
\label{ts4.3pl1}
\end{minipage}
\hfill
\begin{minipage}[b]{0.45\textwidth}
\includegraphics[width=\textwidth]{./figures/itrdb_Lyear_expspline95pct100wl_lateHolocene_seglen251_1500-2010_phase_0019569.eps}
\caption{Phase plot for secular signal at $f=0.002$ cyc/yr, detrending scheme 4.3 (1500-2010).}
\label{map4.3pl1}
\end{minipage}
\end{figure}

\begin{figure}[!tbp]
\centering
\begin{minipage}[b]{0.45\textwidth}
\includegraphics[width=\textwidth]{./figures/itrdb_Lyear_expspline95pct100wl_lateHolocene_seglen251_1500-2010_ts_043542.eps}
\caption{Time series plot for interdecadal signal at $f=0.0435$ cyc/yr, detrending scheme 4.3 (1500-2010).}
\label{ts4.3pl2}
\end{minipage}
\hfill
\begin{minipage}[b]{0.45\textwidth}
\includegraphics[width=\textwidth]{./figures/itrdb_Lyear_expspline95pct100wl_lateHolocene_seglen251_1500-2010_phase_043542.eps}
\caption{Phase plot for interdecadal signal at $f=0.0435$ cyc/yr, detrending scheme 4.3 (1500-2010).}
\label{map4.3pl2}
\end{minipage}
\end{figure}

\clearpage
\newpage

\begin{figure}[!tbp]
\centering
\includegraphics[width=\textwidth,height=0.95\textheight,keepaspectratio]{./figures/nino_comp_1.eps}
\caption{Reconstructed average interannual time signal for scheme 1 vs NINO3.4}

\label{nino1}
\end{figure}

\begin{figure}[!tbp]
\centering
\includegraphics[width=\textwidth,height=0.95\textheight,keepaspectratio]{./figures/nino_comp_2.eps}
\caption{Reconstructed average interannual time signal for scheme 2 vs NINO3.4}

\label{nino2}
\end{figure}

\begin{figure}[!tbp]
\centering
\includegraphics[width=\textwidth,height=0.95\textheight,keepaspectratio]{./figures/nino_comp_3.eps}
\caption{Reconstructed average interannual time signal for scheme 3 vs NINO3.4}

\label{nino3}
\end{figure}

\begin{figure}[!tbp]
\centering
\includegraphics[width=\textwidth,height=0.95\textheight,keepaspectratio]{./figures/nino_comp_4-1.eps}
\caption{Reconstructed average interannual time signal for scheme 4.1 vs NINO3.4}

\label{nino4.1}
\end{figure}

\begin{figure}[!tbp]
\centering
\includegraphics[width=\textwidth,height=0.95\textheight,keepaspectratio]{./figures/nino_comp_4-2.eps}
\caption{Reconstructed average interannual time signal for scheme 4.2 vs NINO3.4}

\label{nino4.2}
\end{figure}

\begin{figure}[!tbp]
\centering
\includegraphics[width=\textwidth,height=0.95\textheight,keepaspectratio]{./figures/nino_comp_4-3.eps}
\caption{Reconstructed average interannual time signal for scheme 4.3 vs NINO3.4}

\label{nino4.3}
\end{figure}

\end{document}
