% This is a general template file for the LaTeX package SVJour3
% for Springer journals. Original by Springer Heidelberg, 2010/09/16
%
% Use it as the basis for your article. Delete % signs as needed.
%
% This template includes a few options for different layouts and
% content for various journals. Please consult a previous issue of
% your journal as needed.
%
\RequirePackage{fix-cm}
%
%\documentclass{svjour3}                     % onecolumn (standard format)
%\documentclass[smallcondensed]{svjour3}     % onecolumn (ditto)
\documentclass[smallextended]{svjour3}       % onecolumn (second format)
%\documentclass[a4paper]{article}
%\documentclass[twocolumn]{svjour3}          % twocolumn
%
\smartqed  % flush right qed marks, e.g. at end of proof
%
\usepackage{graphicx}
\usepackage{booktabs}
\usepackage{afterpage}
\setcounter{tocdepth}{4}
\setcounter{secnumdepth}{4}
\newcommand{\myparagraph}[1]{\paragraph{#1}\mbox{}\\\mbox{}\\}
\usepackage{setspace}
\doublespacing
%
% insert here the call for the packages your document requires
%\usepackage{mathptmx}      % use Times fonts if available on your TeX system
%\usepackage{latexsym}
% etc.
%
% please place your own definitions here and don't use \def but
% \newcommand{}{}
%
% Insert the name of "your journal" with
\journalname{Climate Dynamics}
%
\begin{document}

\title{Minor impacts of major volcanic eruptions in
  dynamically-downscaled last millennium ensemble data} %COMMENT: OK?

\thanks{NSF grants AGS1602564 and 1751535 as well as an AWS computing award.} %COMMENT: get correct verbiage

% Grants or other notes about the article that should go on the front
% page should be placed within the \thanks{} command in the title
% (and the %-sign in front of \thanks{} should be deleted)
%
% General acknowledgments should be placed at the end of the article.



\titlerunning{Volcanic effects on hurricanes}

\author{Benton, Brandon N.$^1$ \and Alessi, Marc J.$^1$ \and Herrera, Dimitris$^1$ \and Li, Xiaolu$^1$ \and Carrillo M., Carlos$^1$ \and Ault, Toby R.$^1$}

%\authorrunning{Short form of author list} % if too long for running head

\institute{Benton, Brandon N. \at
              %Cornell University \\
	      %Tel.: +123-45-678910\\
	      %Fax: +123-45-678910\\
	      \email{bnb32@cornell.edu}           %  \\
%             \emph{Present address:} of F. Author  %  if needed
              \and
	   Ault, Toby R. \at
	   %Cornell University \\
	   \email{tra38@cornell.edu}
}

\date{$^1$Cornell University \\
Received: date / Accepted: date}
% The correct dates will be entered by the editor

\maketitle

\begin{abstract}
  Thermodynamic and dynamic effects of volcanic eruptions on hurricane
  statistics are examined using long two simulations from the
  Community Earth System Model (CESM) Last Millennium Ensemble
  (LME). The first is an unforced control simulation, wherein all
  boundary conditions were held constant at their 850 CE values. The
  second is a ``fully forced'' simulation with time evolving radiative
  changes from solar, volcanic, solar, and land use changes from 850
  through present. The largest magnitude radiative forcings during
  this time period are the large tropical volcanic eruptions, which
  comprise the focus of this study. Potential and simulated hurricane
  statistics are computed from both the control and forced
  simulations. Potential Intensity is evaluated using model output at
  its native (nominally 2 degree lat/lon) spatial resolution, while
  the weather research and forecasting (WRF) model is used for
  dynamically downscaling a total of 100 control years and an
  additional 100 years following the largest volcanic eruptions in the
  fully forced simulation.  Limitations of the downscaling methodology
  are examined by applying the same approach to historical ERAI
  reanalysis data and comparing the downscaled storm tracks and
  intensities to the IBTrACS database. Results suggest small effects
  are observed in averages over all last millennium eruptions which
  are non-significant in comparison to the control. However, for many
  of the major eruptions, significant reductions are seen in hurricane
  frequency, intensity, and lifetime. Strong evidence is also shown
  for correlation between eruption strength and changes in these
  diagnostics.
%\keywords{Hurricanes \and downscaling \and climatology}
% \PACS{PACS code1 \and PACS code2 \and more}
% \subclass{MSC code1 \and MSC code2 \and more}
\end{abstract}
\keywords{Volcanos \and Hurricanes \and Last Millennium Ensemble \and
  Paleoclimate \and Climate modeling}

\section{Introduction}
\label{intro}
\par
\textbf{Hurricanes threaten human lives and likelihoods, inflict severe damage
to property, and incur billions of dollars in economic losses and
recovery efforts.} These events alone caused $42\%$ of the
catastrophe-insured losses in the United States in the period
$1992-2011$ \cite{hodges2017well}. For example, in 2005 Hurricane Katrina
resulted in $1833$ casualties and a financial loss of over $\$125$
billion \cite{hodges2017well}. In 2012, Hurricane Sandy killed at least
233 people and caused $\$70$ billion in damage
\cite{blake2013tropical}. Moreover, in 2017 Hurricane Harvey displaced more
than $30,000$ people and resulted in $103$ deaths
\cite{murphy2018service}. The same year Hurricane Maria resulted in nearly
$3,000$ deaths \cite{maria_rpt}. Combined, Harvey and Maria caused
more than $\$200$ billion in damage \cite{murphy2018service,maria_rpt}. The disruption from these extreme weather events
will likely increase with rising coastal populations and increasing
value of infrastructure in coastal areas
\cite{kerry_tc_clim}. Furthermore, anthropogenic climate change is
expected to increase average sea surface temperatures (SSTs)
\cite{ipcc_2007} and sea level. Accordingly, there is growing interest
in determining if modifications to the incoming flux of solar
radiation could potentially offset key impacts expected to occur from
rising global temperature \cite{msadek}. %%irving2019hurricane 
Whether or not
such strategies are pursued, it is critical to understand the
relationship between hurricane statistics and climate responses to
past radiative forcings to help characterize the full
range of plausible future influences on hurricane activity in the
future.
\par


\textbf{The underlying relationship between hurricanes, radiative
  forcing, and climate change remains an area of considerable debate
  and active inquiry \cite{ting2015,elsner2006,msadek}}. Yet, a number of modeling
studies have suggested that, in general, future storms may pose even
more severe threats to human well-being, infrastructure, and the
economy \cite{IPCC2014c}. For example, \cite{villarini2013} and \cite{emanuel12219} used
data from the Climate Model Intercomparison 5 (CMIP5) ensemble to
evaluate storm intensity during climate change if their tracks were
similar to those that unfolded over the 21st Century. Such studies
suggest that, in general, the number and intensity of the largest
storms (e.g., category 4 and 5 hurricanes) will increase in a warmer
climate due, primarily, to increases in sea surface temperature
(SSTs). 

\textbf{If global greenhouse gas reduction efforts fail or are
  insufficient in the coming decades, some researchers argue that
  ``solar radiation management'' (SRM) interventions to the climate system
  may be preferable to allowing global temperatures to increase
  indefinitely \cite{govindasamy2000,caldeira2008,kravitz2014multi,
  macmartin2019}.} 
  Regardless of the
details of how SRM interventions are modeled, their ultimate effect is
to decrease the total amount of sunlight reaching the surface, which
is directly analogous to the effect of stratospheric aerosols from
volcanic eruptions. Therefore volcanic eruptions of the recent past
and last millennium may give us a glimpse of the risks associated with
SRM due to changes in large-scale changes in circulation \cite{rasch2008}.

\textbf{Volcanic eruptions reduce incoming solar surface radiation,
  which can strongly impact global temperatures, circulation patterns,
  and water cycles} For example, asymmetric
volcanic forcings (e.g., from volcanic aerosols being concentrated in
the stratosphere of one hemisphere) would alter the position of the
Inter Tropical Convergence Zone for at least one year following the
eruption. These effects could potentially be even longer lasting if
coupled interactions between the ocean and atmosphere are engaged 
\cite{colose2016hemispherically,raible2016tambora,stevenson2016nino, 
schurer2014small,schurer2013separating}. 
Given that hurricanes are sensitive to the
regions where moisture convergence occurs, it follows that such
radiative effects on the global circulation would influence hurricane
statistics.


\textbf{Large eruptions were suggested to be the dominant forcing for
  the pre-industrial period by comparisons between all-forcing and
  single-forcing last millennium model simulation and multiproxy
  constructions, though greenhouse gases also had detectable
  contributions \cite{schurer2013separating, schurer2014small}}. 
  Studies have also shown that large tropical volcanic eruptions 
  may have long-lasting
influences on the Atlantic multi-decadal oscillation and lead to El
Ni\~no–-like warming in the cold season after the eruption \cite{gcm_lme,
stevenson2016nino} such impacts could, in turn, affect hurricane 
statistics because these large-scale modes help
govern, in part, the frequency and intensity of storms occuring in any
given year.


\textbf{The historical period only provides a few clues about the
  relationship between volcanic forcing and hurricanes, and much less
  about the effect of volcanic eruptions on hurricanes or TCs.}
Nevertheless, modeling studies suggest that a reduction in TC
accumulated energy, TC duration, and lifetime maximum intensity occurs
following a volcanic eruption due to a decrease in SST and increase in
upper tropospheric/lower stratospheric temperature \cite{volc_hurrs2},
all of which decreases TC efficiency \cite{trop_cool}.  There is
evidence that after some eruptions, an asymmetric increase in
stratospheric aerosols occurs in the hemisphere in which the eruption
took place, modifying the sea surface temperature gradient
\cite{asym_forcing}.  This gradient shifts the location of the
Inter-tropical Convergence Zone (ITCZ) to the opposite hemisphere of
the eruption \cite{asym_forcing}, which hinders hurricane development
in the volcano’s own hemisphere due to a decrease in convection and
increase in wind shear.  In fact, after the northern hemisphere
eruptions of Mount Pinatubo (1991) and El Chichon (1982), North
Atlantic TC activity decreased, while TC activity increased following
the southern hemisphere eruption of Agung (1964)
\cite{volc_hurrs3,volc_hurrs2}.

\textbf{Attempts to use paleoclimate indicators of past events (i.e.,
  paleotempestology) are limited by the paucity of appropriate
  archives and proxies \cite{liu20081200,mann2009atlantic,donnelly2015}}. Importantly, most paleorecords are commonly
constrained to certain geographic areas and currently provide limited
information from other regions also commonly affected by TCs, such as
the Caribbean \cite{oliva2018paleotempestology}. The use of paleotempestological
records is further limited because it is not possible to fully
reconstruct the tracks and lifetime of past TCs, most of which occur
over the ocean \cite{emanuel2005increasing}. These limitations preclude the
extensive use of such data to evaluate the effects of major volcanic
eruptions on TC activity \cite{oliva2018paleotempestology, 
yan2015tropical,korty2012variations}.

\textbf{Hurricanes are mesoscale features of the tropical circulation,
  and as such they depend critically on quantities that are typically
  unresolvable in the coarse resolution grid of the CMIP5 generation
  of models, which typically have nominal horizontal resolutions on
  the order of 50-200km.} Overcoming this limitation requires one of
three approaches. The first approach is relatively simple, and entails
calculating thermodynamic metrics like \textit{potential intensity}
(PI) at the native (coarse) resolution of GCM output
\cite{wang,ke_nolan,tang,bister2002}. Such indices can then be used to infer what
would have happened if hurricanes had been resolved in a given
simulation. A recent study \cite{yan2018divergent} 
did just that using the
last millennium ensemble (LME) to determine the theoretical effects of
volcanic eruptions during the last 1000 years on hurricane/TC
potential intensity. The authors found a significant relationship
lasting up to 3 years post-eruption, but also ``divergent'' responses
at the mid and high latitudes to the volcanic forcing. While this
approach is computationally efficient, it doesn't explicitly attempt
to \textit{simulate} Hurricanes/TCs.

A second approach has been to use a statistical method to downscale
model output \cite{down_method_ke,cam_down_ke}. Although this approach
is computationally lightweight, allowing it to be used to investigate
long term variability in a fully coupled GCM simulation of the last
millennium \cite{lme_down_ke}, it does not directly simulate the
sub-grid scale features of Hurricanes and TCs.

The third approach employs a regional model to \textit{dynamically}
downscale GCM output \cite{down_21st_gv}. Dynamical
downscaling typically requires high performance computing
infrastructure as well as boundary conditions from the ``parent'' GCM
at six hourly temporal resolution. It is therefore much more
computationally expensive than the other two methods, but it provides
greater insight into the storms that would have occurred in a given
GCM framework if it were run with sufficiently high spatial
resolution. Dynamical downscaling has been widely used to evaluate
hurricane statistics during the 20th and 21st century \cite{Emanuel12219}, yet
it has not been widely adapted to the last millennium paleoclimate
modeling context.

\section{Data \& Methods}
\label{methods}
\textbf{We dynamically downscale two members of the ``Last Millennium
  Ensemble'' \cite{gcm_lme}.} The LME consists of over two dozen
fully forced, and single forcing, experiments from the period spanning
850 CE to 2005. While monthly data was archived for most of the
members of the LME, two simulations were run with sufficiently high
temporal output to allow for high resolution dynamical downscaling
using a regional model. One of these runs was a fully forced last
millennium simulation and the other was a long control simulation with
time invariant boundary conditions. We further evaluate the strengths
and limitations of our methodology by comparing downscaled reanalysis
data to an historical database of hurricane tracks and
intensities. The details of this approach are described below.

\subsection{Data}
\subsubsection{Last Millennium Ensemble}
Output from only two members of the LME were archived at sufficiently
high temporal resolution to be used as boundary conditions for WRF: a
fully-forced simulation with time varying boundary conditions
($LME_{forced}$) and a pre-industrial control simulation with time
invariant boundary conditions ($LME_{control}$). Both of these
simulations were run from 850 to 2005 CE using the Community Earth
System Model (CESM) version 1.1, with the Community Atmosphere Model
(CAM) version 5. The resolution of the atmosphere and land grids are
nominally ${\sim}2^\circ$, and ${\sim}1^\circ$ for ocean and sea ice
grids. While both runs were spun up for 200 years under control
conditions prior to 850 CE, $LME_{forced}$ was forced with the
transient evolution of solar intensity, volcanic emissions, greenhouse
gases, aerosols, land-use conditions, as well as insolation changes
from planetary orbit and tilt. In the $LME_{control}$ the boundary
conditions were simply held fixed at their pre-industrial levels, thus
providing an unforced baseline for evaluating changes
in hurricane statistics following large volcanic eruptions.

\subsubsection{ERAI \& IBTRACS}
The native $2^o$ resolution of CAM5 in the LME simulations would make
it impossible to resolve hurricanes, hence we cannot evaluate the
reliability of our downscaling methodology (see section \ref{WRF}) with LME
data alone. We therefore also downscaled the ERA-Interim (ERA-I)
\cite{erai_reanal} reanalysis data to retrospectively predict historical
hurricanes, then compared those predictions against the IBTrACS
\cite{ibtracs} database.

ERA-I comprises a reanalysis dataset starting in 1979 and available
until August 2019. It uses four-dimensional variational data
assimilation (4DVAR), yielding a significant advantage over reanalysis
products using 3DVAR. This improves asynoptic data handling and allows
for the influence of an observation to be more strongly controlled by
model dynamics \cite{tc_reanal:2}. This data assimilation method is
coupled with the ECMWF Integrated Forecast Model (IFS) to extrapolate
fields between observations. A detailed description of IBTrACS is
provided in \cite{ibtracs} and ERAI is comprehensively discussed in
\cite{erai_reanal}. In our work, $6$-hourly ERAI data was downscaled
in WRF and compared to IBTrACS for the period $1995-2005$. The
comparison was made using the suite of diagnostics described in
\ref{diags}.

\subsection{Methods}
\subsubsection{Dynamical Downscaling}
\label{WRF}
We used WRF version 3.9 (WRFV3.9) \cite{wrf_tech} to dynamically
downscale archived data from LME simulations with the physics schemes
shown in table \ref{wrf_specs}: (1) WRF single-moment 6-class for
micro-physics \cite{mp_phys}, (2) Yonsei University for PBL
\cite{pbl_phys}, (3) Kain-Fritsch for convection \cite{cu_phys}, (4,5)
rapid radiative transfer model with greenhouse gases for long-wave and
short-wave radiation \cite{rad_phys}, (6) Noah for land surface
\cite{sfc_phys}, (7) fifth generation mesoscale model for surface
layer \cite{sfclay_phys:1,sfclay_phys:2,sfclay_phys:3}, and
(8) simple mixed-layer for ocean \cite{ocn_phys}. We also turned on
heat and moisture surfaces fluxes (isfflx=1) and modification of
exchange coefficients $C_d$ and $C_k$ according to surface winds
(isftcflx=1).

WRF was run for a total of 250 simulation years over a domain spanning
the North American sector from $130W$ to $15E$, allowing us to
identify and track storms from both the Atlantic and Eastern Pacific,
even after making landfall in North America. We elected to use a
horizontal grid spacing ($\Delta X$) of $30km$. The $30km$ spacing
represents a compromise between our competing requirements for high
resolution output and a large sample size; each downscaled year used
approximately 2,000 core hours on the Cheyenne supercomputer totaling
about one million core hours for all years. While the $30km$
resolution is somewhat coarse for resolving certain features of
hurricanes, our results did not seem particularly sensitive to this
choice in comparison to runs with $\Delta X$ set to only $10km$. However, the
three fold increase in spatial resolution would have translated into
more than a 10 fold increase in core hours, or a 10 fold reduction in
the number of years simulated. All data from the LME were prepared for WRF using procedure and code described in \cite{tech_notes}.

\subsubsection{Tracking Tropical Cyclones with TSTORMS}
\label{tstorms}
We applied the TSTORMS \cite{tc_algo} tracking software, developed and
supported by GFDL, to analyze the results of downscaling. This routine
uses minimum pressure and maximum vorticity criteria to identify
cyclones. Events are stored as ``storms'' if they satisfy the
following conditions for a preset number of days ($n_{days}$): (1)
That the maximum vorticity location is within a threshold radius
($r_{crit}$) of the minimum pressure location, (2) that the core
temperature of the cyclone is higher than outside of the core by a
threshold difference ($twc_{crit}$) and (3) the difference in vertical
distance between pressure levels at $200hPa$ and $1000hPa$ outside and
inside the core exceeds a threshold value ($thick_{crit}$).

As described in \cite{kerry_clivar} and \cite{tc_algo}, tracking
results are sensitive to the details of the tracking scheme that is
employed, and especially the threshold values selected for identifying
storms \cite{tc_track}. To identify sensitivity to threshold values,
we conducted a limited parameter sweep to determine optimal threshold
values. We calculated the difference between ERAI downscaled output
and IBTrACS data, for each set of parameters, using the diagnostics
described in section \ref{diags}. We used the set of parameters that
achieved the minimum difference of ${\sim}13.5\%$. This parameter set
was $r_{crit} = 1.5^{\circ}$, $twc_{crit} = 1.0^{\circ}C$,
$thick_{crit} = 50m$, and $n_{days} = 2$.

\subsubsection{Diagnostics}
\label{diags}
Once hurricanes were identified in our downscaled LME data using
TSTORMS, we calculated 15 diagnostic metrics to evaluate differences
in the statistics of storms in $LME_{control}$ and those occurring
after large eruptions in $LME_{forced}$.  These diagnostics consist of
storm number vs (1) month, (2) year, (3) latitude, (4) longitude, (5)
maximum wind speed, (6) minimum pressure, (7) decay time from maximum
wind speed, and (8) decay time from minimum pressure. Additionally, we
calculated percentage of storms within (9) May to November, (10)
$0-25N$ latitude, (11) $100W-50W$ longitude, (12) $1020hPa-980hPa$
pressure, (13) $0m/s-40m/s$ maximum wind speed, (14) $0-100hrs$ decay
time from maximum wind speed, and (15) $0-100hrs$ decay time from
minimum pressure. Mean values and quantile values (expressed as
percentages) were used to calculate fractional differences between
$LME_{control}$ and $LME_{forced}$, and these differences were
averaged over all diagnostics for a composite percentage
difference. We refer to the mean difference of diagnostics 1-8 as the
total ``average difference'' and the mean difference of diagnostics
9-15 as the total ``percentage difference.'' We also computed these
metrics from our downscaled ERA-I data to compare them to the IBTrACS
database as a test of our methodology, as described in sections
\ref{WRF} and \ref{tstorms}.

The diagnostics described above were used as test statistics to
evaluate whether volcanic eruptions have a measurable effect on
hurricane behavior. These diagnostics were selected in order to assess
hurricane behavior across a broad range of characteristics. The
diagnostics not only quantify hurricane behavior across the temporal
and spatial domain, but also assess more fundamental physical
characteristics. In addition, the diagnostics can be used with limited
data consisting only of time, location, wind speed, and surface
pressure. This presents a versatile and efficient approach to capture
both mean climatology and more fine structured hurricane behavior.

To determine whether volcanic eruptions effect hurricane statistics,
we performed two-sample KS-tests for distributions of each of the
diagnostics. The two samples tested for each diagnostic came from
downscaled $LME_{control}$ and $LME_{forced}$ data. Since
$LME_{control}$ does not include volcanic eruptions, agreement with
$LME_{control}$ is confirmation of the null hypothesis.

\subsubsection{Potential Intensity}
In addition to downscaling, we use the original CESM data to compute
potential intensity (PI) fields. This gives us insight into the
theoretical effect of eruptions on hurricanes, without the
computational overhead of downscaling. Following the thermodynamic
analysis in \cite{pi_ke}, we use Equation (\ref{PI_eqn}) to calculate
PI:

\begin{equation}
{V_m} \propto \sqrt{\frac{T_s-T_{o}}{T_{o}}(k^{*}-k)},
\label{PI_eqn}
\end{equation}

where \ref{PI_eqn} $V_m$ is the maximum tangential wind speed, $T_s$
is the temperature at the ocean surface, $T_o$ is the outflow
temperature at the top of the troposphere, and $k^{*}-k$ is the
enthalpy flux (or latent heat flux) at the sea-air interface. The
enthalpy flux is given by $c_p(T_{SST}-T_{air})+L(q^{*}-q)$, where
$c_p$ is the specific heat capacity at constant pressure, $T_{SST}$ is
the sea surface temperature, $T_{air}$ is the temperature of air at
the surface, $L$ is the latent heat of vaporization, $q^{*}$ is the
saturated specific humidity at the surface, and $q$ is the specific
humidity of air at the surface. We are only interested in the relative
difference between $LME_{control}$ and $LME_{forced}$ given by
Equation (\ref{dpi}), so we are unconcerned with additional scaling
factors.

\begin{equation}
\delta PI = \frac{{V_{m}}^{LME_{forced}}}{{V_{m}}^{LME_{control}}}-1
\label{dpi}
\end{equation}


\section{Results}
\label{results}
He we show comparisons between downscaled output from ERAI and IBTrACS as well as between $LME_{forced}$ and $LME_{control}$. The ERAI vs. IBTrACS comparison provides a baseline of absolute accuracy. The comparison between $LME_{forced}$ and $LME_{control}$ is focused specifically on the effect of aerosol forcing from volcanic eruptions. We downscaled 150 consecutive years of $LME_{control}$ and 100 years of $LME_{forced}$ combined from 2 year runs after 50 separate volcanic eruptions. The reconstruction of these eruptions is described in detail in \cite{erups_recon}. 
\par
In addition to the downscaling results, we look at the much less computationally intensive PI analysis. It is possible to compute the PI field for the entirety of the LME simulations, but it is not feasible to do with dynamical downscaling. We present the average potential intensity anomaly for all eruptions and for the strongest eruptions. We also show the PI anomaly for the 10 strongest eruptions and 10 weakest eruptions. PI shows what hurricane behavior should be expected if all hurricanes achieved maximum possible intensity based on thermodynamic conditions of the environment. However, this theoretical intensity is rarely achieved by hurricanes in practice.

\subsection{ERAI vs IBTrACS}
As shown in table \ref{evi_table}, using our suite of diagnostics, we found an overall agreement between ERAI and IBTrACS of ${\sim}86.5\%$, or a composite difference of ${\sim}13.5\%$. $6$-hourly ERAI data downscaled in WRF was compared to IBTrACS for the same time period ($1995-2005$). Diagnostics distributions for both ERAI and IBTrACS are shown in figure \ref{evi_diags}, and tracks for both cases are shown in figure \ref{erai_ibtracs_tracks}. It is worth noting that truncation of the domain in our ERAI simulations contributes to the differences in latitude and longitude peaks seen in figure \ref{evi_diags}.  
\par
We also implemented a rudimentary version of our own track matching algorithm and we saw similar agreement to that in \cite{hodges2017well}. We also saw close agreement comparing results produced by other diagnostics, similar to those described in \ref{diags}. The physics schemes in section \ref{WRF} and threshold values in section \ref{tstorms} were used in response to a self-selected $15\%$ difference threshold imposed between ERAI and IBTrACS, as quantified by our diagnostics suite. 

\subsection{Effect of Eruptions on Hurricane Statistics}
\subsubsection{Average Effect of Eruptions}
In comparing $LME_{control}$ and $LME_{forced}$ we focused on the effect of aerosol forcing from volcanic eruptions. These signals are shown in figure \ref{erups_plot}. We selected 50 eruptions from $LME_{forced}$ and ran WRF for two years after each of the eruptions. $LME_{control}$ was run using WRF for 150 years to give a sufficient sample of natural variability. The $LME_{control}$ run was ensured to have sufficient length by looking at the SST signal in frequency space, as shown in figure \ref{spectrum}. This figure shows no significant low frequency variability is missing from the control sample. The hurricane tracks over 100 years of both $LME_{control}$ and $LME_{forced}$ are shown in figure \ref{forced_ctrl_tracks}.
\par
Distributions of the diagnostics for $LME_{control}$ and $LME_{forced}$, with all 50 eruptions included, are shown in figure \ref{50_erups}. Performing two sample ks-tests on the distributions, along with significance tests on the difference of mean values, shows that the overall effect of all 50 eruptions is consistent with the null hypothesis. That is, the overall effect of all 50 eruptions is consistent with the natural climate variability seen in $LME_{control}$. Figures \ref{ks_all} and \ref{sig_all} show the results of these tests. The ks-tests show a maximum difference between the two samples (D-value), and a probability that the two samples are drawn from the same distribution (P-value). The significance tests show the fraction of the $LME_{control}$ sample which is greater than and less than the mean value of the corresponding $LME_{forced}$ diagnostic. 
\par
Although the aggregate effect of eruptions on hurricanes seems non-significant, we also calculated pearson correlation coefficients on eruption strength and diagnostic changes. The significance of the calculated coefficients can be evaluated by determining the confidence interval for zero correlation. The 90\% confidence interval for zero correlation with all eruptions is $[-0.235,0.235]$. This is the interval of a discrete normal distribution with $N=50$ samples (eruptions) which includes correlation values with probabilities greater than 0.05. Due to the symmetry of the normal distribution the left and right tails outside this interval total a probability of 0.1. Thus, we can say with at least 90\% confidence that yearly number, intensity, and lifetime, correspond with eruption strength. The correlation coefficients are listed in figure \ref{corr_all}.   

\subsubsection{Effect of Strongest Eruptions}
Distributions of diagnostics with only the 10 strongest eruptions
included are shown in figure \ref{10_erups}. Tables showing the
results of ks-tests and significance tests on the strongest eruptions
are in figures \ref{ks_10} and \ref{sig_10}. We see in the
significance table that the $LME_{forced}$ mean values for yearly
number, intensity, and lifetime suggest we should reject the null
hypothesis at only the $70\%-80\%$ confidence limit. These signals
show that for the 10 largest eruptions the yearly number, intensity,
and lifetimes are reduced. Interestingly, the eruptions with the
largest net effects are 1213 (8th strongest) and 1815 (3rd
strongest). This clearly demonstrates that other factors are at work
besides amount of aerosol forcing. Both the 1213 and 1815 eruptions
have ${\sim}13\%$ total average difference from
$LME_{control}$. Tables shows the respective significance tests are
shown in figures \ref{sig_1213} and \ref{sig_1815}.

\subsection{Potential Intensity}
The PI anomalies ($\delta PI$) for the strongest eruptions are shown in figures \ref{pi_10_avg} and \ref{pi_10_str}. $\delta PI$ for the weakest eruptions is shown in figure \ref{pi_10_wk}. Strength of eruptions were determined by the peak aerosol mass, as shown in figure \ref{erups_plot}. The average fractional PI anomaly for all eruptions is shown in figure \ref{pi_all_avg}. The hatching in figure \ref{pi_all_avg} is based on a p-value threshold of 0.01 for a two-sided t-test at each grid point. This analysis was done for figure \ref{pi_10_avg} as well. The disparity between the two figures supports the notion that an effect on hurricanes is observed only for the largest eruptions. The average decrease in PI for the strongest eruptions is ${\sim}2.2\%$. The average fractional PI anomaly for all eruptions is a decrease of ${\sim}1.0\%$.   

\section{Discussion \& Conclusion}
\label{discuss}
In this work we have explored the effect of volcanic eruptions in 
the past millennium on hurricane climatology. To do this we first 
validated our approach of downscaling CESM data with WRF by comparing 
results of ERAI downscaling with IBTrACS data. We also performed a 
parameter search for our cyclone tracking algorithm in order to achieve 
high accuracy and to understand sensitivity to selected parameters. 
We then compared the results of downscaling our control data from CESM 
with forced data from CESM, where we focused on the years in the forced 
data which bounded the volcanic eruptions. 
\par
Here, our results also show volcanic eruptions suppress hurricane intensity by
influencing both the surface temperature and the vertical temperature
profile, although only large eruptions show strong influences on
hurricane intensity. We found that the aggregate effect of eruptions is 
consistent with the null hypothesis: the control case. However, we see evidence 
that sufficiently strong eruptions do result in lower annual hurricane count, 
reduced intensity, and shorter lifetimes. This evidence is in the form of 
KS and significance tests on diagnostic distributions, as well as 
correlations between strength and changes in the mean values of these diagnostics. PI analysis also supports these conclusions. 
\par
By dynamically downscaling the ERAI dataset we were able to generate 
hurricane statistics based on observational climate data. 
IBTrACS provided observational data for hurricanes for roughly the 
same time period as ERAI. Thus, comparing
the downscaled ERAI results to hurricane tracks and intensities from
IBTrACS allowed us to evaluate the accuracy of our approach. 
In ref. \cite{hodges2017well}, the authors assess how well TCs are
represented in reanalysis products. This work used two TC-track
matching approaches, referred to as (1) ``direct matching" and (2)
``objective matching". The authors further used several diagnostics in
order to compare reanalysis TC tracks to those found in IBTrACS. The
objective matching approach, which employs a tracking algorithm
similar to TSTORMS, found an agreement of ${\sim}60\%$ with ERAI in
the Northern Hemisphere. A simple ``direct matching" implementation of
our own achieved similar agreement. Due to the inherently chaotic nature 
of hurricane genesis exact agreement between ERAI and IBTrACS was not expected. 
Assessing our approach was the primary objective in comparing ERAI with 
IBTrACS. ERAI data has resolution on the order of one degree which 
limits the ability to match individual hurricanes through downscaling. 
We expected ERAI to capture the observational record for mean climate 
and to provide good agreement between downscaled results and overall hurricane
statistics seen in IBTrACS.
\par
The main focus of our work was the comparison between downscaled results of the
$LME_{control}$ and $LME_{forced}$ datasets. There has been extensive use of 
PI analysis to assess hurricane behavior in various enironments 
\cite{yan2018divergent,ting2015,Kossin2009,vecchi2007effect}, and PI is sometimes 
prefered for its theoretical and computational simplicity. Our PI results give us a
single snapshot of the thermodynamic environment for a given time period. This
is a useful low cost supplement to higher resolution analysis but
doesn't tell us anything about the yearly number, lifetime, or intensity 
distributions. It also doesn't tell us anything about individual hurricane
tracks and only provides a rough expected spatial distribution. 
\par
Our PI results are in good agreement with those shown in \cite{yan2018divergent,vecchi2007effect}. Volcanic aerosols are seen to reduce SSTs and this is reflected
in reduced PI. However, in ref \cite{wehner2015}, caution is urged in using PI 
to draw strong conclusions about tropical cyclone projections as it fails to 
capture features seen in high-resolution climate models. Dynamical 
downscaling provides far greater detail in both the spatial and temporal domain. 
This is evident when looking at the cyclone track plots and extensive suite of 
diagnostics used to analyze downscaled results. At the cost of increased 
computation time we used dynamical downscaling to overcome some of these 
resolution deficiencies. 
\par
As the anthropogenic climate crisis worsens, global mitigation efforts 
without climate engineering may be insufficient to avoid a 2 degree C 
warming scenario above pre-industrial levels by 2100 
\cite{intergovernmental2018global}. A research agenda was recently 
published to pursue methods of carbon dioxide removal as part of a climate 
geoengineering mitigation initiative \cite{national2018negative}.  However, 
these technologies would need to remove more than 10 GtCO2 per year by 2050 
in addition to a major phase out of fossil fuel energy source 
\cite{united2017emissions}, and would be costly. As a cheaper technology, 
stratospheric aerosol geoengineering (SAG) may need to be implemented 
temporarily in addition to these methods to avoid, for example, an increase 
in the intensity of hurricanes under a warmer climate. 
\par
The results in this study have significant implications for hurricane 
development in a potential future climate under a SAG regime.  Although we 
analyzed the effects of an increase in stratospheric aerosols from volcanic 
eruptions, the results are analogous as to what could occur under a SAG regime. 
The initial PI analysis demonstrated a slight decrease in the maximum potential 
strength of hurricanes across the Atlantic Ocean during the strongest eruptions, 
presumably from a decrease in the efficiency of a hurricane due to an increase 
in upper tropospheric temperature and a decrease in lower tropospheric 
temperature. Although analyses of impacts were once limited by historical 
observation and courser resolution, we were able to evaluate the direct 
influence of many volcanic eruptions on individual hurricanes.  For example, 
the strongest eruptions in the downscaled $LME_{forced}$ experiment produced 
a slight reduction in hurricane frequency, intensity, and lifetime. These 
impacts would be similarly felt if SAG was implemented, removing some 
uncertainty associated with regional changes in tropical cyclone development 
for the Northern Atlantic Ocean. Under a relatively strong SAG regime and 
according to our results, hurricanes would either remain the same or slightly 
decrease in frequency and intensity.
\par
Although our results show moderate correlation between eruption strength and 
certain diagnostic measures, it is not necessarily true that stronger 
eruptions have a larger effect on hurricane statistics. Additionally, research 
has shown large uncertainties in volcanic reconstructions and seasonality of 
volcanic eruptions \cite{schmidt2012climate,schmidt2011climate,stevenson2017role,raible2016tambora}. This presents a direction for further investigation. 
In this vein, one could look at an ensemble of higher resolution GCM simulations 
on one or two of the strongest volcanic eruptions. This eruption profile 
will be simulated both in the climate conditions during the historical eruption 
as well as under future climate change conditions. An ensemble average or 
simulations with perturbed initial conditions will allow us to home in on the 
sole effect of aerosol forcing. This will also allow us to explore the 
question of whether downscaling introduced any unknown biases. An ensemble 
under future climate change conditions will allow us to explore the interplay 
of large aerosol forcing and strong anthropogenic forcing.    
\par


\bibliographystyle{spphys}
\bibliography{hurrclim} 

\clearpage
\newpage

\begin{table}[!tbp]
\centering
\begin{tabular}{lrrr}
\toprule
             Physics Schemes &  Name & Parameter & Value \\ 
\midrule
            (1) Micro-physics &     WSM6 &  mp\_physics & 6 \\  
            (2) PBL &    YSU &  bl\_pbl\_physics &  1 \\    
            (3) Convection &   Kain-Fritsch &  cu\_physics & 1 \\    
            (4) Long-wave radiation &    RRTMG &   ra\_lw\_physics & 4 \\    
            (5) Short-wave radiation &    RRTMG &   ra\_sw\_physics & 4 \\    
            (6) Land surface &   Noah &   sf\_surface\_physics & 2 \\    
            (7) Surface layer &    MM5 &  sf\_sfclay\_physics &  1 \\    
            (8) Ocean &    Mixed-layer &  sf\_ocean\_physics &  1 \\    
\bottomrule
\end{tabular}
\caption{WRF Physics Schemes}
\label{wrf_specs}
\end{table}

\begin{table}[!tbp]
\centering
\begin{minipage}[b]{0.45\textwidth}
\begin{tabular}{lrrr}
\toprule
             Averages &         ERAI &      IBTrACS \\ 
\midrule
            month &     7.65 &     8.24 \\  
       yearly num &    31.27 &    34.82 \\   
              lat &    18.23 &    21.39 \\    
              lon &   -80.72 &   -88.38 \\    
     max wind m/s &    30.26 &    34.87 \\    
    min press hPa &   988.56 &   979.65 \\    
           w-life &    45.37 &    44.69 \\    
           p-life &    44.02 &    52.53 \\    
\bottomrule
\end{tabular}
\end{minipage}
\hfill
\begin{minipage}[b]{0.45\textwidth}
\begin{tabular}{lrrr}
\toprule
             Percents &         ERAI &      IBTrACS \\ 
\midrule


             May-Nov &     0.88 &     0.99 \\    
               0-25N &     0.78 &     0.73 \\   
             100-50W &     0.65 &     0.44 \\   
             0-40m/s &     0.97 &     0.70 \\   
         1020-980hPa &     0.81 &     0.62 \\   
        (w) 0-100hrs &     0.94 &     0.91 \\   
        (p) 0-100hrs &     0.94 &     0.87 \\
 \\

\bottomrule
\end{tabular}
\end{minipage}

\noindent\fbox{\parbox{\textwidth}{%
\centering
total average difference: 0.096\\
total percent difference: 0.18\\
composite difference: 0.135}}
\caption{ERAI vs IBTrACS stats}
\label{evi_table}
\end{table}

\begin{table}[!tbp]
\centering
\begin{minipage}[b]{0.45\textwidth}
\begin{tabular}{lrrr}
\toprule
             KS-Tests & D-value & P-value \\
\midrule
month & 0.004 & 1.0 \\
yearly num & 0.018 & 1.0 \\
lats & 0.036 & 0.9 \\
lons & 0.038 & 0.86 \\
max wind & 0.024 & 1.0 \\
min press & 0.048 & 0.6 \\
w-life & 0.014 & 1.0 \\
p-life & 0.012 & 1.0 \\

\bottomrule
\end{tabular}
\caption{ks-tests with strong eruptions}
\label{ks_10}
\end{minipage}
\hfill
\begin{minipage}[b]{0.45\textwidth}
\begin{tabular}{lrrr}
\toprule
             Sig-Tests & \% greater & \% less \\
\midrule

month & 0.461 & 0.513 \\
yearly num & 0.584 & 0.351 \\
lats & 0.487 & 0.513 \\
lons & 0.318 & 0.682 \\
max wind & 0.773 & 0.227 \\
min press & 0.286 & 0.714 \\
w-life & 0.675 & 0.325 \\
p-life & 0.708 & 0.292 \\

\bottomrule
\end{tabular}
\caption{significance tests with strong eruptions}
\label{sig_10}
\end{minipage}
\end{table}


\begin{table}[!tbp]
\centering
\begin{minipage}[b]{0.45\textwidth}
\begin{tabular}{lrrr}
\toprule
             Sig-Tests & \% greater &  \% less \\

\midrule

month & 0.63 & 0.357 \\
yearly num & 0.812 & 0.169 \\
lats & 0.747 & 0.253 \\
lons & 0.708 & 0.292 \\
max wind & 1.0 & 0.0 \\
min press & 0.0 & 1.0 \\
w-life & 0.896 & 0.104 \\
p-life & 0.981 & 0.019 \\

\bottomrule
\end{tabular}
\caption{sig-tests for 1213 eruption}
\label{sig_1213}
\end{minipage}
\hfill
\begin{minipage}[b]{0.45\textwidth}
\begin{tabular}{lrrr}
\toprule
             Sig-Tests & \% greater &  \% less \\
\midrule

month & 0.513 & 0.481 \\
yearly num & 0.883 & 0.084 \\
lats & 0.325 & 0.675 \\
lons & 0.195 & 0.805 \\
max wind & 0.831 & 0.169 \\
min press & 0.058 & 0.942 \\
w-life & 0.896 & 0.104 \\
p-life & 0.942 & 0.058 \\

\bottomrule
\end{tabular}
\caption{sig-tests for 1815 eruption}
\label{sig_1815}
\end{minipage}
\end{table}

\begin{table}[!tbp]
\centering
\begin{minipage}[b]{0.45\textwidth}
\begin{tabular}{lrrr}
\toprule
             KS-Tests &     D-value &      P-value\\
\midrule

month & 0.0 & 1.0 \\
yearly num & 0.006 & 1.0 \\
lats & 0.004 & 1.0 \\
lons & 0.0 & 1.0 \\
max wind & 0.006 & 1.0 \\
min press & 0.006 & 1.0 \\
w-life & 0.002 & 1.0 \\
p-life & 0.0 & 1.0 \\

\bottomrule
\end{tabular}
\caption{ks-tests with all eruptions}
\label{ks_all}
\end{minipage}
\hfill
\begin{minipage}[b]{0.45\textwidth}
\begin{tabular}{lrrr}
\toprule
             Sig-Tests & \% greater &  \% less \\
\midrule

month & 0.513 & 0.474 \\
yearly num & 0.435 & 0.565 \\
lats & 0.494 & 0.506 \\
lons & 0.455 & 0.545 \\
max wind & 0.519 & 0.474 \\
min press & 0.513 & 0.487 \\
w-life & 0.565 & 0.435 \\
p-life & 0.506 & 0.494 \\

\bottomrule
\end{tabular}
\caption{significance tests with all eruptions}
\label{sig_all}
\end{minipage}
\end{table}

\begin{table}[!tbp]
\centering
\begin{tabular}{lrrr}
\toprule
             Correlation-Tests &     Correlations \\
\midrule

month & -0.1095 \\
yearly num & -0.2305 \\
lats & 0.0201 \\
lons & 0.2199 \\
max wind & -0.3185 \\
min press & 0.2913 \\
w-life & -0.0901 \\
p-life & -0.2753 \\

\bottomrule
\end{tabular}
\caption{correlations with all eruptions}
\label{corr_all}
\end{table}

\begin{figure}[!tbp]
\centering
\includegraphics[width=\textwidth]{./figures/ERAI_vs_IBTRACS_tracks.eps}
\caption{ERAI (A) vs IBTRACS (B) 1995-2005. Here we see good agreement in the location of TC tracks. We note that our WRF domain truncates the ERAI tracks. We also see some underestimation of intensities in downscaled results. Resolvable intensity depends strongly on WRF resolution.}
\label{erai_ibtracs_tracks}
\end{figure}

\begin{figure}[!tbp]
\centering
\includegraphics[width=\textwidth]{./figures/Forced_vs_Control_tracks.eps}
\caption{$LME_{forced}$ all eruptions (A) vs $LME_{control}$ 1000-1100 (B). We see close agreement between forced and control when comparing all simulation years.}
\label{forced_ctrl_tracks}
\end{figure}

\begin{figure}[!tbp]
\centering
\includegraphics[width=\textwidth]{./figures/erai_ibtracs_diags.eps}
\caption{ERAI vs IBTrACS 1995-2005. Here we see good agreement in lifetime and frequency metrics. Location metrics differ mainly due to WRF domain truncation of ERAI tracks. We also see some slight intensity underestimation in ERAI due to WRF resolution.}
\label{evi_diags}
\end{figure}

\begin{figure}[!tbp]
\centering
\includegraphics[width=\textwidth]{./figures/50_erups_dists.eps}
\caption{$LME_{control}$ vs $LME_{forced}$ with all eruptions. Here we see qualitatively similar profiles for each metric. Notable is the frequency reduction in the forced distributions. }
\label{50_erups}
\end{figure}

\begin{figure}[!tbp]
\centering
\includegraphics[width=\textwidth]{./figures/10_erups_dists.eps}
\caption{$LME_{control}$ vs $LME_{forced}$ with strongest eruptions. Here we again see qualitatively similar profiles for each metric. The frequency reduction here is more pronounced than for the comparison with all eruptions. }
\label{10_erups}
\end{figure}

\begin{figure}[!tbp]
\centering
\includegraphics[width=\textwidth]{./figures/PI_diff_50_avg.eps}
\caption{Average PI anomaly for all eruptions. Hatching is based on a p-value threshold of 0.01 for a two-sided t-test at each grid point. We see that any anomalies in the main development region are non-significant and even observe some warming in the North Atlantic.}
\label{pi_all_avg}
\end{figure}

\begin{figure}[!tbp]
\centering
\includegraphics[width=\textwidth]{./figures/PI_diff_10_avg.eps}
\caption{Average PI anomaly for strongest eruptions. All points are below a p-value threshold of 0.01 for a two-sided t-test. Here we see cooling across the Atlantic basin although the main development region sees some warming.}
\label{pi_10_avg}
\end{figure}

\begin{figure}[!tbp]
\centering
\includegraphics[width=\textwidth]{./figures/PI_diff_50.eps}
\caption{PI anomaly for weakest eruptions with plots ordered by ascending eruption strength. Weakest eruptions are those with lowest peak aerosol mass. Here we do not observe any anomalies consistent across all eruptions.}
\label{pi_10_wk}
\end{figure}

\begin{figure}[!tbp]
\centering
\includegraphics[width=\textwidth]{./figures/PI_diff_10.eps}
\caption{PI anomaly for strongest eruptions with plots ordered by ascending eruption strength. Strongest eruptions are those with highest peak aerosol mass. Here we see cooling within the Atlantic basin for all eruptions.}
\label{pi_10_str}
\end{figure}

\begin{figure}[!tbp]
\centering
\begin{minipage}[b]{0.45\textwidth}
\includegraphics[width=\textwidth]{./figures/eruptions_plot.eps}
\caption{Aerosol mass signals for volcanic eruptions 500-2000 C.E. The peak signals shown here are used to determine eruption strength. }
\label{erups_plot}
\end{minipage}
\hfill
\begin{minipage}[b]{0.45\textwidth}
\includegraphics[width=\textwidth]{./figures/power_spectrum.eps}
\caption{$LME_{control}$ SST Power Spectrum. This plot shows that using 100 years of control data is sufficient and in doing so we are not missing any low frequency content.}
\label{spectrum}
\end{minipage}
\end{figure}

%\begin{acknowledgements}
%If you'd like to thank anyone, place your comments here
%and remove the percent signs.
%\end{acknowledgements}


\end{document}
% end of file template.tex
